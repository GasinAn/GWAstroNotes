\chapter{Varying $G$}

\section{Modification of Amplitude}

\begin{equation}
    \p^c\p_c\bar{h}_{ab}=-16\pi\frac{G_0}{c_0^4}T_{ab},\quad\p^a\bar{h}_{ab}=0
\end{equation}
\begin{equation}
    \Gamma^c_{\ph{c}ab}=\frac{1}{2}\eta^{cd}(2\p_{(a}{h}_{b)d}-\p_d{h}_{ab})
\end{equation}
\begin{equation}
    U^a\p_aU^c+\Gamma^c_{\ph{c}ab}U^aU^b=0
\end{equation}
\begin{equation}
    U^a\p_aU^c=-\frac{1}{2}\eta^{cd}(2\p_{(a}{h}_{b)d}-\p_d{h}_{ab})U^aU^b
\end{equation}

\begin{equation}
    T_{ab}=c_0^2(2U_{(a}J_{b)}+U^cJ_cU_aU_b)
\end{equation}
\begin{equation}
    J_bc_0^2=-U^aT_{ab}
\end{equation}

\begin{equation}
    A_b
    =-\frac{1}{4}U^a\bar{h}_{ab}
\end{equation}
\begin{equation}
    A_0
    =-\frac{1}{4}c_0\bar{h}_{00}
    =-\frac{1}{2}c_0(\bar{h}_{00}-\frac{1}{2}\eta_{00}\eta^{00}\bar{h}_{00})=-\frac{1}{2}c_0h_{00}
\end{equation}
\begin{equation}
    A_i=-\frac{1}{4}c_0\bar{h}_{0i}=-\frac{1}{4}c_0{h}_{0i}
\end{equation}

\begin{equation}
    U^\mu\p_\mu U^i=-\frac{1}{2}\eta^{i\sigma}(\p_\mu{h}_{\nu\sigma}+\p_\nu{h}_{\mu\sigma}-\p_\sigma{h}_{\mu\nu})U^\mu U^\nu
\end{equation}
\begin{align}
    -\frac{1}{2}\eta^{i\sigma}(\p_0{h}_{0\sigma}+\p_0{h}_{0\sigma}-\p_\sigma{h}_{00})U^0 U^0
    &=\frac{1}{2}c_0^2\eta^{i\sigma}\p_\sigma{h}_{00}\\
    &=\frac{1}{2}c_0^2\p^i{h}_{00}\\
    &=-c_0\p^iA_0\\
    &=-E^i
\end{align}
\begin{align}
    -\frac{1}{2}\eta^{i\sigma}(\p_0{h}_{j\sigma}+\p_j{h}_{0\sigma}-\p_\sigma{h}_{0j})U^0 U^j
    &=-\frac{1}{2}c_0\eta^{i\sigma}(\p_j{h}_{0\sigma}-\p_\sigma{h}_{0j})v^j\\
    &=-\frac{1}{2}c_0\eta^{ik}(\p_j{h}_{0k}-\p_k{h}_{0j})v^j\\
    &=2\eta^{ik}(\p_jA_k-\p_kA_j)v^j\\
    &=-2\eta^{ik}(\p_kA_j-\p_jA_k)v^j\\
    &=-2(\p^iA_j-\p_jA^i)v^j\\
    &=-2\varepsilon^{i}_{\ph{i}jk}v^jB^k
\end{align}
\begin{align}
    -\frac{1}{2}\eta^{i\sigma}(\p_j{h}_{k\sigma}+\p_k{h}_{j\sigma}-\p_\sigma{h}_{jk})U^j U^k=0
\end{align}
\begin{equation}
    a^i=-E^i-4\varepsilon^{i}_{\ph{i}jk}v^jB^k
\end{equation}

\begin{equation}
    \p^i(\frac{1}{4\pi G_0}E_i)=\rho
\end{equation}
\begin{equation}
    \p^iB_i=0
\end{equation}
\begin{equation}
    \varepsilon^{i}_{\ph{i}jk}\p^jE^k=-\p_tB^i
\end{equation}
\begin{equation}
    \varepsilon^{i}_{\ph{i}jk}\p^j(\frac{c_0^2}{4\pi G_0}B^k)=j^i+\p_t(\frac{1}{4\pi G_0}E^i)
\end{equation}

\begin{equation}
    \varepsilon_{\text{G}0}:=\frac{1}{4\pi G_0},\quad
    \mu_{\text{G}0}:=\frac{4\pi G_0}{c_0^2}
\end{equation}
\begin{equation}
    \begin{cases}
        \vec{\nabla}\cdot(\varepsilon_{\text{G}0}\vec{E})=\rho\\
        \vec{\nabla}\cdot\vec{B}=0\\
        \vec{\nabla}\times\vec{E}=-\frac{\p}{\p t}\vec{B}\\
        \vec{\nabla}\times(\mu_{\text{G}0}^{-1}\vec{B})=\vec{j}+\frac{\p}{\p t}(\varepsilon_{\text{G}0}\vec{E})
    \end{cases}
\end{equation}
\begin{equation}
    \vec{a}=-\vec{E}-4\vec{v}\times\vec{B}
\end{equation}

\begin{equation}
    \varepsilon_{\text{G}}=\frac{1}{4\pi G},\quad
    \mu_{\text{G}}=\frac{4\pi G}{c^2}
\end{equation}
\begin{equation}
    x^\mu=(ct,x,y,z)
\end{equation}
\begin{equation}
    \begin{cases}
        \vec{\nabla}\cdot(\varepsilon_{\text{G}}\vec{E})=\rho\\
        \vec{\nabla}\cdot\vec{B}=0\\
        \vec{\nabla}\times\vec{E}=-\frac{\p}{\p t}\vec{B}\\
        \vec{\nabla}\times(\mu_{\text{G}}^{-1}\vec{B})=\vec{j}+\frac{\p}{\p t}(\varepsilon_{\text{G}}\vec{E})
    \end{cases}
\end{equation}
\begin{equation}
    \vec{a}=-\vec{E}-4\vec{v}\times\vec{B}
\end{equation}
\begin{equation}
    A_\mu=-\frac{1}{4}c_{}\bar{h}_{0\mu}
\end{equation}

\begin{equation}
    \begin{cases}
        \vec{\nabla}\cdot\vec{E}=\varepsilon_{\text{G}}^{-1}\rho\\
        \vec{\nabla}\times\vec{B}=\mu_{\text{G}}\vec{j}+\varepsilon_{\text{G}}\mu_{\text{G}}\frac{\p}{\p t}\vec{E}
    \end{cases}
\end{equation}
\begin{equation}
    \frac{1}{c^2}\frac{\p}{\p t}\varphi+\vec{\nabla}\cdot\vec{A}=0
\end{equation}
\begin{equation}
    \begin{cases}
        -\frac{1}{c^2}\frac{\p^2}{\p t^2}\varphi+\vec{\nabla}^2\varphi=\varepsilon_{\text{G}}^{-1}\rho\\
        -\frac{1}{c^2}\frac{\p^2}{\p t^2}\vec{A}+\vec{\nabla}^2\vec{A}=\mu_{\text{G}}\vec{j}
    \end{cases}
\end{equation}
\begin{equation}
    \begin{cases}
        -\frac{1}{c^2}\frac{\p^2}{\p t^2}c^{-1}\varphi+\vec{\nabla}^2c^{-1}\varphi=\mu_{\text{G}}c\rho\\
        -\frac{1}{c^2}\frac{\p^2}{\p t^2}\vec{A}+\vec{\nabla}^2\vec{A}=\mu_{\text{G}}\vec{j}
    \end{cases}
\end{equation}

\begin{equation}
    \begin{cases}
        \vec{\nabla}\cdot(\varepsilon_{\text{G}}\vec{E})=0\\
        \vec{\nabla}\cdot(\mu_{\text{G}}\vec{H})=0\\
        \vec{\nabla}\times\vec{E}=-\frac{\p}{\p t}(\mu_{\text{G}}\vec{H})\\
        \vec{\nabla}\times\vec{H}=+\frac{\p}{\p t}(\varepsilon_{\text{G}}\vec{E})
    \end{cases}
\end{equation}
\begin{equation}
    {E}_r=0,\quad{H}_r=0
\end{equation}
\begin{equation}
    \begin{cases}
        \frac{\varepsilon_{\text{G}}}{r\sin\theta}\frac{\p}{\p\theta}(\sin\theta{E}_\theta)+\frac{\varepsilon_{\text{G}}}{r\sin\theta}\frac{\p}{\p\phi}({E}_\phi)=0\\
        \frac{\mu_{\text{G}}}{r\sin\theta}\frac{\p}{\p\theta}(\sin\theta{H}_\theta)+\frac{\mu_{\text{G}}}{r\sin\theta}\frac{\p}{\p\phi}({H}_\phi)=0\\
        \frac{1}{r\sin\theta}[\frac{\p}{\p\theta}(\sin\theta{E}_\phi)-\frac{\p}{\p\phi}({E}_\theta)]\vec{e}_r-\frac{1}{r}\frac{\p}{\p r}(r{E}_\phi)\vec{e}_\theta+\frac{1}{r}\frac{\p}{\p r}(r{E}_\theta)\vec{e}_\phi=-\mu_{\text{G}}\frac{\p}{\p t}({H}_\theta\vec{e}_\theta+{H}_\phi\vec{e}_\phi)\\
        \frac{1}{r\sin\theta}[\frac{\p}{\p\theta}(\sin\theta{H}_\phi)-\frac{\p}{\p\phi}({H}_\theta)]\vec{e}_r-\frac{1}{r}\frac{\p}{\p r}(r{H}_\phi)\vec{e}_\theta+\frac{1}{r}\frac{\p}{\p r}(r{H}_\theta)\vec{e}_\phi=+\varepsilon_{\text{G}}\frac{\p}{\p t}({E}_\theta\vec{e}_\theta+{E}_\phi\vec{e}_\phi)
    \end{cases}
\end{equation}
\begin{equation}
    \vec{E}={E}_\theta\vec{e}_\theta,\quad\vec{H}={H}_\phi\vec{e}_\phi
\end{equation}
\begin{equation}
    \begin{cases}
        \frac{\varepsilon_{\text{G}}}{r\sin\theta}\frac{\p}{\p\theta}(\sin\theta E_\theta)=0\\
        \frac{\mu_{\text{G}}}{r\sin\theta}\frac{\p}{\p\phi}({H}_\phi)=0\\
        -\frac{1}{r\sin\theta}\frac{\p}{\p\phi}({E}_\theta)\vec{e}_r+\frac{1}{r}\frac{\p}{\p r}(r{E}_\theta)\vec{e}_\phi=-\mu_{\text{G}}\frac{\p}{\p t}({H}_\phi)\vec{e}_\phi\\
        +\frac{1}{r\sin\theta}\frac{\p}{\p\theta}(\sin\theta{H}_\phi)\vec{e}_r-\frac{1}{r}\frac{\p}{\p r}(r{H}_\phi)\vec{e}_\theta=+\varepsilon_{\text{G}}\frac{\p}{\p t}({E}_\theta)\vec{e}_\theta\\
    \end{cases}
\end{equation}
\begin{equation}
    \begin{cases}
        \frac{\p}{\p r}(r{E}_\theta)+\mu_{\text{G}}\frac{\p}{\p t}(r{H}_\phi)=0\\
        \frac{\p}{\p r}(r{H}_\phi)+\varepsilon_{\text{G}}\frac{\p}{\p t}(r{E}_\theta)=0\\
    \end{cases}
\end{equation}
\begin{equation}
    \begin{cases}
        \mu_{\text{G}}\frac{\p}{\p r}\mu_{\text{G}}^{-1}\frac{\p}{\p r}(r{E}_\theta)-\varepsilon_{\text{G}}\mu_{\text{G}}\frac{\p}{\p t}\frac{\p}{\p t}(r{E}_\theta)=0\\
        \varepsilon_{\text{G}}\frac{\p}{\p r}\varepsilon_{\text{G}}^{-1}\frac{\p}{\p r}(r{H}_\phi)-\varepsilon_{\text{G}}\mu_{\text{G}}\frac{\p}{\p t}\frac{\p}{\p t}(r{H}_\phi)=0\\
    \end{cases}
\end{equation}
\begin{equation}
    \begin{cases}
        \mu_{\text{G}}\frac{\p}{\p r}\mu_{\text{G}}^{-1}\frac{\p}{\p r}(r{E}_\theta)-\frac{\p}{\p(ct)}\frac{\p}{\p(ct)}(r{E}_\theta)=0\\
        \varepsilon_{\text{G}}\frac{\p}{\p r}\varepsilon_{\text{G}}^{-1}\frac{\p}{\p r}(r{H}_\phi)-\frac{\p}{\p(ct)}\frac{\p}{\p(ct)}(r{H}_\phi)=0\\
    \end{cases}
\end{equation}
\begin{equation}
    \begin{cases}
        \frac{\p}{\p r}\frac{\p}{\p r}(r{E}_\theta)-\frac{\p}{\p r}(\ln\mu_{\text{G}})\frac{\p}{\p r}(r{E}_\theta)-\frac{\p}{\p(ct)}\frac{\p}{\p(ct)}(r{E}_\theta)=0\\
        \frac{\p}{\p r}\frac{\p}{\p r}(r{H}_\phi)-\frac{\p}{\p r}(\ln\varepsilon_{\text{G}})\frac{\p}{\p r}(r{H}_\phi)-\frac{\p}{\p(ct)}\frac{\p}{\p(ct)}(r{H}_\phi)=0\\
    \end{cases}
\end{equation}
\begin{equation}
    \frac{\p^2}{\p r^2}f(r,t)-p(r)\frac{\p}{\p r}f(r,t)-\frac{\p^2}{\p (ct)^2}f(r,t)=0
\end{equation}
\begin{equation}
    f(r,t)=f(r)e^{-ikct}
\end{equation}
\begin{equation}
    \frac{\d^2}{\d r^2}f(r)-p(r)\frac{\d}{\d r}f(r)+k^2f(r)=0
\end{equation}
\begin{equation}
    \frac{\d^2}{\d r^2}f(r)-p\frac{\d}{\d r}f(r)+k^2f(r)=0
\end{equation}
\begin{equation}
    f(r)=e^{(p/2)r}[C_+e^{i\sqrt{k^2-(p/2)^2}r}+C_-e^{-i\sqrt{k^2-(p/2)^2}r}]
\end{equation}
\begin{equation}
    f(r,t)=e^{(p/2)r}[C_+e^{i(+\sqrt{k^2-(p/2)^2}r-kct)}+C_-e^{i(-\sqrt{k^2-(p/2)^2}r-kct)}]
\end{equation}
\begin{equation}
    f(r,t)=e^{(p/2)r}[C_+e^{i(+\sqrt{(\omega/c)^2-(p/2)^2}r-\omega t)}+C_-e^{i(-\sqrt{(\omega/c)^2-(p/2)^2}r-\omega t)}]
\end{equation}
\begin{equation}
    f(r,t)=e^{\int(p/2)\d r}[C_+e^{i(+\int\sqrt{(\omega/c)^2-(p/2)^2}\d r-\omega t)}+C_-e^{i(-\int\sqrt{(\omega/c)^2-(p/2)^2}\d r-\omega t)}]
\end{equation}
\begin{equation}
    \begin{cases}
        r_2\left\lvert {E}_\theta\right\rvert_{r=r_2}=r_1\left\lvert {E}_\theta\right\rvert_{r=r_1}e^{\int_{r_1}^{r_2}\frac{1}{2}\frac{\p}{\p r}(\ln\mu_{\text{G}})\,\d r}\\
        r_2\left\lvert {H}_\phi\right\rvert_{r=r_2}=r_1\left\lvert {H}_\phi\right\rvert_{r=r_1}e^{\int_{r_1}^{r_2}\frac{1}{2}\frac{\p}{\p r}(\ln\varepsilon_{\text{G}})\,\d r}
    \end{cases}
\end{equation}
\begin{equation}
    \begin{cases}
        {E}_2=\sqrt{\frac{{\mu_{\text{G}}}_2}{{\mu_{\text{G}}}_1}}\frac{r_1}{r_2}{E}_1\\
        {H}_2=\sqrt{\frac{{\varepsilon_{\text{G}}}_2}{{\varepsilon_{\text{G}}}_1}}\frac{r_1}{r_2}{H}_1
    \end{cases}
\end{equation}
\begin{equation}
    \begin{cases}
        {E}_2/c_2=\sqrt{\frac{{\mu_{\text{G}}}_2}{{\mu_{\text{G}}}_1}}\frac{c_1}{c_2}\frac{r_1}{r_2}{E}_1/c_1\\
        {B}_2=\sqrt{\frac{{\mu_{\text{G}}}_2}{{\mu_{\text{G}}}_1}}\frac{c_1}{c_2}\frac{r_1}{r_2}{B}_1
    \end{cases}
\end{equation}
\begin{equation}
    \begin{cases}
        (\omega/c_2)c_2(\bar{h}_{00})_2=\sqrt{\frac{{\mu_{\text{G}}}_2}{{\mu_{\text{G}}}_1}}\frac{c_1}{c_2}\frac{r_1}{r_2}(\omega/c_1)c_1(\bar{h}_{00})_1\\
        (\omega/c_2)c_2(\bar{h}_{0i})_2=\sqrt{\frac{{\mu_{\text{G}}}_2}{{\mu_{\text{G}}}_1}}\frac{c_1}{c_2}\frac{r_1}{r_2}(\omega/c_1)c_1(\bar{h}_{0i})_1
    \end{cases}
\end{equation}
\begin{equation}
    h_2=\sqrt{\frac{c_1^4/G_1}{c_2^4/G_2}}\frac{r_1}{r_2}h_1
\end{equation}

双星系统引力辐射本为
\begin{equation}
    h=\frac{\mathcal{M}[\pi \mathcal{M}F(t)]^{2/3}}{r}Q(\theta,\phi,\psi,\iota)\cos[\int 2\pi F(t)\,\d t]
\end{equation}
设双星系统常量$c^*$, $G^*$, 一观者临近双星系统且与双星系统相对静止, 其与双星系统距离为$r$, 测得强度$h_r$, 频率$F_r$, 则\footnote{$\mathcal{M}$和$c^*$, $G^*$简并, 所以可以笼统地仍记作$\mathcal{M}$.}
\begin{equation}
    h_r=\frac{\mathcal{M}[\pi \mathcal{M}F_r(t)]^{2/3}}{r/c^*}Q(\theta,\phi,\psi,\iota)
\end{equation}

设地球观者与双星系统距离为$d$, 双星系统红移为$z$, 测得强度$h_d$, 频率$F_d=F_r/(1+z)$, 则
\begin{align}
    h_d&=\sqrt{\frac{{c^*}^4/G^*}{c^4/G}}\frac{r}{d}h_r\\
    &=\sqrt{\frac{{c^*}^4/G^*}{c^4/G}}\frac{\mathcal{M}[\pi \mathcal{M}F_r(t)]^{2/3}}{d/c^*}Q(\theta,\phi,\psi,\iota)
\end{align}
所以地球观者测得
\begin{equation}
    h=\sqrt{\frac{{c^*}^4/G^*}{c^4/G}}\frac{\mathcal{M}[\pi \mathcal{M}F_r(t)]^{2/3}}{d/c^*}Q(\theta,\phi,\psi,\iota)\cos[\int 2\pi \frac{F_r(t)}{1+z}\,\d t]
\end{equation}
记$F_\text{obs}(t)=F_r(t)/(1+z)$, $\mathcal{M}_\text{obs}=\mathcal{M}(1+z)$, 光度距离$d_\text{L}=d(1+z)$, 则
\begin{align}
    h&=\sqrt{\frac{{c^*}^4/G^*}{c^4/G}}\frac{\mathcal{M}[\pi \mathcal{M}F_r(t)]^{2/3}}{d(1+z)/c^*}Q(\theta,\phi,\psi,\iota)\cos[\int 2\pi F_\text{obs}(t)\,\d t]\\
    &=\sqrt{\frac{{c^*}^4/G^*}{c^4/G}}\frac{\mathcal{M}_\text{obs}[\pi \mathcal{M}_\text{obs}F_\text{obs}(t)]^{2/3}}{d_\text{L}/c^*}Q(\theta,\phi,\psi,\iota)\cos[\int 2\pi F_\text{obs}(t)\,\d t]\\
    &=\sqrt{\frac{{c^*}^6/G^*}{c^6/G}}\frac{\mathcal{M}_\text{obs}[\pi \mathcal{M}_\text{obs}F_\text{obs}(t)]^{2/3}}{d_\text{L}/c}Q(\theta,\phi,\psi,\iota)\cos[\int 2\pi F_\text{obs}(t)\,\d t]
\end{align}
用引力波测距测得$d_\text{L,G}$, 则
\begin{equation}
    d_\text{L,G}=d_\text{L}\sqrt{\frac{c^6/G}{{c^*}^6/G^*}}
\end{equation}

\cite{Poisson1995}
\begin{equation}
    h(t)=\frac{\mathcal{M}[\pi \mathcal{M}F(t)]^{2/3}}{\xi\,d_\text{L}}Q(\text{angles})\cos\Phi(t)
\end{equation}
\begin{equation}
    \tilde{h}(f)=\frac{\sqrt{30}}{48\pi^{2/3}}\frac{\mathcal{M}^{5/6}Q}{\xi\,d_\text{L}}f^{-7/6}e^{i[2\pi ft(f)-\Phi(f)-\frac{\pi}{4}]}
\end{equation}
问题转化为估计$\xi$
\begin{equation}
    p(\mu)\propto p^{(0)}(\mu)\exp[
        -\frac{1}{2}\Gamma_{ab}(\mu^a-\hat{\mu}^a)(\mu^b-\hat{\mu}^b)
    ]
\end{equation}
\begin{equation}
    p^{(0)}(\mu)\propto\exp[
        -\frac{1}{2}\Gamma^{(0)}_{ab}(\mu^a-\bar{\mu}^a)(\mu^b-\bar{\mu}^b)
    ]
\end{equation}
设待估参数为$\mu=(\ln\xi,\ln (d_\text{L}/{d_\text{L}}_0),\ln Q,\dots)$, $\dots$为其他参数(如$\mathcal{M}$), 则$\tilde{h}_{,\ln\xi}=\tilde{h}_{,\ln (d_\text{L}/{d_\text{L}}_0)}=-\tilde{h}_{,\ln Q}=-\tilde{h}$, $\tilde{h}$对其他参数求偏导皆为纯虚数, 则由$\Gamma_{ab}=\left\langle h_{,a}|h_{,b}\right\rangle $和$\text{SNR}:=\rho=\sqrt{\left\langle h|h\right\rangle}$得
\begin{equation}
    \Gamma_{ab}=\begin{bmatrix}
        \rho^2&\rho^2&-\rho^2&0&\dots\\
        \rho^2&\rho^2&-\rho^2&0&\dots\\
        -\rho^2&-\rho^2&\rho^2&0&\dots\\
        0&0&0&?&\ldots\\
        \vdots&\vdots&\vdots&\vdots&\ddots 
    \end{bmatrix}
\end{equation}
又设
\begin{equation}
    \Gamma^{(0)}_{ab}=\begin{bmatrix}
        0&0&0&0&\dots\\
        0&1/\sigma_{\ln d_\text{L}}^2&0&0&\dots\\
        0&0&1/\sigma_{\ln Q}^2&0&\dots\\
        0 &0&0&0&\ldots\\
        \vdots&\vdots&\vdots&\vdots&\ddots 
    \end{bmatrix}
\end{equation}
则由$\Sigma_{ab}=(\Gamma^{(0)}_{ab}+\Gamma_{ab})^{-1}$得
\begin{equation}
    \Sigma_{ab}=\begin{bmatrix}
        \begin{bmatrix}
            \rho^2&\rho^2&-\rho^2\\
            \rho^2&\rho^2+1/\sigma_{\ln (d_\text{L}/{d_\text{L}}_0)}^2&-\rho^2\\
            -\rho^2&-\rho^2&\rho^2+1/\sigma_{\ln Q}^2
        \end{bmatrix}^{-1}&0\\
        0&[?]^{-1}
    \end{bmatrix}
\end{equation}
而
\begin{align}
    &\begin{bmatrix}
        \rho^2&\rho^2&-\rho^2\\
        \rho^2&\rho^2+1/\sigma_{\ln (d_\text{L}/{d_\text{L}}_0)}^2&-\rho^2\\
        -\rho^2&-\rho^2&\rho^2+1/\sigma_{\ln Q}^2
    \end{bmatrix}^{-1}\\
    &=\begin{bmatrix}
        1/\rho^2+\sigma_{\ln (d_\text{L}/{d_\text{L}}_0)}^2+\sigma_{\ln Q}^2&-\sigma_{\ln (d_\text{L}/{d_\text{L}}_0)}^2&\sigma_{\ln Q}^2\\
        -\sigma_{\ln (d_\text{L}/{d_\text{L}}_0)}^2&\sigma_{\ln (d_\text{L}/{d_\text{L}}_0)}^2&0\\
        \sigma_{\ln Q}^2&0&\sigma_{\ln Q}^2
    \end{bmatrix}
\end{align}

\section{Modification of Phase}

\begin{equation}
    \frac{d^2}{d z^2}H(z)+2p(z)\frac{d}{d z}H(z)+\left[\omega^2+q(z)\right]H(z)=0.\label{dessoleq}
\end{equation}
\begin{equation}\label{HAphi}
    H=Ae^{i\Phi}.
\end{equation}
$k=\frac{d \Phi}{d z}$,
\begin{equation}\label{rpart}
    \frac{d^2 A}{d z^2}+2p\frac{d A}{d z}+\left[\omega^2\left(1-\frac{k^2}{\omega^2}\right)+q\right]A=0,
\end{equation}
\begin{equation}\label{ipart}
    2\frac{d A}{d z}k+A\frac{d k}{d z}+2pAk=0,
\end{equation}
\begin{equation}
    2\frac{1}{A}\frac{d A}{d z}+\frac{1}{k}\frac{d k}{d z}+2p=0,
\end{equation}
\begin{equation}\label{Apk}
    A\propto e^{-\int p\,dz}k^{-1/2}.
\end{equation}
$\Gamma=e^{\int p \,dz}$ and $K=(k/\omega)^{-1/2}$,
\begin{equation}\label{equK0}
    \frac{d^2 K}{d z^2}-\left(\frac{1}{\Gamma}\frac{d^2\Gamma}{d z^2}-q\right)K+\omega^2K(1-K^{-4})=0,
\end{equation}
$\Xi=\frac{1}{\Gamma}\frac{d^2\Gamma}{d z^2}-q$ and make $\omega=1$,
\begin{equation}\label{equK}
    \frac{d^2 K}{d z^2}+K[(1-\Xi)-K^{-4}]=0.
\end{equation}
$\Xi=\text{const}$,
\begin{equation}\label{K0}
    K=(1-\Xi)^{-1/4}=1+\frac{1}{4}\Xi+\frac{5}{32}\Xi^2+O(\Xi^3),
\end{equation}
\begin{equation}
    k=(1-\Xi)^{1/2}=1-\frac{1}{2}\Xi-\frac{1}{8}\Xi^2+O(\Xi^3),
\end{equation}
$\Xi\neq\text{const}$, $\Xi(z)=\kappa^2\tilde{\Xi}(\tilde{z})$, where $\tilde{z}=\kappa z$.
\begin{equation}\label{equKs}
    K^3\frac{d^2 K}{d \tilde{z}^2}\kappa^2-K^4\tilde{\Xi}(\tilde{z})\kappa^2+K^4-1=0.
\end{equation}
\begin{equation}\label{K}
    K=\sum_{n=0}^\infty K_n(\tilde{z})\kappa^{2n},
\end{equation}
\begin{gather}
    K_0^4-1=0,\\
    K_0^3K_0''-K_0^4\tilde{\Xi}+4K_0^3K_1=0,\\
    (K_0^3K_1''+3K_0^2K_1K_0'')-4K_0^3K_1\tilde{\Xi}+(4K_0^3K_2+6K_0^2K_1^2)=0.
\end{gather}
\begin{gather}
    K_0=1,\\
    K_1=\frac{1}{4}\tilde{\Xi},\\
    K_2=\frac{5}{32}\tilde{\Xi}^2-\frac{1}{16}\frac{d^2\tilde{\Xi}}{d\tilde{z}^2},
\end{gather}

\begin{equation}
    h(z,t)=h(z)e^{-i\omega t}
\end{equation}
\begin{equation}
    h(z)=\Gamma^{-1}(z)K(z)(C_+e^{+i\omega\int K^{-2}(z)\,\d z}+C_-e^{-i\omega\int K^{-2}(z)\,\d z})
\end{equation}
\begin{equation}
    h(z,t)=\int_{-\infty}^{+\infty}\tilde{h}(z;f)e^{-i2\pi f t}\,\d f
\end{equation}
\begin{equation}
    \tilde{h}(z;f)=\Gamma^{-1}(z)K(z;f)[C_+(f)e^{+i2\pi f\int K^{-2}(z;f)\,\d z}+C_-(f)e^{-i2\pi f\int K^{-2}(z;f)\,\d z}]
\end{equation}
\begin{equation}
    \tilde{h}(z;f)=\Gamma^{-1}(z)K(z;f)[C_+(f)e^{+i2\pi f\int_0^z K^{-2}(z';f)\,\d z'}+C_-(f)e^{-i2\pi f\int_0^z K^{-2}(z';f)\,\d z'}]
\end{equation}
\begin{equation}
    \tilde{h}_0(z;f)=\Gamma^{-1}(0)K(0;f)[C_+(f)e^{+i2\pi f\int_0^z\,\d z'}+C_-(f)e^{-i2\pi f\int_0^z\,\d z'}]
\end{equation}
\begin{equation}
    \tilde{h}(z;f)=\frac{\Gamma^{-1}(z)K(z;f)}{\Gamma^{-1}(0)K(0;f)}[C_+(f)e^{+i2\pi f\int_0^z K^{-2}(z';f)\,\d z'}+C_-(f)e^{-i2\pi f\int_0^z K^{-2}(z';f)\,\d z'}]
\end{equation}
\begin{equation}
    \tilde{h}_0(z;f)=[C_+(f)e^{+i2\pi f\int_0^z\,\d z'}+C_-(f)e^{-i2\pi f\int_0^z\,\d z'}]
\end{equation}
\begin{equation}
    \tilde{h}_0(z;f)=C_+(f)e^{+i2\pi fz}+C_-(f)e^{-i2\pi fz}
\end{equation}
\begin{equation}
    \p_z\tilde{h}_0(z;f)=C_+(f)(i2\pi f)e^{+i2\pi fz}-C_-(f)(i2\pi f)e^{-i2\pi fz}
\end{equation}
\begin{equation}
    C_+(f)e^{+i2\pi fz}=\frac{1}{2}[\tilde{h}_0(z;f)+\p_z\tilde{h}_0(z;f)(i2\pi f)^{-1}]
\end{equation}
\begin{equation}
    C_-(f)e^{-i2\pi fz}=\frac{1}{2}[\tilde{h}_0(z;f)-\p_z\tilde{h}_0(z;f)(i2\pi f)^{-1}]
\end{equation}
\begin{equation}
    \tilde{h}(z;f)=\frac{\Gamma^{-1}(z)K(z;f)}{\Gamma^{-1}(0)K(0;f)}C_+(f)e^{+i2\pi f\int_0^z K^{-2}(z';f)\,\d z'}
\end{equation}
\begin{equation}
    \tilde{h}_0(z;f)=C_+(f)e^{+i2\pi f\int_0^z\,\d z'}
\end{equation}
\begin{equation}
    K(z)=1+\frac{1}{4\omega^2}\Xi(z)
\end{equation}
\begin{equation}
    \tilde{h}(z;f)=\frac{\Gamma^{-1}(z)[1+\frac{\Xi(z)}{4(2\pi f)^2}]}{\Gamma^{-1}(0)[1+\frac{\Xi(0)}{4(2\pi f)^2}]}e^{+i(2\pi f)\int_0^z -\frac{\Xi(z')}{2(2\pi f)^2}\,\d z'}\tilde{h}_0(z;f)
\end{equation}
\begin{equation}
    \tilde{h}(z;f)=\frac{\Gamma^{-1}(z)}{\Gamma^{-1}(0)}[1+\frac{\Xi(z)-\Xi(0)}{4}(2\pi f)^{-2}]e^{-i(2\pi f)\int_0^z\frac{\Xi(z')}{2}(2\pi f)^{-2}\,\d z'}\tilde{h}_0(z;f)
\end{equation}
\begin{equation}
    \tilde{h}(f)=\gamma(1+\xi f^{-2})e^{i\Omega f^{-1}}\tilde{h}_0(f)
\end{equation}

\begin{equation}
    h(z,t)=\frac{1}{2\pi}\int_{-\infty}^{+\infty}\tilde{h}(z;\omega)e^{-i\omega t}\,\d \omega
\end{equation}
\begin{equation}
    \tilde{h}(z;\omega)=\Gamma^{-1}(z)K(z;\omega)[C_+(\omega)e^{+i\omega\int K^{-2}(z;\omega)\,\d z}+C_-(\omega)e^{-i\omega\int K^{-2}(z;\omega)\,\d z}]
\end{equation}
\begin{equation}
    h(z,t)=\frac{1}{2\pi}\int_{-\infty}^{+\infty}\hat{h}(t;k)e^{+ikz}\,\d k
\end{equation}
\begin{equation}
    \hat{h}(t;k)=\Gamma^{-1}(t)K(t;k)[C_+(k)e^{+ik\int K^{-2}(t;k)\,\d t}+C_-(k)e^{-ik\int K^{-2}(t;k)\,\d t}]
\end{equation}

\section{Relationship to LVK}

\cite{Abbott2021a}, $\varphi_{2} \to \varphi_{2}(1+\delta\hat{\varphi}_{2})$,
\begin{equation}
    \tilde{\Omega}=\frac{3\varphi_{2}\delta\hat{\varphi}_{2}}{128\pi\eta}
\end{equation}
\begin{equation}
    \delta\hat{\varphi}_{2} = \frac{128\pi\eta\tilde{\Omega}}{3\varphi_{2}}
\end{equation}

\section{Hierarchical combination}

\cite{Abbott2021}, For a given beyond-GR parameter $x$, this distribution $p(x|d)$ is the expectation for $x$ after marginalizing over the hyperparameters $\mu$ and $\sigma$,
\begin{equation}
    p(x|d)=\int p(x|\mu,\sigma)p(\mu,\sigma|d)\,\d\mu\d\sigma,
\end{equation}
where $d$ represents the data for \emph{all} detected events, and $p(x|\mu,\sigma)\sim\mathcal{N}(\mu,\sigma)$ by construction.
