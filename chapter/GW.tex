\chapter{引力波}\label{GW}

\section{Linearized Gravity}\label{Linearized_Gravity}

\cite{Wald1984}. 流形$\mathbb{R}^{4}$. 任意坐标系$\{x^{\mu}\}$, $g_{\mu\nu}=\eta_{\mu\nu}+h_{\mu\nu}=\eta_{\mu\nu}+\gamma_{\mu\nu}s+\text{O}(s^2)$. 设$g^{\mu\nu}={?}^{\mu\nu}+{??}^{\mu\nu}s+\text{O}(s^2)$, 则$\delta^{\mu}_{\ph{\mu}\lambda}={?}^{\mu\nu}\eta_{\nu\lambda}+{?}^{\mu\nu}\gamma_{\nu\lambda}s+{??}^{\mu\nu}\eta_{\nu\lambda}s+\text{O}(s^2)$, 所以${?}^{\mu\nu}=\eta^{\mu\nu}$, ${??}^{\mu\nu}={??}^{\mu\sigma}\delta_{\sigma}^{\ph{\sigma}\nu}={??}^{\mu\sigma}\eta_{\sigma\lambda}\eta^{\lambda\nu}=-{?}^{\mu\sigma}\gamma_{\sigma\lambda}\eta^{\lambda\nu}=-\eta^{\mu\sigma}\gamma_{\sigma\lambda}\eta^{\lambda\nu}=-\gamma^{\mu\nu}$, 所以$g^{\mu\nu}=\eta^{\mu\nu}-\gamma^{\mu\nu}s+\text{O}(s^2)=\eta^{\mu\nu}-h^{\mu\nu}+\text{O}(s^2)$.
\begin{equation}
    R_{\mu\lambda\nu\sigma}=\p_\sigma\p_{[\mu}h_{\lambda]\nu}-\p_\nu\p_{[\mu}h_{\lambda]\sigma}+\text{O}(s^2).
\end{equation}

$\bar{h}_{\mu\nu}:=h_{\mu\nu}-\frac{1}{2}\eta_{\mu\nu}\eta^{\lambda\sigma}h_{\lambda\sigma}=h_{\mu\nu}-\frac{1}{2}\eta_{\mu\nu}h$.
\begin{equation}
    -\frac{1}{2} \partial^{\lambda} \partial_{\lambda} \bar{h}_{\mu \nu}+\partial^{\lambda} \partial_{(\mu} \bar{h}_{\nu) \lambda}-\frac{1}{2} \eta_{\mu \nu} \partial^{\lambda} \partial^{\sigma} \bar{h}_{\lambda \sigma}+\text{O}(s^2)=8 \pi T_{\mu \nu}.
\end{equation}
存在$\{x^{\mu}\}$, 使得$\p^{\nu}\bar{h}_{\mu\nu}+\text{O}(s^2)=0$ (Lorentz gauge). [证: 设$x'^\mu=x^\mu-\xi^\mu=x^\mu-\zeta^\mu s-\text{O}(s^2)$, 则$\frac{\p ?}{\p x'^\mu}=\frac{\p ?}{\p x^\lambda}\frac{\p x^\lambda}{\p x'^\mu}=\frac{\p ?}{\p x^\lambda}(\delta^{\lambda}_{\ph{\lambda}\mu}+\frac{\p \xi^\lambda}{\p x'^\mu})=\frac{\p ?}{\p x^\nu}+\text{O}(s^2)$, $g'_{\mu\nu}=g_{\lambda\sigma}\frac{\p x^\lambda}{\p x'^\mu}\frac{\p x^\sigma}{\p x'^\nu}=g_{\lambda\sigma}(\delta^{\lambda}_{\ph{\lambda}\mu}+\frac{\p \xi^\lambda}{\p x'^\mu})(\delta^{\sigma}_{\ph{\sigma}\nu}+\frac{\p \xi^\sigma}{\p x'^\nu})=g_{\mu\nu}+g_{\mu\sigma}\frac{\p \xi^\sigma}{\p x'^\nu}+g_{\lambda\nu}\frac{\p \xi^\lambda}{\p x'^\mu}=g_{\mu\nu}+(\eta_{\mu\sigma}+\text{O}(s))(\frac{\p \xi^\sigma}{\p x^\nu}+\text{O}(s^2))+(\eta_{\lambda\nu}+\text{O}(s))(\frac{\p \xi^\lambda}{\p x^\mu}+\text{O}(s^2))=g_{\mu\nu}+\p_\mu\xi_\nu+\p_\nu\xi_\mu+\text{O}(s^2)$, 所以$h'_{\mu\nu}=g'_{\mu\nu}-\eta_{\mu\nu}=g_{\mu\nu}-\eta_{\mu\nu}+\p_\mu\xi_\nu+\p_\nu\xi_\mu+\text{O}(s^2)=h_{\mu\nu}+\p_\mu\xi_\nu+\p_\nu\xi_\mu+\text{O}(s^2)$, 因此存在$\xi^\mu$, 使得$\p'^{\nu}\bar{h}'_{\mu\nu}+\text{O}(s^2)=0$.] 令$\{x^{\mu}\}$满足$\p^{\nu}\bar{h}_{\mu\nu}+\text{O}(s^2)=0$, 则
\begin{equation}
    \p^{\lambda}\p_{\lambda}\bar{h}_{\mu\nu}+\text{O}(s^2)=-16\pi T_{\mu \nu}.\label{lin_gravity}
\end{equation}
略去$\text{O}(s^2)$条件: $h_{\mu\nu}$, $\p_\lambda h_{\mu\nu}$\dots{}小. 下略$\text{O}(s^2)$.

Lorentz gauge等价于协和坐标条件.

\section{Radiation Gauge}

\cite{Wald1984}. 存在$\{x^{\mu}\}$, 使得``无源处'' $h=0\Ra\bar{h}_{\mu\nu}=h_{\mu\nu}$ (TT gauge \cite{王运永2020})且$h_{0\mu}=0$. \cite{Maggiore2014}, 解$\p^{\lambda}\p_{\lambda}\bar{h}_{ij}=0$得$h_{ij}=A_{ij}(\vec{k})e^{ik^\mu x_\mu}$ ($A_{ij}$称为polarization tensor). $h_{(ij)}=0$, $h=0$, $\p^{j}h_{ij}=0$ $\Ra$ $A_{(ij)}=0$, $A=0$, $k^{j}A_{ij}=0$. 令$\vec{e}_z\parallel\vec{k}$,
\begin{equation}
    h_{xy}=\begin{bmatrix}
        +h_+&h_\times\\
        h_\times&-h_+
    \end{bmatrix}e^{i\omega(t-z)}.\label{h_xy}
\end{equation}

\cite{Maggiore2014}. Lorentz gauge $\to$ radiation gauge, $P_{ij}:=\delta_{ij}-n_in_j$, $\Lambda_{ijkl}=P_{ik}P_{jl}-\frac{1}{2}P_{ij}P_{kl}$, $h_{ij}^\text{r}=\Lambda_{ijkl}h_{kl}^\text{L}=\Lambda_{ijkl}\bar{h}_{kl}^\text{L}$. \cite{Sathyaprakash2009}. Step 1: 坐标系空间旋转, 使$\vec{e}_z\parallel\vec{n}$. Step 2: 取$x$, $y$分量$h_{xy}$. Step 3: 去迹. [$h_+=\frac{1}{2}(h_{xx}-h_{yy})$, $h_\times=h_{xy}=h_{yx}$.]

$x'^\mu=x^\mu-\xi^\mu$, $h'_{\mu\nu}=h_{\mu\nu}+\p_\mu\xi_\nu+\p_\nu\xi_\mu$, $h_{\mu\nu}=h'_{\mu\nu}-\p'_\mu\xi_\nu-\p'_\nu\xi_\mu$.

辩曰: 令
\begin{equation}
    \xi=\frac{1}{2}\begin{bmatrix}
        +h_+&h_\times\\
        h_\times&-h_+
   \end{bmatrix}\begin{bmatrix}
    x\\y
    \end{bmatrix}e^{i\omega(t-z)},
\end{equation}
则式 \eqref{h_xy} 中$h_{xy}$为$0$, 岂非无波欤? 对曰: 诚如是, 然$h_{01}$, $h_{02}$, $h_{31}$, $h_{32}$非为$0$, 故不可谓无波也.

另可考\cite{Jaranowski2009}.

\section{Fourier Transformation}

\cite{Maggiore2014}. 
\begin{equation}
    h_{ij}=\frac{1}{(2\pi)^3}\int\d^3\vec{k}\left[\A_{ij}(\vec{k})e^{+ik_\mu x^\mu}+\A_{ij}^*(\vec{k})e^{-ik_\mu x^\mu}\right]
\end{equation}
$\d^2\vec{n}:=\sin\theta\d\theta\d\phi$,
\begin{align}
    h_{ij}&=\int_0^\infty\d f\, f^2\int\d^2\vec{n}\left[\A_{ij}(f,\vec{n})e^{-2\pi i f(t-\vec{n}\cdot\vec{x})}+\text{c.c.}\right]\\
    &=\int_0^\infty\d f\left[e^{-2\pi i ft}f^2\int\d^2\vec{n}\,\A_{ij}(f,\vec{n})e^{+2\pi i f\vec{n}\cdot\vec{x}}+\text{c.c.}\right]\\
    &:=\int_0^\infty\d f\left[\tilde{h}_{ij}(f,\vec{x})e^{-2\pi i ft}+\tilde{h}_{ij}^*(f,\vec{x})e^{+2\pi i ft}\right]\\
    &:=\int\d f\,\tilde{h}_{ij}(f,\vec{x})e^{-2\pi i ft}.
\end{align}
When we observe on Earth a GW emitted by a single astrophysical source, and the linear dimensions of the detector are much smaller than wavelength of the GW, choosing the origin of the coordinate system centered on the detector, $\tilde{h}_{ij}(f,\vec{x})\approx\tilde{h}_{ij}(f):=\tilde{h}_{ij}(f,\vec{x}=\vec{0})$,
\begin{align}
    h_{ij}&=\int\d f\,\tilde{h}_{ij}(f)e^{-2\pi i ft}.
\end{align}
The dependence on $\vec{x}$ must be kept in some cases (see \cite{Maggiore2014}).

\section{TT frame}

TT gauge $\Ra$ TT frame. \cite{Maggiore2014}, free test body $x^\mu(\tau)$, $\frac{\d x^i}{\d \tau}|_{\tau=0}=0$ $\Ra$ $\frac{\d x^i}{\d \tau}\equiv0$ and $\frac{\d x^0}{\d \tau}\equiv1$. $\frac{\d x^i}{\d t}|_{t=0}=0$可得相似结论.

设一测试体在$(0,0,0)$, 另一测试体在$(\Delta x^1,\Delta x^2,\Delta x^3)$, 定义$(\Delta x)^2=\delta_{ij}\Delta x^i\Delta x^j$, 则$(\Delta s)^2=g_{ij}\Delta x^i\Delta x^j=(\Delta x)^2(1+h_{ij}\frac{\Delta x^i}{\Delta x}\frac{\Delta x^j}{\Delta x})$, $\Delta s\approx\Delta x(1+\frac{1}{2}h_{ij}\frac{\Delta x^i}{\Delta x}\frac{\Delta x^j}{\Delta x})$, $\Delta \ddot{s}\approx\frac{1}{2}\ddot{h}_{ij}\frac{\Delta x^i}{\Delta x}\frac{\Delta x^j}{\Delta x}\Delta x$. 定义$n^i=\frac{\Delta x^i}{\Delta x}$, 则$\Delta \ddot{s}\approx n^i(\frac{1}{2}\ddot{h}_{ij}\Delta x^j)$. 定义$\Delta s^i=\Delta s\,n^i$, 则$\Delta s=\Delta s\,n^in_i=\Delta s^in_i=n^i\Delta s_i$, 则$\Delta s_i\approx \frac{1}{2}\ddot{h}_{ij}\Delta x^j\approx\frac{1}{2}\ddot{h}_{ij}\Delta s^j$.

\section{Proper detector frame}

设一基准测试体, 取其固有坐标系, 另一测试体世界线$x^{i}(t)$. \cite{Jaranowski2009}, Let us now imagine that for times $t \le 0$ there were no waves ($h^\text{TT}_{ij} = 0$) in the vicinity of the two observers and that the observers were at rest with respect to each other before the wave has come, so $x^i(t) = x^i_0 = \text{const.}$, $\d x^i/\d t = 0$, for $t \le 0$. At $t = 0$ some wave arrives. Then according to geodesic deviation equation,
\begin{equation}
    \frac{\d^2x^i}{\d t^2}=\frac{1}{2}\frac{\p^2h^\text{TT}_{ij}}{\p t^2}x^j_0,
\end{equation}
\begin{equation}
    x^i=(\delta_{ij}+\frac{1}{2}h^\text{TT}_{ij})x^j_0.
\end{equation}

考虑对观测器材有影响的其他效应, 见\cite{Maggiore2014}, \cite{Ni1978}.
