\chapter{频段}

\section{波源}

\cite{王2020,Schutz1999,Cutler2002,Riles2013}

\section{探测器}

\cite{王2020,Ni2010}

(i) 超高频段 (> 1 THz): 检测方法包括THz共振器, 光学共振器以及尚未发明的巧妙方法. 引力波, 暗能量和膨胀. 

(ii) 甚高频段 (100 kHz--1 THz): 微波共振器/波导探测器, 光学干涉仪和Gaussian光束探测器对该频段敏感. 

(iii) 高频段 (10 Hz--100 kHz): 低温共振器和激光干涉地面探测器对该频段最敏感. 

(iv) 中频段 (0.1 Hz--10 Hz): 空间干涉仪探测器 (臂长1000--100000 km). 

(v) 低频段 (100 nHz--0.1 Hz): 激光干涉空间探测器对该频段最敏感. 

(vi) 极低频段 (300 pHz--100 nHz): 脉冲星计时观测对该频段最敏感. 

(vii) 超低频段 (10 fHz--300 pHz): 类星体自行的天文观测对该频段最敏感. 

(viii) 极低 (Hubble) 频段 (1 aHz--10 fHz): 宇宙微波背景实验对该频段最敏感.

(ix) 超越Hubble频率带 (< 1Hz): 暴涨宇宙学模型给出了该频带内引力波的强度. 可以通过验证暴涨宇宙学模型间接验证这些引力波的存在. 
