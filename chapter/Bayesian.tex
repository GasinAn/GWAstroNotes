\chapter{Bayesian 统计}

[\href{https://andrewwang.rbind.io/courses/bayesian_statistics/}{(Shujia Wang, 贝叶斯统计及应用)}, \href{https://andrewwang.rbind.io/courses/bayesian_statistics/}{(kausiujik, 统计小站: 贝叶斯分析)}]

\section{Bayesian 公式}

\begin{equation}
    p(\vec{\theta}|\vec{x})=\frac{\mathcal{L}(\vec{\theta}|\vec{x})\pi(\vec{\theta})}{\mathcal{Z}}.
\end{equation}
$p(\vec{\theta}|\vec{x})$ 是 posterior, $\pi(\vec{\theta})$ 是 prior, $\mathcal{L}(\vec{\theta}|\vec{x})=p(\vec{x}|\vec{\theta})$ 是 likelihood, $\mathcal{Z}=\int\mathcal{L}(\vec{\theta}|\vec{x})\pi(\vec{\theta})\,\d\vec{\theta}$ 是 evidence.

\section{点估计}

$\hat{\theta}=\t{E}[p(\vec{\theta}|\vec{x})]$ 使 $\t{E}[p(\vec{\theta}-\hat{\theta}|\vec{x})]=\t{D}[p(\vec{\theta}|\vec{x})]$ 最小.

\section{区间估计}

若存在 $C\in\{\theta\}$, 使得 $\int_{C}\p(\theta|\vec{x})\ge1-\alpha$, 则称 $[\hat{\theta}_\t{L},\hat{\theta}_\t{U}]$ 为 $\theta$ 的可信水平为 $1-\alpha$ 的 Bayesian 可信集. 警告: Bayesian 可信区间不是经典置信区间.

若存在 $C\in\{\theta\}$, 使得 $\int_{C}\p(\theta|\vec{x})=1-\alpha$, 且任意 $\theta_1\in C$, $\theta_2\notin C$, $\pi(\theta_1|\vec{x})\ge\pi(\theta_2|\vec{x})$, 则称 $C$ 为 $\theta$ 的可信水平为 $1-\alpha$ 的 Bayesian HPD 可信集. 作法: 在 $\theta-p$图中, 划一平行于 $\theta$ 轴的横线, 与 $p(\theta)$ 曲线的交点向下划垂直于 $\theta$ 轴的竖线, 获得可信集端点.

Bayesian HPD 可信区间使 Bayesian 可信区间最短.

若 Bayesian HPD 可信集不是可信区间, 则不用 Bayesian HPD 可信集, 而用 $\alpha/2$ 和 $1-\alpha/2$ 分位数获得等尾可信区间, 并检查 prior.

\section{Prior}

\subsection{共轭 Prior}

\subsection{无信息 Prior}

\subsubsection{均匀 Prior}

\subsubsection{Jeffreys Prior}

Fisher 信息矩阵 $\mathcal{I}_{ij}(\vec{\theta}):=\t{E}[\frac{\p\ln p(\vec{x}|\vec{\theta})}{\p \theta_i}\cdot\frac{\p\ln p(\vec{x}|\vec{\theta})}{\p \theta_j}]$.

Jeffreys prior $\pi_\t{J}(\vec{\theta}):\propto\det(\mathcal{I}_{ij}(\vec{\theta}))^{1/2}$, 

\subsection{有信息 Prior}

\section{Hierarchical Prior}

$\pi(\vec{\theta}|\vec{\lambda})$ 中未定且需定的 parameter 称为 hyperparameter. 为难定的 hyperparameter 给的 prior 称为 hyperprior. prior 和 hyperprior 组成 hierarchical prior.

\section{算法}

\subsection{MCMC}

用 Monte Carlo 法模拟根据 posterior 构造的 Markov chain 来估计 posterior 的方法.

构造一个 Markov chain $\vec{\theta}^{(1)},\dots,\vec{\theta}^{(n)},\dots$, 使得 $\lim_{n\to\infty}p(\vec{\theta}^{(n)})=p(\vec{\theta}|\vec{x})$, 用 Monte Carlo 法模拟 $\vec{\theta}^{(1)},\dots,\vec{\theta}^{(n)}$, 去除 $\vec{\theta}^{(1)},\dots,\vec{\theta}^{(m)}$, 用 $\vec{\theta}^{(m+1)},\dots,\vec{\theta}^{(n)}$ 估计 $p(\vec{\theta}|\vec{x})$.
