\chapter{干涉仪}

\section{类光测地线}

$h_{ab}=h\cos[\omega_\t{GW}(x^0-x^3)][(\d x^1)_a(\d x^1)_b-(\d x^2)_a(\d x^2)_b]$, 不妨设$x^3=0$. 设切矢$T^a=(\p/\p \beta)^a$, $T^\mu=\d x^\mu/\d \beta$,
\begin{equation}
    \frac{\d T^\mu}{\d \beta}+\Gamma^{\mu}_{\ph{\mu}\nu\sigma}T^\nu T^\sigma=0,
\end{equation}
\begin{equation}
    g_{\mu\nu}T^\mu T^\nu=0.
\end{equation}
\begin{equation}
    \Gamma^{\mu}_{\ph{\mu}\nu\sigma}=\frac{1}{2}\eta^{\mu\lambda}(\p_{\nu}h_{\sigma\lambda}+\p_{\sigma}h_{\nu\lambda}-\p_{\lambda}h_{\nu\sigma})
\end{equation}
\begin{equation}
    \Gamma^{0}_{\ph{0}\nu\sigma}=\frac{1}{2}\p_{0}h_{\nu\sigma}
\end{equation}
\begin{equation}
    \Gamma^{1}_{\ph{1}\nu\sigma}=\frac{1}{2}(\p_{\nu}h_{\sigma1}+\p_{\sigma}h_{\mu1})
\end{equation}
\begin{equation}
    \frac{\d T^0}{\d \beta}+\frac{1}{2}\p_0h_{11}T^1 T^1=0,
\end{equation}
\begin{equation}
    \frac{\d T^1}{\d \beta}+\p_0h_{11}T^0 T^1=0,
\end{equation}
\begin{equation}
    -T^0 T^0+(1+h_{11})T^1 T^1=0.
\end{equation}
猜测: $x^0\approx x^1\approx\beta$, $T^0\approx T^1\approx1$,
\begin{equation}
    \frac{\d T^0}{\d \beta}\approx-\frac{1}{2}\p_0h_{11},
\end{equation}
\begin{equation}
    \frac{\d T^1}{\d \beta}\approx-\p_0h_{11},
\end{equation}
$\p/\p x^0\approx\p/\p \beta$,
\begin{equation}
    T^0\approx1-\frac{1}{2}h_{11},
\end{equation}
\begin{equation}
    T^1\approx1-h_{11}.
\end{equation}
验证: $(\p/\p x^1)^a$是Killing的, 沿测地线$g_{ab}(\p/\p x^1)^a(\p/\p\beta)^b=g_{11}(\d x^1)_b(\p/\p\beta)^b=g_{11}(\d x^1/\d\beta)=g_{11}T^{1}=0$守恒. 于是
\begin{equation}
    T^a=(1-\frac{1}{2}h_{11})(\p/\p x^0)^a+(1-h_{11})(\p/\p x^1)^a
\end{equation}
\begin{equation}
    T_a=-(1-\frac{1}{2}h_{11})(\d x^0)_a+(\d x^1)_a,
\end{equation}
如此可过渡到 \ref{TT_explanation} 节.

$E^a=C^a\cos\theta$, $\nabla_a\theta=\p_a\theta=(\d\theta)_a=\frac{\p\theta}{\p x^0}(\d x^0)_a+\frac{\p\theta}{\p 1^0}(\d x^1)_a=\omega T_a$, $\theta=\int T_0\,\d x^0+\int T_1\,\d x^1+\theta_0$.

辩曰: 引力波至时, 臂长有伸缩, 然光波长无伸缩欤? 对曰: 光波长伸缩同! 以``坐标距离''考之, 臂长与光波长不变, 而波速变为$1\mp\frac{1}{2}h_{11}$倍, 往返时间变为$1\pm\frac{1}{2}h_{11}$倍, 知有图样. 以``物理距离''考之, 臂长与光波长变为$1\mp\frac{1}{2}h_{11}$倍, 波速不变, 而往返时间变为$1\pm\frac{1}{2}h_{11}$倍, 亦知有图样. \href{https://www.ligo.org/science/faq.php}{LSC FAQ} 亦云: ``引力波确实会拉伸和挤压臂中光的波长. 但是干涉图案并不是由于臂的长度和光的波长之间的差异而产生的. 相反, 它是由光波的``波峰和波谷''从一只臂到达的时间与光在另一只臂传播到达的时间不同引起的. 因此, 激光的作用与其说是尺子, 不如说是秒表. ''

\section{干涉图样(TT frame)}\label{TT_explanation}

\cite{Maggiore2014}. 设入射电场$\vec{E}_0e^{-i\omega_\t{L}t+i\vec{k}_\t{L}\cdot\vec{x}}$. 设splitter在$(0,0)$, reflector x在$(L_x,0)$, reflector y在$(0,L_y)$.

无GW时, 有$\vec{E}_\t{in}=\vec{E}_0e^{-i\omega_\t{L}t}$. $\vec{E}_\t{out}=\vec{E}_\t{form x}+\vec{E}_\t{form y}$, $t$时的$\vec{E}_\t{form x}$在$t-\frac{2L_x}{c}$时入splitter, $t$时的$\vec{E}_\t{form y}$在$t-\frac{2L_y}{c}$时入splitter, 考虑反相, $\vec{E}_\t{form x}=-\frac{1}{2}\vec{E}_0e^{-i\omega_\t{L}t+2ik_\t{L}L_x}$, $\vec{E}_\t{form y}=+\frac{1}{2}\vec{E}_0e^{-i\omega_\t{L}t+2ik_\t{L}L_y}$, $\vec{E}_\t{out}=\vec{E}_0\sin(\phi_0)e^{-i\omega_\t{L}(t-\frac{2L}{c})-i\frac{\pi}{2}}$, where $\phi_0=k_\t{L}(L_y-L_x)$ and $L=(L_x+L_y)/2$.

有GW时, 设GW只有$+$mode且方向为$z_+$, $h_+=h_0\cos[\omega_\t{gw}(t-z/c)]$, 
\begin{equation}
    \d s^2=-c^2\d t^2+(1 +h_+)\d x^2+ (1-h_+)\d y^2+\d z^2.
\end{equation}
$h_+(t):=h_+|_{z=0}$. 光$\d s^2=0$, 保留一阶项, x方向光轨迹
\begin{equation}
    \d x=\pm c\d t[1-\frac{1}{2}h_+(t)],
\end{equation}
y方向光轨迹
\begin{equation}
    \d y=\pm c\d t[1+\frac{1}{2}h_+(t)],
\end{equation}
$+$号是splitter到reflector, $-$号是reflector到splitter. 

设photon $t=t_0$到splitter, $t=t_1$到x reflector, $t=t_2$到splitter, 则
\begin{eqnarray}
    t_2-t_0=\frac{2L_x}{c}+\frac{1}{2}\int^{t_2}_{t_0}\d t'h_+(t')\\
    \approx\frac{2L_x}{c}+\frac{1}{2}\int^{t_0+\frac{2L_x}{c}}_{t_0}\d t'h_+(t')\\
    =\frac{2L_x}{c}+\frac{L_x}{c}h_+(t_0+\frac{L_x}{c})\sinc(\omega_\t{gw}\frac{L_x}{c}).
\end{eqnarray}
特殊情况: $\omega_\t{gw}\frac{L_x}{c}\ll1$, $t_2-t_0\approx\frac{2L_x}{c}+\frac{L_x}{c}h_+(t_1)$. $\omega_\t{gw}\frac{L_x}{c}\gg1$, $t_2-t_0\approx\frac{2L_x}{c}$.

y方向, x改成y, $+h_+$改成$-h_+$.

$\vec{E}_\t{in}=\vec{E}_0e^{-i\omega_\t{L}t}$, $t$时的$\vec{E}_\t{form x}$在$t-\frac{2L_x}{c}-\frac{L_x}{c}h_+(t-\frac{L_x}{c})\sinc(\omega_\t{gw}\frac{L_x}{c})$时入splitter, $t$时的$\vec{E}_\t{form y}$在$t-\frac{2L_y}{c}+\frac{L_y}{c}h_+(t-\frac{L_y}{c})\sinc(\omega_\t{gw}\frac{L_y}{c})$时入splitter, $\vec{E}_\t{form x}=-\frac{1}{2}\vec{E}_0e^{-i\omega_\t{L}(t-\frac{2L}{c})+i\phi_0+i\Delta\phi(t)}$, $\vec{E}_\t{form y}=+\frac{1}{2}\vec{E}_0e^{-i\omega_\t{L}(t-\frac{2L}{c})-i\phi_0-i\Delta\phi(t)}$, where $\phi_0=k_\t{L}(L_y-L_x)$, $\Delta\phi(t)=h_+(t-\frac{L}{c})k_\t{L}L\sinc(\omega_\t{gw}\frac{L}{c})$, and $L=(L_x+L_y)/2$. 特殊情况: $\omega_\t{gw}\frac{L_x}{c}\ll1$, $\Delta\phi(t)\approx h_+(t-\frac{L}{c})k_\t{L}L$. $\omega_\t{gw}\frac{L_x}{c}\gg1$, $\Delta\phi(t)\approx0$. Finally, $\vec{E}_\t{out}=\vec{E}_0\sin[\phi_0+\Delta\phi(t)]e^{-i\omega_\t{L}(t-\frac{2L}{c})-i\frac{\pi}{2}}$.

辩曰: 反射镜运动, 有Doppler效应, 何以应对? 对曰: 需有long wave approximation $\omega_\t{gw}\frac{L}{c}\ll1$, 此情形下单纯传播引入相移$\Delta\phi_\t{p}\propto h$, Doppler效应引入相移$\Delta\phi_\t{d}\propto \frac{\frac{\d h}{\d t}L}{c}\sim\frac{\omega_\t{gw}hL}{c}\ll\Delta\phi_\t{p}$可忽略. 详细论证: reflector接收, 相移$2\pi[(1\pm h)L/\lambda_\t{L}']$, splitter再接收, 相移$2\pi[(1\pm h)L/\lambda_\t{L}'']$, 其中$\lambda_\t{L}''/\lambda_\t{L}'=\lambda_\t{L}'/\lambda_\t{L}=1\pm(\d h/\d t)L/c$, $(1/\lambda_\t{L}'')/(1/\lambda_\t{L}')=(1/\lambda_\t{L}')/(1/\lambda_\t{L})=1\mp(\d h/\d t)L/c$, 总相移$2\pi[2(1\pm h)(1\mp(\d h/\d t)L/c)L/\lambda_\t{L}]\approx2\pi[2(1\pm h\mp(\d h/\d t)L/c)L/\lambda_\t{L}]$. 若$\omega_\t{gw}L/c\ll1$ ($v/c\ll h$), 则$h\gg(\d h/\d t)L/c\approx h(\omega_\t{gw}L/c)$, 总相移$2\pi[2(1\pm h)L/\lambda_\t{L}]$. 注: $(\d h/\d t)|_{t_2}-(\d h/\d t)|_{t_1}\approx(\d^2 h/\d t^2)(L/c)\approx(\d h/\d t)(\omega_\t{gw}L/c)\ll(\d h/\d t)$.
