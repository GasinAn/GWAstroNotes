\chapter{干涉仪}

\cite{Maggiore2014}. 设入射电场$\vec{E}_\text{in}=\vec{E}_0e^{-i\omega_\text{L}t+i\vec{k}_\text{L}\cdot\vec{x}}$. 设splitter在$\vec{x}=0$处, 则$\vec{E}_\text{in}=\vec{E}_0e^{-i\omega_\text{L}t}$. $\vec{E}_\text{out}=\vec{E}_\text{form x}+\vec{E}_\text{form y}$, $t$时的$\vec{E}_\text{form x}$在$t-\frac{2L_x}{c}$时入splitter, $t$时的$\vec{E}_\text{form y}$在$t-\frac{2L_y}{c}$时入splitter, 考虑反相, $\vec{E}_\text{form x}=-\frac{1}{2}\vec{E}_0e^{-i\omega_\text{L}t+2ik_\text{L}L_x}$, $\vec{E}_\text{form y}=+\frac{1}{2}\vec{E}_0e^{-i\omega_\text{L}t+2ik_\text{L}L_y}$, $\vec{E}_\text{out}=\vec{E}_0\sin(\phi_0)e^{-i\omega_\text{L}(t-\frac{2L}{c})-i\frac{\pi}{2}}$, where $\phi_0=k_\text{L}(L_y-L_x)$ and $L=(L_x+L_y)/2$.

\section{简单简单解释}

reflector接收, 相移$2\pi[(1\pm h)L/\lambda_\text{L}']$, splitter再接收, 相移$2\pi[(1\pm h)L/\lambda_\text{L}'']$, 其中$\lambda_\text{L}''/\lambda_\text{L}'=\lambda_\text{L}'/\lambda_\text{L}=1\pm(\d h/\d t)L/c$, $(1/\lambda_\text{L}'')/(1/\lambda_\text{L}')=(1/\lambda_\text{L}')/(1/\lambda_\text{L})=1\mp(\d h/\d t)L/c$, 总相移$2\pi[2(1\pm h)(1\mp(\d h/\d t)L/c)L/\lambda_\text{L}]\approx2\pi[2(1\pm h\mp(\d h/\d t)L/c)L/\lambda_\text{L}]$. 若$\omega_\text{gw}L/c\ll1$ ($v/c\ll h$), 则$h\gg(\d h/\d t)L/c\approx h(\omega_\text{gw}L/c)$. 注: $(\d h/\d t)|_{t_2}-(\d h/\d t)|_{t_1}\approx(\d^2 h/\d t^2)(L/c)\approx(\d h/\d t)(\omega_\text{gw}L/c)\ll(\d h/\d t)$.

\section{简单解释}

$h:=h_0\sin(\omega_\text{gw} t)$. splitter在$(0,0)$, reflector x在$(L(1+h),0)$, reflector y在$(0,L(1-h))$. 设photon $t=t_0$到splitter, $t=t_1$到x reflector, $t=t_2$到splitter, $c(t_1-t_0)=L(1 \pm h(t_1))$, $c(t_2-t_1)=L(1 \pm h(t_1))$.

解: $t_0=t_1-(L/c)(1 \pm h(t_1))$, $t_2=t_1+(L/c)(1 \pm h(t_1))$, $\omega_\text{gw} (t_2-(L/c))=(\omega_\text{gw} t_1) \pm \omega_\text{gw} (L/c) h_0 \sin(\omega_\text{gw} t_1)$, $\omega_\text{gw} t_1 \approx \omega_\text{gw} (t_2-(L/c)) \mp \omega_\text{gw} (L/c) h_0 \sin(\omega_\text{gw} (t_2-(L/c)))$, $t_1=t_2-(L/c)(1 \pm h(t_2-(L/c)))$, $h(t_1) \approx h(t_2-(L/c))$, $t_0=t_2-2(L/c)(1 \pm h(t_2-(L/c)))$.

$E_\text{in}(t):=e^{-i\omega_\text{L} t}$, $E_\text{out}(t_2)=E_\text{in}(t_0)$, $E_\text{out}(t)=e^{-i\omega_\text{L} (t-2(L/c)(1 \pm h(t-(L/c))))}$.

\section{TT frame解释}

设splitter在$(0,0)$, reflector x在$(L_x,0)$, reflector y在$(0,L_y)$, 显然无GW时如上.

设GW只有$+$mode且方向为$z_+$, $h_+=h_0\cos[\omega_\text{gw}(t-z/c)]$, 
\begin{equation}
    \d s^2=-c^2\d t^2+(1 +h_+)\d x^2+ (1-h_+)\d y^2+\d z^2.
\end{equation}
$h_+(t):=h_+|_{z=0}$. 光$\d s^2=0$, 保留一阶项, x方向光轨迹
\begin{equation}
    \d x=\pm c\d t[1-\frac{1}{2}h_+(t)],
\end{equation}
y方向光轨迹
\begin{equation}
    \d y=\pm c\d t[1+\frac{1}{2}h_+(t)],
\end{equation}
$+$号是splitter到reflector, $-$号是reflector到splitter.

设photon $t=t_0$到splitter, $t=t_1$到x reflector, $t=t_2$到splitter, 则
\begin{eqnarray}
    t_2-t_0=\frac{2L_x}{c}+\frac{1}{2}\int^{t_2}_{t_0}\d t'h_+(t')\\
    \approx\frac{2L_x}{c}+\frac{1}{2}\int^{t_0+\frac{2L_x}{c}}_{t_0}\d t'h_+(t')\\
    =\frac{2L_x}{c}+\frac{L_x}{c}h_+(t_0+\frac{L_x}{c})\sinc(\omega_\text{gw}\frac{L_x}{c}).
\end{eqnarray}
$\omega_\text{gw}\frac{L_x}{c}\ll1$, $t_2-t_0\approx\frac{2L_x}{c}+\frac{L_x}{c}h_+(t_1)$. $\omega_\text{gw}\frac{L_x}{c}\gg1$, $t_2-t_0\approx\frac{2L_x}{c}$.

y方向, x改成y, $+h_+$改成$-h_+$.

$\vec{E}_\text{in}=\vec{E}_0e^{-i\omega_\text{L}t}$, $t$时的$\vec{E}_\text{form x}$在$t-\frac{2L_x}{c}-\frac{L_x}{c}h_+(t-\frac{L_x}{c})\sinc(\omega_\text{gw}\frac{L_x}{c})$时入splitter, $t$时的$\vec{E}_\text{form y}$在$t-\frac{2L_y}{c}+\frac{L_y}{c}h_+(t-\frac{L_y}{c})\sinc(\omega_\text{gw}\frac{L_y}{c})$时入splitter, $\vec{E}_\text{form x}=-\frac{1}{2}\vec{E}_0e^{-i\omega_\text{L}(t-\frac{2L}{c})+i\phi_0+i\Delta\phi(t)}$, $\vec{E}_\text{form y}=+\frac{1}{2}\vec{E}_0e^{-i\omega_\text{L}(t-\frac{2L}{c})-i\phi_0-i\Delta\phi(t)}$, where $\phi_0=k_\text{L}(L_y-L_x)$, $\Delta\phi(t)=h_+(t-\frac{L}{c})k_\text{L}L\sinc(\omega_\text{gw}\frac{L}{c})$, and $L=(L_x+L_y)/2$. Finally, $\vec{E}_\text{out}=\vec{E}_0\sin[\phi_0+\Delta\phi(t)]e^{-i\omega_\text{L}(t-\frac{2L}{c})-i\frac{\pi}{2}}$.
