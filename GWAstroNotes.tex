% 编译方式: latexmk -xelatex 或 xelatex -> bibtex -> xelatex*2
\documentclass{ctexbook}
\usepackage{amsfonts}
\usepackage{amsmath}
\usepackage{amssymb}
\usepackage{hyperref}
\usepackage{syntonly}
\usepackage{verbatim}
%\syntaxonly
\pagestyle{plain}
\makeatletter
\newcommand{\starttoc}{
    \chapter*{\contentsname}
    \@starttoc{toc}
}
\makeatother
%\renewcommand{\tableofcontents}{\twocolumn\starttoc\onecolumn}
\hypersetup{
    colorlinks,
    linkcolor=blue,
    citecolor=red,
    filecolor=cyan,
    urlcolor=magenta,
}
\allowdisplaybreaks
\def\b{\boldsymbol}
\def\d{\mathrm{d}}
\def\p{\partial}
\def\ph{\phantom}
\def\t{\text}
\def\ti{\tilde}
\def\v{\vec}
\def\La{\Leftarrow}
\def\Ra{\Rightarrow}
\DeclareMathOperator{\atanxy}{atan2}
\DeclareMathOperator{\sinc}{sinc}
\DeclareMathOperator{\sgn}{sgn}
\DeclareMathOperator{\Arg}{Arg}
\title{引力波天文学笔记}
\author{GasinAn}
\begin{document}
    \maketitle
    \thispagestyle{empty}

\begin{flushleft}
    Copyright \textcopyright{} 2024 by GasinAn

    \ 

    All rights reserved. No part of this book may be reproduced, in any form or by any means, without permission in writing from the publisher, except by a \LaTeX er.

    \ 

    The author and publisher of this book have used their best efforts in preparing this book. These efforts include the development, research, and testing of the theories, technologies and programs to determine their effectiveness. The author and publisher make no warranty of any kind, express or implied, with regard to these techniques or programs contained in this book. The author and publisher shall not be liable in any event of incidental or consequential damages in connection with, or arising out of, the furnishing, performance, or use of these techniques or programs.

    \ 

    Printed in China
\end{flushleft}

    \tableofcontents
    \chapter{引力波}

\section{Linearized Gravity}

\cite{Wald1984}. 流形$\mathbb{R}^{4}$. 任意坐标系$\{x^{\mu}\}$, $g_{\mu\nu}=\eta_{\mu\nu}+h_{\mu\nu}=\eta_{\mu\nu}+\gamma_{\mu\nu}s+\text{O}(s^2)$, 得
\begin{equation}
    R_{\mu\nu\lambda\sigma}=\p_\sigma\p_{[\mu}h_{\lambda]\nu}-\p_\nu\p_{[\mu}h_{\lambda]\sigma}+\text{O}(s^2).
\end{equation}
$\bar{h}_{\mu\nu}:=h_{\mu\nu}-\frac{1}{2}\eta_{\mu\nu}\eta^{\lambda\sigma}h_{\lambda\sigma}=h_{\mu\nu}-\frac{1}{2}\eta_{\mu\nu}h$.
\begin{equation}
    -\frac{1}{2} \partial^{\lambda} \partial_{\lambda} \bar{h}_{\mu \nu}+\partial^{\lambda} \partial_{(\mu} \bar{h}_{\nu) \lambda}-\frac{1}{2} \eta_{\mu \nu} \partial^{\lambda} \partial^{\sigma} \bar{h}_{\lambda \sigma}+\text{O}(s^2)=8 \pi T_{\mu \nu}.
\end{equation}
存在$\{x^{\mu}\}$, 使得$\p^{\nu}\bar{h}_{\mu\nu}+\text{O}(s^2)=0$ (Lorentz gauge). 
令$\{x^{\mu}\}$满足$\p^{\nu}\bar{h}_{\mu\nu}+\text{O}(s^2)=0$, 则
\begin{equation}
    \p^{\lambda}\p_{\lambda}\bar{h}_{\mu\nu}+\text{O}(s^2)=-16\pi T_{\mu \nu}.\label{lin_gravity}
\end{equation}
略去$\text{O}(s^2)$条件: $h_{\mu\nu}$, $\p_\lambda h_{\mu\nu}$\dots{}小.

\section{Radiation Gauge}

\cite{Wald1984}. 存在$\{x^{\mu}\}$, 使得``无源处'' $h+\text{O}(s^2)=0$ (TT gauge \cite{Wang2020})且$h_{0\mu}+\text{O}(s^2)=0$. 

\section{Quadrupole Approximation}

\cite{Wald1984}. 下略$\text{O}(s^2)$. 由\eqref{lin_gravity}得
\begin{equation}
    \bar{h}_{\mu\nu}(t,\vec{r}) = 4\int 
    \frac{T_{\mu\nu}(t-\left\lvert\vec{r}-\vec{r}'\right\rvert,\vec{r}')}{\left\lvert\vec{r}-\vec{r}'\right\rvert}\,\d V'.
\end{equation}
\begin{align}
    \hat{\bar{h}}_{\mu\nu}(\omega,\vec{r})&:=\frac{1}{\sqrt{2\pi}}
    \int\bar{h}_{\mu\nu}(t,\vec{r})e^{i\omega t}\d t\\
    &=4\int 
    \frac{\hat{T}_{\mu\nu}(\omega,\vec{r}')}{\left\lvert\vec{r}-\vec{r}'\right\rvert}e^{i\omega\left\lvert\vec{r}-\vec{r}'\right\rvert}\,\d V'.
\end{align}
由$\p^{\nu}\bar{h}_{\mu\nu}=0$,
\begin{equation}
    -i\omega\hat{\bar{h}}_{0\mu}=\sum_{i}\frac{\p\hat{\bar{h}}_{i\mu}}{\p x^{i}}.
\end{equation}
$\left\lvert \vec{r}\right\rvert \gg \left\lvert \vec{r}'\right\rvert$且$\omega\ll1/\left\lvert \vec{r}'\right\rvert$,
\begin{equation}
    \hat{\bar{h}}_{ij}(\omega,\vec{r}) = 4
    \frac{e^{i\omega\left\lvert\vec{r}\right\rvert}}{\left\lvert\vec{r}\right\rvert}
    \int \hat{T}_{ij}(\omega,\vec{r}')\,\d V'.
\end{equation}
\begin{align}
    \int \hat{T}_{ij}\,\d V'
    &=\int \sum_{k}(\hat{T}_{kj}\frac{\p x'^i}{\p x'^k})\,\d V'\\
    &= \sum_{k}\left[\int \frac{\p }{\p x'^k}(\hat{T}_{kj}x'^i)\,\d V'-\int \frac{\p \hat{T}_{kj}}{\p x'^k}x'^i\,\d V'\right]\\
    &= \sum_{k}\int \p'_k\,(\hat{T}_{kj}x'^i)\,\d V' - \sum_{k}\int \frac{\p \hat{T}_{kj}}{\p x'^k}x'^i\,\d V'\\
    &= \int \hat{T}_{kj}x'^i\,\d S' - \sum_{k}\int \frac{\p \hat{T}_{kj}}{\p x'^k}x'^i\,\d V'\\
    &= -\sum_{k}\int \frac{\p \hat{T}_{kj}}{\p x'^k}x'^i\,\d V'\\
    &= -\int (\sum_{k}\p'_k\hat{T}_{kj})x'^i\,\d V'\\
    &= -\int (\p_0\hat{T}_{0j})x'^i\,\d V'\\
    &= -i\omega\int \hat{T}_{0j}x'^i\,\d V'\\
    &= \int \hat{T}_{(ij)}\,\d V'\\
    &= -i\omega\int \hat{T}_{0(j}x'^{i)}\,\d V'\\
    &= -\frac{i\omega}{2}\int (\hat{T}_{0j}x'^{i}+\hat{T}_{0i}x'^{j})\,\d V',\\
\end{align}
\begin{align}
    -\frac{i\omega}{2}\int (\hat{T}_{0j}x'^{i}+\hat{T}_{0i}x'^{j})\,\d V'
    &= -\frac{i\omega}{2}\int \sum_{k}(\hat{T}_{0k}x'^{i}\frac{\p x'^{j}}{\p x'^{k}}+\hat{T}_{0k}\frac{\p x'^{i}}{\p x'^{k}}x'^{j})\,\d V'\\
    &= -\frac{i\omega}{2}\sum_{k}\left[\int \frac{\p }{\p x'^k}(\hat{T}_{0k}x'^ix'^j)\,\d V'-\int \frac{\p \hat{T}_{0k}}{\p x'^k}x'^ix'^j\,\d V'\right]\\
    &= -\frac{i\omega}{2}\sum_{k}\int \p'_k\,(\hat{T}_{0k}x'^ix'^j)\,\d V' +\frac{i\omega}{2}\sum_{k}\int \frac{\p \hat{T}_{0k}}{\p x'^k}x'^ix'^j\,\d V'\\
    &= -\frac{i\omega}{2}\sum_{k}\int \hat{T}_{0k}x'^ix'^j\,\d S' +\frac{i\omega}{2}\sum_{k}\int \frac{\p \hat{T}_{0k}}{\p x'^k}x'^ix'^j\,\d V'\\
    &= \frac{i\omega}{2}\sum_{k}\int \frac{\p \hat{T}_{0k}}{\p x'^k}x'^ix'^j\,\d V'\\
    &= \frac{i\omega}{2}\int (\sum_{k}\p'_k\hat{T}_{0k})x'^ix'^j\,\d V'\\
    &= \frac{i\omega}{2}\int (\p_0\hat{T}_{00})x'^ix'^j\,\d V'\\
    &= -\frac{\omega^2}{2}\int \hat{T}_{00}\,x'^ix'^j\,\d V'.
\end{align}
\begin{equation}
    q_{ij}(t):=\int{T}_{00}\,x'^ix'^j\,\d V',
\end{equation}
\begin{equation}
    \hat{\bar{h}}_{ij}(\omega,\vec{r}) = -2\omega^2\frac{e^{i\omega\left\lvert\vec{r}\right\rvert}}{\left\lvert\vec{r}\right\rvert}\hat{q}_{ij}(\omega),
\end{equation}
\begin{equation}
    {\bar{h}}_{ij}(t,\vec{r}) = \frac{2}{\left\lvert\vec{r}\right\rvert}
    \frac{\d^2}{\d t^2}{q}_{ij}(t-\left\lvert\vec{r}\right\rvert).\label{qf}
\end{equation}

\section{$+$ Mode and $\times$ Mode}

寻新标架$(e'^1)_a=(e^+)_a$, $(e'^2)_a=(e^\times)_a$, $(e'^3)_a=(e^r)_a$, ${\bar{h}}_{ij}(e^i)_a(e^j)_b={\bar{h}}'_{ij}(e'^i)_a(e'^j)_b$, 取$x$, $y$分量后去迹, $h_+=\frac{1}{2}({\bar{h}}'_{11}-{\bar{h}}'_{22})$, $h_\times={\bar{h}}'_{12}={\bar{h}}'_{21}$? \cite{Sathyaprakash2009}

\cite{Blanchet1997}, $\vec{n}:=\frac{\vec{r}}{\left\lvert\vec{r}\right\rvert}$,
\begin{equation}
    h_{ij}^{\text{TT}}=\frac{2}{\left\lvert\vec{r}\right\rvert}\mathcal{P}_{ijkm}\frac{\d^2}{\d t^2}{Q}^{km}(t-\left\lvert\vec{r}\right\rvert), \label{TT}
\end{equation}
\begin{equation}
    \mathcal{P}_{ijkm}:=
    \left(\delta_{ik} -\vec{n}_i\vec{n}_k\right)
    \left(\delta_{jm} -\vec{n}_j\vec{n}_m\right)
    -\frac{1}{2}
    \left(\delta_{ij} -\vec{n}_i\vec{n}_j\right)
    \left(\delta_{km} -\vec{n}_k\vec{n}_m\right),
\end{equation}
\begin{equation}
    Q^{km}(t):=\int{T}_{00}\,\left(x'^kx'^m-\frac{1}{3}\delta^{km}\sum_nx'^nx'^n\right)\,\d V'
\end{equation}

\section{电磁---引力对比}

\begin{equation}
    A_\mu(t,\vec{r})=\frac{\mu_0}{4\pi}\int\frac{J_\mu(t-\left\lvert \vec{r}-\vec{r}'\right\rvert ,\vec{r}')}{\left\lvert \vec{r}-\vec{r}'\right\rvert}\d V'
\end{equation}
\begin{equation}
    \bar{h}_{\mu\nu}(t,\vec{r})=4G\int\frac{T_{\mu\nu}(t-\left\lvert\vec{r}-\vec{r}'\right\rvert,\vec{r}')}{\left\lvert\vec{r}-\vec{r}'\right\rvert}\,\d V'
\end{equation}

\begin{equation}
    A_\mu(t,\vec{r})=\frac{1}{\sqrt{2\pi}}
    \int\hat{A}_{\mu}(\omega,\vec{r})e^{-i\omega t}\d t
\end{equation}
\begin{equation}
    \bar{h}_{\mu\nu}(t,\vec{r})=\frac{1}{\sqrt{2\pi}}
    \int\hat{\bar{h}}_{\mu\nu}(\omega,\vec{r})e^{-i\omega t}\d t
\end{equation}

\begin{equation}
    \hat{A}_{\mu}(\omega,\vec{r})=\frac{\mu_0}{4\pi}\int\frac{\hat{J}_{\mu}(\omega,\vec{r}')}{\left\lvert\vec{r}-\vec{r}'\right\rvert}e^{i\omega\left\lvert\vec{r}-\vec{r}'\right\rvert}\,\d V'
\end{equation}
\begin{equation}
    \hat{\bar{h}}_{\mu\nu}(\omega,\vec{r})=4G\int\frac{\hat{T}_{\mu\nu}(\omega,\vec{r}')}{\left\lvert\vec{r}-\vec{r}'\right\rvert}e^{i\omega\left\lvert\vec{r}-\vec{r}'\right\rvert}\,\d V'
\end{equation}

\begin{equation}
    \hat{A}_{\mu}(\omega,\vec{r})=\frac{\mu_0}{4\pi}\frac{e^{i\omega\left\lvert\vec{r}\right\rvert}}{\left\lvert\vec{r}\right\rvert}\int\hat{J}_{\mu}(\omega,\vec{r}')e^{-i\omega(\frac{\vec{r}}{\left\lvert\vec{r}\right\rvert}\cdot\vec{r}')}\,\d V'
\end{equation}
\begin{equation}
    \hat{\bar{h}}_{\mu\nu}(\omega,\vec{r})=4G\frac{e^{i\omega\left\lvert\vec{r}\right\rvert}}{\left\lvert\vec{r}\right\rvert}\int\hat{T}_{\mu\nu}(\omega,\vec{r}')e^{-i\omega(\frac{\vec{r}}{\left\lvert\vec{r}\right\rvert}\cdot\vec{r}')}\,\d V'
\end{equation}

\begin{equation}
    \hat{A}_{\mu}(\omega,\vec{r})=\frac{\mu_0}{4\pi}\frac{e^{i\omega\left\lvert\vec{r}\right\rvert}}{\left\lvert\vec{r}\right\rvert}\int\hat{J}_{\mu}(\omega,\vec{r}')\left[1-i\omega(\frac{\vec{r}}{\left\lvert\vec{r}\right\rvert}\cdot\vec{r}')-\dots\right]\,\d V'
\end{equation}
\begin{equation}
    \hat{\bar{h}}_{\mu\nu}(\omega,\vec{r})=4G\frac{e^{i\omega\left\lvert\vec{r}\right\rvert}}{\left\lvert\vec{r}\right\rvert}\int\hat{T}_{\mu\nu}(\omega,\vec{r}')\left[1-i\omega(\frac{\vec{r}}{\left\lvert\vec{r}\right\rvert}\cdot\vec{r}')-\dots\right]\,\d V'
\end{equation}

\subsection{电偶极---引力对比}

\begin{equation}
    \hat{A}_i=\frac{\mu_0}{4\pi}\frac{e^{i\omega\left\lvert\vec{r}\right\rvert}}{\left\lvert\vec{r}\right\rvert}\int\hat{J}_i\,\d V'
\end{equation}
\begin{equation}
    \hat{\bar{h}}_{ij}=4G\frac{e^{i\omega\left\lvert\vec{r}\right\rvert}}{\left\lvert\vec{r}\right\rvert}\int\hat{T}_{ij}\,\d V'
\end{equation}

\begin{equation}
    \int \hat{J}_{i}\,\d V'=-i\omega\int\hat{J}_{0}x'^i\,\d V'
\end{equation}
\begin{equation}
    \int\hat{T}_{ij}\,\d V'=-\frac{\omega^2}{2}\int \hat{T}_{00}\,x'^ix'^j\,\d V'
\end{equation}

\begin{equation}
    \hat{p}_i=\int\hat{J}_{0}x'^i\,\d V'
\end{equation}
\begin{equation}
    \hat{q}_{ij}=\int \hat{T}_{00}\,x'^ix'^j\,\d V'
\end{equation}

\begin{equation}
    \hat{A}_i=\frac{\mu_0}{4\pi}\frac{e^{i\omega\left\lvert\vec{r}\right\rvert}}{\left\lvert\vec{r}\right\rvert}(-i\omega \hat{p}_i)
\end{equation}
\begin{equation}
    \hat{\bar{h}}_{ij}=4G\frac{e^{i\omega\left\lvert\vec{r}\right\rvert}}{\left\lvert\vec{r}\right\rvert}(-\frac{\omega^2}{2}\hat{q}_{ij})
\end{equation}

\begin{equation}
    {A}_i=\frac{\mu_0}{4\pi}\frac{1}{\left\lvert\vec{r}\right\rvert}\frac{\d}{\d t}p_i(t-\left\lvert\vec{r}\right\rvert)
\end{equation}
\begin{equation}
    {\bar{h}}_{ij}=4G\frac{1}{\left\lvert\vec{r}\right\rvert}\frac{1}{2}\frac{\d^2}{\d t^2}q_{ij}(t-\left\lvert\vec{r}\right\rvert)
\end{equation}

\subsection{电四极---引力对比}

\begin{equation}
    \hat{A}_{i}(\omega,\vec{r})=\frac{\mu_0}{4\pi}\frac{e^{i\omega\left\lvert\vec{r}\right\rvert}}{\left\lvert\vec{r}\right\rvert}()(-i\omega)\int\hat{J}_{i}(\omega,\vec{r}')(\frac{\vec{r}}{\left\lvert\vec{r}\right\rvert}\cdot\vec{r}')\,\d V'
\end{equation}

\begin{equation}
    \hat{A}_i=\frac{\mu_0}{4\pi}\frac{e^{i\omega\left\lvert\vec{r}\right\rvert}}{\left\lvert\vec{r}\right\rvert}(-i\omega)\int\hat{J}_i'n^j{x}_j'\,\d V'
\end{equation}

\begin{equation}
    \hat{A}_i=\frac{\mu_0}{4\pi}\frac{e^{i\omega\left\lvert\vec{r}\right\rvert}}{\left\lvert\vec{r}\right\rvert}(-i\omega)\int n^j{x}_j'\hat{J}_i'\,\d V'
\end{equation}

\begin{equation}
    \hat{A}_i=\frac{\mu_0}{4\pi}\frac{e^{i\omega\left\lvert\vec{r}\right\rvert}}{\left\lvert\vec{r}\right\rvert}(-i\omega)n^j\!\left[\int {x}_{(j}'\hat{J}_{i)}'\,\d V'\right]
\end{equation}

\begin{align}
    \int {x}_{(j}'\hat{J}_{i)}'\,\d V'
    &= \frac{1}{2}\int (\hat{J}_{j}'{x}_{i}'+\hat{J}_{i}'{x}_{j}')\,\d V'\\
    &= \frac{1}{2}\int \sum_{k}(\hat{J}_{k}'x'^{i}\frac{\p x'^{j}}{\p x'^{k}}+\hat{J}_{k}'\frac{\p x'^{i}}{\p x'^{k}}x'^{j})\,\d V'\\
    &= \frac{1}{2}\sum_{k}\left[\int \frac{\p }{\p x'^k}(\hat{J}_{k}'x'^ix'^j)\,\d V'-\int \frac{\p \hat{J}_{k}'}{\p x'^k}x'^ix'^j\,\d V'\right]\\
    &= \frac{1}{2}\sum_{k}\int \p'_k\,(\hat{J}_{k}'x'^ix'^j)\,\d V' -\frac{1}{2}\sum_{k}\int \frac{\p \hat{J}_{k}'}{\p x'^k}x'^ix'^j\,\d V'\\
    &= \frac{1}{2}\sum_{k}\int \hat{J}_{k}'x'^ix'^j\,\d S' -\frac{1}{2}\sum_{k}\int \frac{\p \hat{J}_{k}'}{\p x'^k}x'^ix'^j\,\d V'\\
    &= -\frac{1}{2}\sum_{k}\int \frac{\p \hat{J}_{k}'}{\p x'^k}x'^ix'^j\,\d V'\\
    &= -\frac{1}{2}\int (\sum_{k}\p'_k\hat{J}_{k}')x'^ix'^j\,\d V'\\
    &= -\frac{1}{2}\int (\p_0\hat{J}_{0}')x'^ix'^j\,\d V'\\
    &= -\frac{i\omega}{2}\int \hat{J}_{0}'x'^ix'^j\,\d V'
\end{align}

\begin{equation}
    \hat{D}_{ij}=\int \hat{J}_{0}'\,x'^ix'^j\,\d V'
\end{equation}

\begin{equation}
    \hat{A}_i=\frac{\mu_0}{4\pi}\frac{e^{i\omega\left\lvert\vec{r}\right\rvert}}{\left\lvert\vec{r}\right\rvert}(-\frac{\omega^2}{2}n^j\hat{D}_{ij}
    )
\end{equation}

\begin{equation}
    {A}_i=\frac{\mu_0}{4\pi}\frac{1}{\left\lvert\vec{r}\right\rvert}n^j\frac{1}{2}\frac{\d^2}{\d t^2}D_{ij}(t-\left\lvert\vec{r}\right\rvert)
\end{equation}

\section{Varying $G$}

\begin{equation}
    T_{ab}=2U_{(a}J_{b)}+U^cJ_cU_aU_b
\end{equation}
\begin{equation}
    J_b=-U^aT_{ab}
\end{equation}
\begin{equation}
    \p^a\bar{h}_{ab}=0
\end{equation}
\begin{equation}
    \p^c\p_c\bar{h}_{ab}=-16\pi\frac{G_0}{c_0^3}T_{ab}
\end{equation}
\begin{equation}
    \Gamma^c_{\ph{c}ab}=\frac{1}{2}\eta^{cd}(2\p_{(a}{h}_{b)d}-\p_d{h}_{ab})
\end{equation}
\begin{equation}
    U^a\p_aU^c+\Gamma^c_{\ph{c}ab}U^aU^b=0
\end{equation}
\begin{equation}
    U^a\p_aU^c=-\frac{1}{2}\eta^{cd}(2\p_{(a}{h}_{b)d}-\p_d{h}_{ab})U^aU^b
\end{equation}

\begin{equation}
    A_b
    =-\frac{1}{4}U^a\bar{h}_{ab}
\end{equation}
\begin{equation}
    A_0
    =-\frac{1}{4}c_0\bar{h}_{00}
    =-\frac{1}{2}c_0(\bar{h}_{00}-\frac{1}{2}\eta_{00}\eta^{00}\bar{h}_{00})=-\frac{1}{2}c_0h_{00}
\end{equation}
\begin{equation}
    A_i=-\frac{1}{4}c_0\bar{h}_{0i}=-\frac{1}{4}c_0{h}_{0i}
\end{equation}

\begin{equation}
    U^\mu\p_\mu U^i=-\frac{1}{2}\eta^{i\sigma}(\p_\mu{h}_{\nu\sigma}+\p_\nu{h}_{\mu\sigma}-\p_\sigma{h}_{\mu\nu})U^\mu U^\nu
\end{equation}
\begin{align}
    -\frac{1}{2}\eta^{i\sigma}(\p_0{h}_{0\sigma}+\p_0{h}_{0\sigma}-\p_\sigma{h}_{00})U^0 U^0
    &=\frac{1}{2}c_0^2\eta^{i\sigma}\p_\sigma{h}_{00}\\
    &=\frac{1}{2}c_0^2\p^i{h}_{00}\\
    &=c_0\p^iA_0\\
    &=-E^i
\end{align}
\begin{align}
    -\frac{1}{2}\eta^{i\sigma}(\p_0{h}_{j\sigma}+\p_j{h}_{0\sigma}-\p_\sigma{h}_{0j})U^0 U^j
    &=-\frac{1}{2}c_0\eta^{i\sigma}(\p_j{h}_{0\sigma}-\p_\sigma{h}_{0j})v^j\\
    &=-\frac{1}{2}c_0\eta^{ik}(\p_j{h}_{0k}-\p_k{h}_{0j})v^j\\
    &=2\eta^{ik}(\p_jA_k-\p_kA_j)v^j\\
    &=2(\p_jA^i-\p^iA_j)v^j\\
    &=-2\varepsilon^{i}_{\ph{i}jk}v^jB^k
\end{align}
\begin{align}
    -\frac{1}{2}\eta^{i\sigma}(\p_j{h}_{k\sigma}+\p_k{h}_{j\sigma}-\p_\sigma{h}_{jk})U^j U^k=0
\end{align}
\begin{equation}
    a^i=-E^i-4\varepsilon^{i}_{\ph{i}jk}v^jB^k
\end{equation}

\begin{equation}
    \p^i(\frac{c_0}{16\pi G_0}E_i)=\rho
\end{equation}
\begin{equation}
    \p^iB_i=0
\end{equation}
\begin{equation}
    \varepsilon^{i}_{\ph{i}jk}\p^jE^k=-\p_tB^i
\end{equation}
\begin{equation}
    \varepsilon^{i}_{\ph{i}jk}\p^j(\frac{c_0^3}{16\pi G_0}B^k)=j^i+\p_t(\frac{c_0}{16\pi G_0}E^i)
\end{equation}

\begin{equation}
    \varepsilon_{\text{G}0}:=\frac{c_0}{16\pi G_0},\quad
    \mu_{\text{G}0}:=\frac{16\pi G_0}{c_0^3}
\end{equation}
\begin{equation}
    \begin{cases}
        \vec{\nabla}\cdot(\varepsilon_{\text{G}0}\vec{E})=\rho\\
        \vec{\nabla}\cdot\vec{B}=0\\
        \vec{\nabla}\times\vec{E}=-\frac{\p}{\p t}\vec{B}\\
        \vec{\nabla}\times(\mu_{\text{G}0}^{-1}\vec{B})=\vec{j}+\frac{\p}{\p t}(\varepsilon_{\text{G}0}\vec{E})
    \end{cases}
\end{equation}
\begin{equation}
    \vec{a}=-\vec{E}-4\vec{v}\times\vec{B}
\end{equation}

\begin{equation}
    \vec{a}=-\vec{E}-4\vec{v}\times\vec{B}
\end{equation}
\begin{equation}
    \begin{cases}
        \vec{\nabla}\cdot(\varepsilon_{\text{G}}\vec{E})=\rho\\
        \vec{\nabla}\cdot\vec{B}=0\\
        \vec{\nabla}\times\vec{E}=-\frac{\p}{\p t}\vec{B}\\
        \vec{\nabla}\times(\mu_{\text{G}}^{-1}\vec{B})=\vec{j}+\frac{\p}{\p t}(\varepsilon_{\text{G}}\vec{E})
    \end{cases}
\end{equation}

\begin{equation}
    {E}_r=0,\quad{H}_r=0
\end{equation}
\begin{equation}
    \begin{cases}
        \frac{\varepsilon_{\text{G}}}{r\sin\theta}\frac{\p}{\p\theta}(\sin\theta{E}_\theta)+\frac{\varepsilon_{\text{G}}}{r\sin\theta}\frac{\p}{\p\phi}({E}_\phi)=0\\
        \frac{\mu_{\text{G}}}{r\sin\theta}\frac{\p}{\p\theta}(\sin\theta{H}_\theta)+\frac{\mu_{\text{G}}}{r\sin\theta}\frac{\p}{\p\phi}({H}_\phi)=0\\
        \frac{1}{r\sin\theta}[\frac{\p}{\p\theta}(\sin\theta{E}_\phi)-\frac{\p}{\p\phi}({E}_\theta)]\vec{e}_r-\frac{1}{r}\frac{\p}{\p r}(r{E}_\phi)\vec{e}_\theta+\frac{1}{r}\frac{\p}{\p r}(r{E}_\theta)\vec{e}_\phi=-\mu_{\text{G}}\frac{\p}{\p t}({H}_\theta\vec{e}_\theta+{H}_\phi\vec{e}_\phi)\\
        \frac{1}{r\sin\theta}[\frac{\p}{\p\theta}(\sin\theta{H}_\phi)-\frac{\p}{\p\phi}({H}_\theta)]\vec{e}_r-\frac{1}{r}\frac{\p}{\p r}(r{H}_\phi)\vec{e}_\theta+\frac{1}{r}\frac{\p}{\p r}(r{H}_\theta)\vec{e}_\phi=+\varepsilon_{\text{G}}\frac{\p}{\p t}({E}_\theta\vec{e}_\theta+{E}_\phi\vec{e}_\phi)
    \end{cases}
\end{equation}
\begin{equation}
    \vec{E}={E}_\theta\vec{e}_\theta,\quad\vec{H}={H}_\phi\vec{e}_\phi
\end{equation}
\begin{equation}
    \begin{cases}
        \vec{\nabla}\cdot(\varepsilon_{\text{G}}\vec{E})=0\\
        \vec{\nabla}\cdot(\mu_{\text{G}}\vec{H})=0\\
        \vec{\nabla}\times\vec{E}=-\frac{\p}{\p t}(\mu_{\text{G}}\vec{H})\\
        \vec{\nabla}\times\vec{H}=+\frac{\p}{\p t}(\varepsilon_{\text{G}}\vec{E})
    \end{cases}
\end{equation}
\begin{equation}
    \begin{cases}
        \frac{\varepsilon_{\text{G}}}{r\sin\theta}\frac{\p}{\p\theta}(\sin\theta E_\theta)=0\\
        \frac{\mu_{\text{G}}}{r\sin\theta}\frac{\p}{\p\phi}({H}_\phi)=0\\
        -\frac{1}{r\sin\theta}\frac{\p}{\p\phi}({E}_\theta)\vec{e}_r+\frac{1}{r}\frac{\p}{\p r}(r{E}_\theta)\vec{e}_\phi=-\mu_{\text{G}}\frac{\p}{\p t}({H}_\phi)\vec{e}_\phi\\
        +\frac{1}{r\sin\theta}\frac{\p}{\p\theta}(\sin\theta{H}_\phi)\vec{e}_r-\frac{1}{r}\frac{\p}{\p r}(r{H}_\phi)\vec{e}_\theta=+\varepsilon_{\text{G}}\frac{\p}{\p t}({E}_\theta)\vec{e}_\theta\\
    \end{cases}
\end{equation}
\begin{equation}
    \begin{cases}
        \frac{\p}{\p r}(r{E}_\theta)+\mu_{\text{G}}\frac{\p}{\p t}(r{H}_\phi)=0\\
        \frac{\p}{\p r}(r{H}_\phi)+\varepsilon_{\text{G}}\frac{\p}{\p t}(r{E}_\theta)=0\\
    \end{cases}
\end{equation}
\begin{equation}
    \begin{cases}
        \mu_{\text{G}}\frac{\p}{\p r}\mu_{\text{G}}^{-1}\frac{\p}{\p r}(r{E}_\theta)-\varepsilon_{\text{G}}\mu_{\text{G}}\frac{\p}{\p t}\frac{\p}{\p t}(r{E}_\theta)=0\\
        \varepsilon_{\text{G}}\frac{\p}{\p r}\varepsilon_{\text{G}}^{-1}\frac{\p}{\p r}(r{H}_\phi)-\varepsilon_{\text{G}}\mu_{\text{G}}\frac{\p}{\p t}\frac{\p}{\p t}(r{H}_\phi)=0\\
    \end{cases}
\end{equation}
\begin{equation}
    \begin{cases}
        \mu_{\text{G}}\frac{\p}{\p r}\mu_{\text{G}}^{-1}\frac{\p}{\p r}(r{E}_\theta)-\frac{\p}{\p(v_\text{p}t)}\frac{\p}{\p(v_\text{p}t)}(r{E}_\theta)=0\\
        \varepsilon_{\text{G}}\frac{\p}{\p r}\varepsilon_{\text{G}}^{-1}\frac{\p}{\p r}(r{H}_\phi)-\frac{\p}{\p(v_\text{p}t)}\frac{\p}{\p(v_\text{p}t)}(r{H}_\phi)=0\\
    \end{cases}
\end{equation}
\begin{equation}
    \begin{cases}
        \frac{\p}{\p r}\frac{\p}{\p r}(r{E}_\theta)-\frac{\p}{\p r}(\ln\mu_{\text{G}})\frac{\p}{\p r}(r{E}_\theta)-\frac{\p}{\p(v_\text{p}t)}\frac{\p}{\p(v_\text{p}t)}(r{E}_\theta)=0\\
        \frac{\p}{\p r}\frac{\p}{\p r}(r{H}_\phi)-\frac{\p}{\p r}(\ln\varepsilon_{\text{G}})\frac{\p}{\p r}(r{H}_\phi)-\frac{\p}{\p(v_\text{p}t)}\frac{\p}{\p(v_\text{p}t)}(r{H}_\phi)=0\\
    \end{cases}
\end{equation}
\begin{equation}
    \frac{\p^2}{\p r^2}f(r,t)-p(r)\frac{\p}{\p r}f(r,t)-\frac{\p^2}{\p t^2}f(r,t)=0
\end{equation}
\begin{equation}
    \frac{\d^2}{\d r^2}f(r)-p(r)\frac{\d}{\d r}f(r)+f(r)=0
\end{equation}
\begin{equation}
    \frac{\d^2}{\d r^2}f(r)-p\frac{\d}{\d r}f(r)+f(r)=0
\end{equation}
\begin{equation}
    f(r)=e^{(p/2)r}[C_+e^{i\sqrt{1-(p/2)^2}r}+C_-e^{-i\sqrt{1-(p/2)^2}r}]
\end{equation}
\begin{equation}
    f(r)=e^{\int(p/2)\,\d r}[C_+e^{i\int\sqrt{1-(p/2)^2}\,\d r}+C_-e^{-i\int\sqrt{1-(p/2)^2}\,\d r}]
\end{equation}
\begin{equation}
    \begin{cases}
        r_2\left\lvert {E}_\theta\right\rvert_{r=r_2}=r_1\left\lvert {E}_\theta\right\rvert_{r=r_1}e^{\int_{r_1}^{r_2}\frac{1}{2}\frac{\p}{\p r}(\ln\mu_{\text{G}})\,\d r}\\
        r_2\left\lvert {H}_\phi\right\rvert_{r=r_2}=r_1\left\lvert {H}_\phi\right\rvert_{r=r_1}e^{\int_{r_1}^{r_2}\frac{1}{2}\frac{\p}{\p r}(\ln\varepsilon_{\text{G}})\,\d r}
    \end{cases}
\end{equation}
\begin{equation}
    \begin{cases}
        r_2\left\lvert {E}_\theta\right\rvert_{r=r_2}=r_1\left\lvert {E}_\theta\right\rvert_{r=r_1}\sqrt{\frac{\left\lvert \mu_{\text{G}}\right\rvert_{r=r_2}}{\left\lvert \mu_{\text{G}}\right\rvert_{r=r_1}}}\\
        r_2\left\lvert {H}_\phi\right\rvert_{r=r_2}=r_1\left\lvert {H}_\phi\right\rvert_{r=r_1}\sqrt{\frac{\left\lvert \varepsilon_{\text{G}}\right\rvert_{r=r_2}}{\left\lvert \varepsilon_{\text{G}}\right\rvert_{r=r_1}}}
    \end{cases}
\end{equation}
\begin{equation}
    L|_{r=r_2}=L|_{r=r_1}\frac{v_\text{p}|_{r=r_1}}{v_\text{p}|_{r=r_2}}
\end{equation}
\begin{equation}
    \begin{cases}
        \Delta t_{{E}_\theta}|_{r=r_1}^{r=r_2}=\int_{r_1}^{r_2}\frac{\sqrt{1-[\frac{1}{2}\frac{\p}{\p r}(\ln\mu_{\text{G}})]^2}}{v_\text{p}}\d r\\
        \Delta t_{{H}_\phi}|_{r=r_1}^{r=r_2}=\int_{r_1}^{r_2}\frac{\sqrt{1-[\frac{1}{2}\frac{\p}{\p r}(\ln\varepsilon_{\text{G}})]^2}}{v_\text{p}}\d r\\
    \end{cases}
\end{equation}

\begin{equation}
    S_\text{G}\propto\dot{h}^2\propto\omega^2 h^2
\end{equation}
\begin{equation}
    S_\text{G}\propto\frac{c^3}{G}\omega^2h^2
\end{equation}

双星系统引力辐射本为
\begin{equation}
    h=\frac{\mathcal{M}[\pi \mathcal{M}F(t)]^{2/3}}{r}Q(\theta,\phi,\psi,\iota)\cos[\int 2\pi F(t)\,\d t]
\end{equation}
设双星系统常量$c^*$, $G^*$, 一观者临近双星系统且与双星系统相对静止, 其与双星系统距离为$r$, 测得强度$h_r$, 频率$F_r$, 则\footnote{$\mathcal{M}$和$c^*$, $G^*$简并, 所以可以笼统地仍记作$\mathcal{M}$.}
\begin{equation}
    h_r=\frac{\mathcal{M}[\pi \mathcal{M}F_r(t)]^{2/3}}{r/c^*}Q(\theta,\phi,\psi,\iota)
\end{equation}
与双星系统距离为$r$的观者测得的引力辐射光度$L_r\propto4\pi r^2({c^*}^3/G^*)F_r^2 h_r^2$, 设地球观者与双星系统距离为$d$, 双星系统红移为$z$, 测得强度$h_d$, 频率$F_d$, 则地球观者测得的引力辐射光度正比于$L_d\propto4\pi d^2(c^3/G)F_d^2 h_d^2$, 且有$L_d=(c^*/c)L_r/(1+z)^2$, 所以$r^2({c^*}^3/G^*)F_r^2 h_r^2/(1+z)^2=(c^*/c)d^2(c^3/G)F_d^2 h_d^2$,又有$F_d=F_r/(1+z)$, 所以$r^2({c^*}^3/G^*)h_r^2=(c^*/c)d^2(c^3/G)h_d^2$, 则
\begin{align}
    h_d&=\sqrt{\frac{{c^*}^2/G^*}{c^2/G}}\frac{r^2}{d^2}h_r\\
    &=\sqrt{\frac{{c^*}^2/G^*}{c^2/G}}\frac{\mathcal{M}[\pi \mathcal{M}F_r(t)]^{2/3}}{d/c^*}Q(\theta,\phi,\psi,\iota)
\end{align}
所以地球观者测得
\begin{equation}
    h=\sqrt{\frac{{c^*}^2/G^*}{c^2/G}}\frac{\mathcal{M}[\pi \mathcal{M}F_r(t)]^{2/3}}{d/c^*}Q(\theta,\phi,\psi,\iota)\cos[\int 2\pi \frac{F_r(t)}{1+z}\,\d t]
\end{equation}
记$F_\text{obs}(t)=F_r(t)/(1+z)$, $\mathcal{M}_\text{obs}=\mathcal{M}(1+z)$, 光度距离$d_\text{L}=d(1+z)$, 则
\begin{align}
    h&=\sqrt{\frac{{c^*}^2/G^*}{c^2/G}}\frac{\mathcal{M}[\pi \mathcal{M}F_r(t)]^{2/3}}{d(1+z)/c^*}Q(\theta,\phi,\psi,\iota)\cos[\int 2\pi F_\text{obs}(t)\,\d t]\\
    &=\sqrt{\frac{{c^*}^2/G^*}{c^2/G}}\frac{\mathcal{M}_\text{obs}[\pi \mathcal{M}_\text{obs}F_\text{obs}(t)]^{2/3}}{d_\text{L}/c^*}Q(\theta,\phi,\psi,\iota)\cos[\int 2\pi F_\text{obs}(t)\,\d t]\\
    &=\sqrt{\frac{{c^*}^4/G^*}{c^4/G}}\frac{\mathcal{M}_\text{obs}[\pi \mathcal{M}_\text{obs}F_\text{obs}(t)]^{2/3}}{d_\text{L}/c}Q(\theta,\phi,\psi,\iota)\cos[\int 2\pi F_\text{obs}(t)\,\d t]
\end{align}
用引力波测距测得$d_\text{L,G}$, 则
\begin{equation}
    d_\text{L,G}=d_\text{L}\sqrt{\frac{c^4/G}{{c^*}^4/G^*}}
\end{equation}

\cite{Poisson1995}
\begin{equation}
    h(t)=\frac{\mathcal{M}[\pi \mathcal{M}F(t)]^{2/3}}{\xi\,d_\text{L}}Q(\text{angles})\cos\Phi(t)
\end{equation}
\begin{equation}
    \tilde{h}(f)=\frac{\sqrt{30}}{48\pi^{2/3}}\frac{\mathcal{M}^{5/6}Q}{\xi\,d_\text{L}}f^{-7/6}e^{i[2\pi ft(f)-\Phi(f)-\frac{\pi}{4}]}
\end{equation}
问题转化为估计$\xi$
\begin{equation}
    p(\mu)\propto p^{(0)}(\mu)\exp[
        -\frac{1}{2}\Gamma_{ab}(\mu^a-\hat{\mu}^a)(\mu^b-\hat{\mu}^b)
    ]
\end{equation}
\begin{equation}
    p^{(0)}(\mu)\propto\exp[
        -\frac{1}{2}\Gamma^{(0)}_{ab}(\mu^a-\bar{\mu}^a)(\mu^b-\bar{\mu}^b)
    ]
\end{equation}
设待估参数为$\mu=(\ln\xi,\ln (d_\text{L}/{d_\text{L}}_0),\ln Q,\dots)$, $\dots$为其他参数(如$\mathcal{M}$), 则$\tilde{h}_{,\ln\xi}=\tilde{h}_{,\ln (d_\text{L}/{d_\text{L}}_0)}=-\tilde{h}_{,\ln Q}=-\tilde{h}$, $\tilde{h}$对其他参数求偏导皆为纯虚数, 则由$\Gamma_{ab}=\left\langle h_{,a}|h_{,b}\right\rangle $和$\text{SNR}:=\rho=\sqrt{\left\langle h|h\right\rangle}$得
\begin{equation}
    \Gamma_{ab}=\begin{bmatrix}
        \rho^2&\rho^2&-\rho^2&0&\dots\\
        \rho^2&\rho^2&-\rho^2&0&\dots\\
        -\rho^2&-\rho^2&\rho^2&0&\dots\\
        0&0&0&?&\ldots\\
        \vdots&\vdots&\vdots&\vdots&\ddots 
    \end{bmatrix}
\end{equation}
又设
\begin{equation}
    \Gamma^{(0)}_{ab}=\begin{bmatrix}
        0&0&0&0&\dots\\
        0&1/\sigma_{\ln d_\text{L}}^2&0&0&\dots\\
        0&0&1/\sigma_{\ln Q}^2&0&\dots\\
        0 &0&0&0&\ldots\\
        \vdots&\vdots&\vdots&\vdots&\ddots 
    \end{bmatrix}
\end{equation}
则由$\Sigma_{ab}=(\Gamma^{(0)}_{ab}+\Gamma_{ab})^{-1}$得
\begin{equation}
    \Sigma_{ab}=\begin{bmatrix}
        \begin{bmatrix}
            \rho^2&\rho^2&-\rho^2\\
            \rho^2&\rho^2+1/\sigma_{\ln (d_\text{L}/{d_\text{L}}_0)}^2&-\rho^2\\
            -\rho^2&-\rho^2&\rho^2+1/\sigma_{\ln Q}^2
        \end{bmatrix}^{-1}&0\\
        0&[?]^{-1}
    \end{bmatrix}
\end{equation}
而
\begin{align}
    &\begin{bmatrix}
        \rho^2&\rho^2&-\rho^2\\
        \rho^2&\rho^2+1/\sigma_{\ln (d_\text{L}/{d_\text{L}}_0)}^2&-\rho^2\\
        -\rho^2&-\rho^2&\rho^2+1/\sigma_{\ln Q}^2
    \end{bmatrix}^{-1}\\
    &=\begin{bmatrix}
        1/\rho^2+\sigma_{\ln (d_\text{L}/{d_\text{L}}_0)}^2+\sigma_{\ln Q}^2&-\sigma_{\ln (d_\text{L}/{d_\text{L}}_0)}^2&\sigma_{\ln Q}^2\\
        -\sigma_{\ln (d_\text{L}/{d_\text{L}}_0)}^2&\sigma_{\ln (d_\text{L}/{d_\text{L}}_0)}^2&0\\
        \sigma_{\ln Q}^2&0&\sigma_{\ln Q}^2
    \end{bmatrix}
\end{align}

    \chapter{能量}

\section{共形无限远}

类时无限远是点

类光无限远是3维面

类空无限远是点

\section{共形规范}

$\Omega$的选择有任意性, 每种选择称为一种共形规范

    \chapter{多极矩}

\section{Quadrupole Approximation}

\cite{Wald1984}. 由\eqref{lin_gravity}得
\begin{equation}
    \bar{h}_{\mu\nu}(t,\vec{r}) = 4\int 
    \frac{T_{\mu\nu}(t-\left\lvert\vec{r}-\vec{r}'\right\rvert,\vec{r}')}{\left\lvert\vec{r}-\vec{r}'\right\rvert}\,\d V'.
\end{equation}
\begin{align}
    \hat{\bar{h}}_{\mu\nu}(\omega,\vec{r})&:=\frac{1}{\sqrt{2\pi}}
    \int\bar{h}_{\mu\nu}(t,\vec{r})e^{i\omega t}\d t\\
    &=4\int 
    \frac{\hat{T}_{\mu\nu}(\omega,\vec{r}')}{\left\lvert\vec{r}-\vec{r}'\right\rvert}e^{i\omega\left\lvert\vec{r}-\vec{r}'\right\rvert}\,\d V'.
\end{align}
由$\p^{\nu}\bar{h}_{\mu\nu}=0$,
\begin{equation}
    -i\omega\hat{\bar{h}}_{0\mu}=\sum_{i}\frac{\p\hat{\bar{h}}_{i\mu}}{\p x^{i}}.
\end{equation}
$\left\lvert \vec{r}\right\rvert \gg \left\lvert \vec{r}'\right\rvert$且$\omega\ll1/\left\lvert \vec{r}'\right\rvert$,
\begin{equation}
    \hat{\bar{h}}_{ij}(\omega,\vec{r}) = 4
    \frac{e^{i\omega\left\lvert\vec{r}\right\rvert}}{\left\lvert\vec{r}\right\rvert}
    \int \hat{T}_{ij}(\omega,\vec{r}')\,\d V'.
\end{equation}
\begin{align}
    \int \hat{T}_{ij}\,\d V'
    &=\int \sum_{k}(\hat{T}_{kj}\frac{\p x'^i}{\p x'^k})\,\d V'\\
    &= \sum_{k}\left[\int \frac{\p }{\p x'^k}(\hat{T}_{kj}x'^i)\,\d V'-\int \frac{\p \hat{T}_{kj}}{\p x'^k}x'^i\,\d V'\right]\\
    &= \sum_{k}\int \p'_k\,(\hat{T}_{kj}x'^i)\,\d V' - \sum_{k}\int \frac{\p \hat{T}_{kj}}{\p x'^k}x'^i\,\d V'\\
    &= \int \hat{T}_{kj}x'^i\,\d S' - \sum_{k}\int \frac{\p \hat{T}_{kj}}{\p x'^k}x'^i\,\d V'\\
    &= -\sum_{k}\int \frac{\p \hat{T}_{kj}}{\p x'^k}x'^i\,\d V'\\
    &= -\int (\sum_{k}\p'_k\hat{T}_{kj})x'^i\,\d V'\\
    &= -\int (\p_0\hat{T}_{0j})x'^i\,\d V'\\
    &= -i\omega\int \hat{T}_{0j}x'^i\,\d V'\\
    &= \int \hat{T}_{(ij)}\,\d V'\\
    &= -i\omega\int \hat{T}_{0(j}x'^{i)}\,\d V'\\
    &= -\frac{i\omega}{2}\int (\hat{T}_{0j}x'^{i}+\hat{T}_{0i}x'^{j})\,\d V',
\end{align}
\begin{align}
    -\frac{i\omega}{2}\int (\hat{T}_{0j}x'^{i}+\hat{T}_{0i}x'^{j})\,\d V'
    &= -\frac{i\omega}{2}\int \sum_{k}(\hat{T}_{0k}x'^{i}\frac{\p x'^{j}}{\p x'^{k}}+\hat{T}_{0k}\frac{\p x'^{i}}{\p x'^{k}}x'^{j})\,\d V'\\
    &= -\frac{i\omega}{2}\sum_{k}\left[\int \frac{\p }{\p x'^k}(\hat{T}_{0k}x'^ix'^j)\,\d V'-\int \frac{\p \hat{T}_{0k}}{\p x'^k}x'^ix'^j\,\d V'\right]\\
    &= -\frac{i\omega}{2}\sum_{k}\int \p'_k\,(\hat{T}_{0k}x'^ix'^j)\,\d V' +\frac{i\omega}{2}\sum_{k}\int \frac{\p \hat{T}_{0k}}{\p x'^k}x'^ix'^j\,\d V'\\
    &= -\frac{i\omega}{2}\sum_{k}\int \hat{T}_{0k}x'^ix'^j\,\d S' +\frac{i\omega}{2}\sum_{k}\int \frac{\p \hat{T}_{0k}}{\p x'^k}x'^ix'^j\,\d V'\\
    &= \frac{i\omega}{2}\sum_{k}\int \frac{\p \hat{T}_{0k}}{\p x'^k}x'^ix'^j\,\d V'\\
    &= \frac{i\omega}{2}\int (\sum_{k}\p'_k\hat{T}_{0k})x'^ix'^j\,\d V'\\
    &= \frac{i\omega}{2}\int (\p_0\hat{T}_{00})x'^ix'^j\,\d V'\\
    &= -\frac{\omega^2}{2}\int \hat{T}_{00}\,x'^ix'^j\,\d V'.
\end{align}
\begin{equation}
    q_{ij}(t):=\int{T}_{00}\,x'^ix'^j\,\d V',
\end{equation}
\begin{equation}
    \hat{\bar{h}}_{ij}(\omega,\vec{r}) = -2\omega^2\frac{e^{i\omega\left\lvert\vec{r}\right\rvert}}{\left\lvert\vec{r}\right\rvert}\hat{q}_{ij}(\omega),
\end{equation}
\begin{equation}
    {\bar{h}}_{ij}(t,\vec{r}) = \frac{2}{\left\lvert\vec{r}\right\rvert}
    \frac{\d^2}{\d t^2}{q}_{ij}(t-\left\lvert\vec{r}\right\rvert).\label{qf}
\end{equation}

\section{电磁---引力对比}

\begin{equation}
    A_\mu(t,\vec{r})=\frac{\mu_0}{4\pi}\int\frac{J_\mu(t-\left\lvert \vec{r}-\vec{r}'\right\rvert ,\vec{r}')}{\left\lvert \vec{r}-\vec{r}'\right\rvert}\d V'
\end{equation}
\begin{equation}
    \bar{h}_{\mu\nu}(t,\vec{r})=4G\int\frac{T_{\mu\nu}(t-\left\lvert\vec{r}-\vec{r}'\right\rvert,\vec{r}')}{\left\lvert\vec{r}-\vec{r}'\right\rvert}\,\d V'
\end{equation}

\begin{equation}
    A_\mu(t,\vec{r})=\frac{1}{\sqrt{2\pi}}
    \int\hat{A}_{\mu}(\omega,\vec{r})e^{-i\omega t}\d t
\end{equation}
\begin{equation}
    \bar{h}_{\mu\nu}(t,\vec{r})=\frac{1}{\sqrt{2\pi}}
    \int\hat{\bar{h}}_{\mu\nu}(\omega,\vec{r})e^{-i\omega t}\d t
\end{equation}

\begin{equation}
    \hat{A}_{\mu}(\omega,\vec{r})=\frac{\mu_0}{4\pi}\int\frac{\hat{J}_{\mu}(\omega,\vec{r}')}{\left\lvert\vec{r}-\vec{r}'\right\rvert}e^{i\omega\left\lvert\vec{r}-\vec{r}'\right\rvert}\,\d V'
\end{equation}
\begin{equation}
    \hat{\bar{h}}_{\mu\nu}(\omega,\vec{r})=4G\int\frac{\hat{T}_{\mu\nu}(\omega,\vec{r}')}{\left\lvert\vec{r}-\vec{r}'\right\rvert}e^{i\omega\left\lvert\vec{r}-\vec{r}'\right\rvert}\,\d V'
\end{equation}

\begin{equation}
    \hat{A}_{\mu}(\omega,\vec{r})=\frac{\mu_0}{4\pi}\frac{e^{i\omega\left\lvert\vec{r}\right\rvert}}{\left\lvert\vec{r}\right\rvert}\int\hat{J}_{\mu}(\omega,\vec{r}')e^{-i\omega(\frac{\vec{r}}{\left\lvert\vec{r}\right\rvert}\cdot\vec{r}')}\,\d V'
\end{equation}
\begin{equation}
    \hat{\bar{h}}_{\mu\nu}(\omega,\vec{r})=4G\frac{e^{i\omega\left\lvert\vec{r}\right\rvert}}{\left\lvert\vec{r}\right\rvert}\int\hat{T}_{\mu\nu}(\omega,\vec{r}')e^{-i\omega(\frac{\vec{r}}{\left\lvert\vec{r}\right\rvert}\cdot\vec{r}')}\,\d V'
\end{equation}

\begin{equation}
    \hat{A}_{\mu}(\omega,\vec{r})=\frac{\mu_0}{4\pi}\frac{e^{i\omega\left\lvert\vec{r}\right\rvert}}{\left\lvert\vec{r}\right\rvert}\int\hat{J}_{\mu}(\omega,\vec{r}')\left[1-i\omega(\frac{\vec{r}}{\left\lvert\vec{r}\right\rvert}\cdot\vec{r}')-\dots\right]\,\d V'
\end{equation}
\begin{equation}
    \hat{\bar{h}}_{\mu\nu}(\omega,\vec{r})=4G\frac{e^{i\omega\left\lvert\vec{r}\right\rvert}}{\left\lvert\vec{r}\right\rvert}\int\hat{T}_{\mu\nu}(\omega,\vec{r}')\left[1-i\omega(\frac{\vec{r}}{\left\lvert\vec{r}\right\rvert}\cdot\vec{r}')-\dots\right]\,\d V'
\end{equation}

\subsection{电偶极---引力对比}

\begin{equation}
    \hat{A}_i=\frac{\mu_0}{4\pi}\frac{e^{i\omega\left\lvert\vec{r}\right\rvert}}{\left\lvert\vec{r}\right\rvert}\int\hat{J}_i\,\d V'
\end{equation}
\begin{equation}
    \hat{\bar{h}}_{ij}=4G\frac{e^{i\omega\left\lvert\vec{r}\right\rvert}}{\left\lvert\vec{r}\right\rvert}\int\hat{T}_{ij}\,\d V'
\end{equation}

\begin{equation}
    \int \hat{J}_{i}\,\d V'=-i\omega\int\hat{J}_{0}x'^i\,\d V'
\end{equation}
\begin{equation}
    \int\hat{T}_{ij}\,\d V'=-\frac{\omega^2}{2}\int \hat{T}_{00}\,x'^ix'^j\,\d V'
\end{equation}

\begin{equation}
    \hat{p}_i=\int\hat{J}_{0}x'^i\,\d V'
\end{equation}
\begin{equation}
    \hat{q}_{ij}=\int \hat{T}_{00}\,x'^ix'^j\,\d V'
\end{equation}

\begin{equation}
    \hat{A}_i=\frac{\mu_0}{4\pi}\frac{e^{i\omega\left\lvert\vec{r}\right\rvert}}{\left\lvert\vec{r}\right\rvert}(-i\omega \hat{p}_i)
\end{equation}
\begin{equation}
    \hat{\bar{h}}_{ij}=4G\frac{e^{i\omega\left\lvert\vec{r}\right\rvert}}{\left\lvert\vec{r}\right\rvert}(-\frac{\omega^2}{2}\hat{q}_{ij})
\end{equation}

\begin{equation}
    {A}_i=\frac{\mu_0}{4\pi}\frac{1}{\left\lvert\vec{r}\right\rvert}\frac{\d}{\d t}p_i(t-\left\lvert\vec{r}\right\rvert)
\end{equation}
\begin{equation}
    {\bar{h}}_{ij}=4G\frac{1}{\left\lvert\vec{r}\right\rvert}\frac{1}{2}\frac{\d^2}{\d t^2}q_{ij}(t-\left\lvert\vec{r}\right\rvert)
\end{equation}

\subsection{电四极---引力对比}

\begin{equation}
    \hat{A}_{i}(\omega,\vec{r})=\frac{\mu_0}{4\pi}\frac{e^{i\omega\left\lvert\vec{r}\right\rvert}}{\left\lvert\vec{r}\right\rvert}()(-i\omega)\int\hat{J}_{i}(\omega,\vec{r}')(\frac{\vec{r}}{\left\lvert\vec{r}\right\rvert}\cdot\vec{r}')\,\d V'
\end{equation}

\begin{equation}
    \hat{A}_i=\frac{\mu_0}{4\pi}\frac{e^{i\omega\left\lvert\vec{r}\right\rvert}}{\left\lvert\vec{r}\right\rvert}(-i\omega)\int\hat{J}_i'n^j{x}_j'\,\d V'
\end{equation}

\begin{equation}
    \hat{A}_i=\frac{\mu_0}{4\pi}\frac{e^{i\omega\left\lvert\vec{r}\right\rvert}}{\left\lvert\vec{r}\right\rvert}(-i\omega)\int n^j{x}_j'\hat{J}_i'\,\d V'
\end{equation}

\begin{equation}
    \hat{A}_i=\frac{\mu_0}{4\pi}\frac{e^{i\omega\left\lvert\vec{r}\right\rvert}}{\left\lvert\vec{r}\right\rvert}(-i\omega)n^j\!\left[\int {x}_{(j}'\hat{J}_{i)}'\,\d V'\right]
\end{equation}

\begin{align}
    \int {x}_{(j}'\hat{J}_{i)}'\,\d V'
    &= \frac{1}{2}\int (\hat{J}_{j}'{x}_{i}'+\hat{J}_{i}'{x}_{j}')\,\d V'\\
    &= \frac{1}{2}\int \sum_{k}(\hat{J}_{k}'x'^{i}\frac{\p x'^{j}}{\p x'^{k}}+\hat{J}_{k}'\frac{\p x'^{i}}{\p x'^{k}}x'^{j})\,\d V'\\
    &= \frac{1}{2}\sum_{k}\left[\int \frac{\p }{\p x'^k}(\hat{J}_{k}'x'^ix'^j)\,\d V'-\int \frac{\p \hat{J}_{k}'}{\p x'^k}x'^ix'^j\,\d V'\right]\\
    &= \frac{1}{2}\sum_{k}\int \p'_k\,(\hat{J}_{k}'x'^ix'^j)\,\d V' -\frac{1}{2}\sum_{k}\int \frac{\p \hat{J}_{k}'}{\p x'^k}x'^ix'^j\,\d V'\\
    &= \frac{1}{2}\sum_{k}\int \hat{J}_{k}'x'^ix'^j\,\d S' -\frac{1}{2}\sum_{k}\int \frac{\p \hat{J}_{k}'}{\p x'^k}x'^ix'^j\,\d V'\\
    &= -\frac{1}{2}\sum_{k}\int \frac{\p \hat{J}_{k}'}{\p x'^k}x'^ix'^j\,\d V'\\
    &= -\frac{1}{2}\int (\sum_{k}\p'_k\hat{J}_{k}')x'^ix'^j\,\d V'\\
    &= -\frac{1}{2}\int (\p_0\hat{J}_{0}')x'^ix'^j\,\d V'\\
    &= -\frac{i\omega}{2}\int \hat{J}_{0}'x'^ix'^j\,\d V'
\end{align}

\begin{equation}
    \hat{D}_{ij}=\int \hat{J}_{0}'\,x'^ix'^j\,\d V'
\end{equation}

\begin{equation}
    \hat{A}_i=\frac{\mu_0}{4\pi}\frac{e^{i\omega\left\lvert\vec{r}\right\rvert}}{\left\lvert\vec{r}\right\rvert}(-\frac{\omega^2}{2}n^j\hat{D}_{ij}
    )
\end{equation}

\begin{equation}
    {A}_i=\frac{\mu_0}{4\pi}\frac{1}{\left\lvert\vec{r}\right\rvert}n^j\frac{1}{2}\frac{\d^2}{\d t^2}D_{ij}(t-\left\lvert\vec{r}\right\rvert)
\end{equation}

    \chapter{双星系统}

\section{基本公式}

\def\M{\mathcal{M}}
\begin{equation}
    \M:=\mu^{3/5}M^{2/5}
\end{equation}
\begin{equation}
    h_+=\frac{4\M}{D}[\pi \M F(t)]^{2/3}\frac{1+\cos^2\iota}{2}\cos\Phi(t)
\end{equation}
\begin{equation}
    h_\times=\frac{4\M}{D}[\pi \M F(t)]^{2/3}\cos\iota\sin\Phi(t)
\end{equation}
\begin{equation}
    h=F_+h_++F_\times h_\times
\end{equation}

\section{Post-Newtonian Approximation}

2PN: \cite{Blanchet1995,Poisson1995}

\section{Stationary Phase Approximation}

\cite{Poisson1995}, if $\zeta(t)$ varies slowly near $t=t_0$ where the phase has a stationary point: $\phi'(t_0)=0$,
\begin{align}
    \int\zeta(t)e^{i\phi(t;f)}\,\d t&=\int\zeta(t)e^{i[\phi(t_0)+\phi'(t_0)(t-t_0)+\frac{1}{2}\phi''(t_0)(t-t_0)^2+\dots]}\,\d t\\
    &\simeq e^{i\phi(t_0)}\int\zeta(t)e^{i[\frac{1}{2}\phi''(t_0)(t-t_0)^2]}\,\d t\\
    &\simeq e^{i\phi(t_0)}\int\zeta(t_0)e^{\frac{-\sqrt{-i\phi''(t_0)}^2(t-t_0)^2}{2}}\,\d t\\
    &=\frac{\sqrt{2\pi}}{\sqrt{-i\phi''(t_0)}}\zeta(t_0)e^{i\phi(t_0)}.
\end{align}
\begin{align}
    h&=\frac{\mathcal{M}}{D}[\pi \mathcal{M}F(t)]^{2/3}Q\cos\Phi(t)\\
    &=\frac{\mathcal{M}}{D}[\pi \mathcal{M}F(t)]^{2/3}Q\frac{1}{2}[e^{i\Phi(t)}+e^{-i\Phi(t)}]
\end{align}
\begin{align}
    \tilde{h}(f)&=\int h(t)e^{i2\pi ft}\,\d t\\
    &=\int \frac{\mathcal{M}}{D}[\pi \mathcal{M}F(t)]^{2/3}Q\frac{1}{2}[e^{i\Phi(t)}+e^{-i\Phi(t)}]e^{i2\pi ft}\,\d t\\
    &=\int\frac{\mathcal{M}}{D}[\pi \mathcal{M}F(t)]^{2/3}Q\frac{1}{2}\{e^{i[2\pi ft+\Phi(t)]}+e^{i[2\pi ft-\Phi(t)]}\}\,\d t\\
    &\simeq\int\frac{\mathcal{M}}{D}[\pi \mathcal{M}F(t)]^{2/3}Q\frac{1}{2}e^{i[2\pi ft-\Phi(t)]}\,\d t\\
    &=\int\frac{\mathcal{M}}{D}[\pi \mathcal{M}F]^{2/3}Q\frac{1}{2}e^{i[2\pi ft(F)-\Phi(F)]}\frac{\d t}{\d F}\,\d F\\
    &\simeq\frac{\sqrt{2\pi}}{\sqrt{-i[2\pi ft(F)-\Phi(F)]''_{F=f}}}\\
    &\left[\frac{\mathcal{M}}{D}(\pi \mathcal{M}F)^{2/3}Q\frac{1}{2}\frac{\d t}{\d F}\right]_{F=f}e^{i[2\pi ft(f)-\Phi(f)]}\\
    &\simeq\frac{\sqrt{2\pi}}{\sqrt{-i\left\{2\pi f\left[-\frac{5}{256}\M(\pi \M F)^{-8/3}\right]-\left[\frac{1}{16}(\pi\M F)^{-5/3}\right]\right\}''_{F=f}}}\\
    &\left\{\frac{\mathcal{M}}{D}(\pi \mathcal{M}F)^{2/3}Q\frac{1}{2}\left[\frac{5\pi\M^2}{96}(\pi\M F)^{-11/3}\right]\right\}_{F=f}e^{i[2\pi ft(f)-\Phi(f)]}\\
    &=\frac{\sqrt{30}}{48\pi^{2/3}}\frac{\mathcal{M}^{5/6}Q}{D}f^{-7/6}e^{i[2\pi ft(f)-\Phi(f)-\frac{\pi}{4}]}\quad(\texttt{pnspa.py})
\end{align}

或\cite{Arun2005}, $h(t)=2A(t)\cos\phi(t)$, $\d\ln A/\d t\ll\d\phi/\d t$且$\left\lvert \d^2\phi/\d t^2\right\rvert \ll(\d\phi/\d t)^2$.

    \chapter{宇宙学效应}

\cite{Maggiore2014},
\begin{equation}
    \frac{\d\eta}{\d(ct)}=\frac{1}{a},
\end{equation}
\begin{equation}
    \d s^2=-\d(ct)^2+a^2[\frac{\d r^2}{1-kr^2}+r^2(\d\theta^2+\sin^2\theta\d\phi^2)],
\end{equation}
\begin{equation}
    \d s^2=a^2[-\d\eta^2+\frac{\d r^2}{1-kr^2}+r^2(\d\theta^2+\sin^2\theta\d\phi^2)].
\end{equation}
$\square\Phi=0$, $\Phi:=f/r$, $f:=g/a$,
\begin{equation}
    \p_r^2g+(\p_\eta^2a/a)g-\p_\eta^2g=0,
\end{equation}
$\p_\eta^2a/a\sim\eta^2$, $\omega^2\gg1/\eta^2$,
\begin{equation}
    g\simeq e^{i\omega(\eta-r)}.
\end{equation}

    \chapter{干涉仪}

\cite{Maggiore2014}. 设入射电场$\vec{E}_\text{in}=\vec{E}_0e^{-i\omega_\text{L}t+i\vec{k}_\text{L}\cdot\vec{x}}$. 设splitter在$\vec{x}=0$处, 则$\vec{E}_\text{in}=\vec{E}_0e^{-i\omega_\text{L}t}$. $\vec{E}_\text{out}=\vec{E}_\text{form x}+\vec{E}_\text{form y}$, $t$时的$\vec{E}_\text{form x}$在$t-\frac{2L_x}{c}$时入splitter, $t$时的$\vec{E}_\text{form y}$在$t-\frac{2L_y}{c}$时入splitter, 考虑反相, $\vec{E}_\text{form x}=-\frac{1}{2}\vec{E}_0e^{-i\omega_\text{L}t+2ik_\text{L}L_x}$, $\vec{E}_\text{form y}=+\frac{1}{2}\vec{E}_0e^{-i\omega_\text{L}t+2ik_\text{L}L_y}$, $\vec{E}_\text{out}=\vec{E}_0\sin(\phi_0)e^{-i\omega_\text{L}(t-\frac{2L}{c})-i\frac{\pi}{2}}$, where $\phi_0=k_\text{L}(L_y-L_x)$ and $L=(L_x+L_y)/2$.

用TT frame计算. 设splitter在$(0,0)$, reflector x在$(L_x,0)$, reflector y在$(0,L_y)$, 显然无GW时如上.

设GW只有$+$mode且方向为$z_+$, $h_+=h_0\cos[\omega_\text{gw}(t-z/c)]$, 
\begin{equation}
    \d s^2=-c^2\d t^2+(1 +h_+)\d x^2+ (1-h_+)\d y^2+\d z^2.
\end{equation}
$h_+(t):=h_+|_{z=0}$. 光$\d s^2=0$, 保留一阶项, x方向光轨迹
\begin{equation}
    \d x=\pm c\d t[1-\frac{1}{2}h_+(t)],
\end{equation}
y方向光轨迹
\begin{equation}
    \d y=\pm c\d t[1+\frac{1}{2}h_+(t)],
\end{equation}
$+$号是splitter到reflector, $-$号是reflector到splitter.

设photon $t=t_0$到splitter, $t=t_1$到x reflector, $t=t_2$到splitter, 则
\begin{eqnarray}
    t_2-t_0=\frac{2L_x}{c}+\frac{1}{2}\int^{t_2}_{t_0}\d t'h_+(t')\\
    \approx\frac{2L_x}{c}+\frac{1}{2}\int^{t_0+\frac{2L_x}{c}}_{t_0}\d t'h_+(t')\\
    =\frac{2L_x}{c}+\frac{L_x}{c}h_+(t_0+\frac{L_x}{c})\sinc(\omega_\text{gw}\frac{L_x}{c}).
\end{eqnarray}
$\omega_\text{gw}\frac{L_x}{c}\ll1$, $t_2-t_0\approx\frac{2L_x}{c}+\frac{L_x}{c}h_+(t_1)$. $\omega_\text{gw}\frac{L_x}{c}\gg1$, $t_2-t_0\approx\frac{2L_x}{c}$.

y方向, x改成y, $+h_+$改成$-h_+$.

$\vec{E}_\text{in}=\vec{E}_0e^{-i\omega_\text{L}t}$, $t$时的$\vec{E}_\text{form x}$在$t-\frac{2L_x}{c}-\frac{L_x}{c}h_+(t-\frac{L_x}{c})\sinc(\omega_\text{gw}\frac{L_x}{c})$时入splitter, $t$时的$\vec{E}_\text{form y}$在$t-\frac{2L_y}{c}+\frac{L_y}{c}h_+(t-\frac{L_y}{c})\sinc(\omega_\text{gw}\frac{L_y}{c})$时入splitter, $\vec{E}_\text{form x}=-\frac{1}{2}\vec{E}_0e^{-i\omega_\text{L}(t-\frac{2L}{c})+i\phi_0+i\Delta\phi(t)}$, $\vec{E}_\text{form y}=+\frac{1}{2}\vec{E}_0e^{-i\omega_\text{L}(t-\frac{2L}{c})-i\phi_0-i\Delta\phi(t)}$, where $\phi_0=k_\text{L}(L_y-L_x)$, $\Delta\phi(t)=h_+(t-\frac{L}{c})k_\text{L}L\sinc(\omega_\text{gw}\frac{L}{c})$, and $L=(L_x+L_y)/2$. Finally, $\vec{E}_\text{out}=\vec{E}_0\sin[\phi_0+\Delta\phi(t)]e^{-i\omega_\text{L}(t-\frac{2L}{c})-i\frac{\pi}{2}}$.

    \chapter{数据分析}

\cite{Finn1992}, \cite{Maggiore2014}.

\def\la{\langle}
\def\ra{\rangle}
\begin{equation}
    R(\tau):=\text{E}(N_tN_{t+\tau}),
\end{equation}
\begin{equation}
    \frac{1}{2}S_{N}(f):=\tilde{R}(f):=\int R(\tau)e^{i2\pi f\tau}\,\d\tau.
\end{equation}
\begin{equation}
    \la p|q \ra:=4\text{Re}\int_0^\infty\frac{\tilde{p}^*(f)\tilde{q}(f)}{S_{N}(f)}\,\d f.
\end{equation}

\section{matched filtering}

\begin{equation}
    \hat{\mathcal{S}}:=\int S_tK(t)\,\d t
\end{equation}
\begin{align}
    \frac{\mathcal{S}}{\mathcal{N}}&:=\frac{\text{E}(\int (h(t)+N_t)K(t)\,\d t)}{\sqrt{\text{D}(\int N_tK(t)\,\d t)}}\\
    &=\frac{\int h(t)K(t)\,\d t}{\sqrt{\int \text{E}(N_{t_1}N_{t_2})K(t_1)K(t_2)\,\d t_1\d t_2}}\\
    &=\frac{\int h(t)K(t)\,\d t}{\sqrt{\int R(t_2-t_1)K(t_1)K(t_2)\,\d t_1\d t_2}}\\
    &=\frac{\int \ti{h}(f)\ti{K}^*(f)\,\d f}{\sqrt{\int\frac{1}{2}S_{N}(f)\ti{K}(f)\ti{K}^*(f)\,\d f}}\\
    &=\frac{\la \frac{1}{2}S_{N}\ti{K}|h \ra}{\la \frac{1}{2}S_{N}\ti{K}|\frac{1}{2}S_{N}\ti{K} \ra^{1/2}}
\end{align}
\begin{equation}
    \max{(\frac{\mathcal{S}}{\mathcal{N}})}=\la h|h \ra^{1/2}
\end{equation}

\section{parameter estimation}

\begin{equation}
    p(\mu|d)\propto p(\mu)\exp\left[-\frac{1}{2}\sum_{m,n}C_{mn}^{-1}(d_m-h_m)(d_n-h_n)\right],
\end{equation}
\begin{equation}
    p(\mu|d)\propto p(\mu)\exp\left[-\frac{1}{2}\la d-h|d-h \ra\right].
\end{equation}

\section{sensitivity}

\begin{equation}
    \Gamma_{mn}=\text{E}(\la d-h|\p_m h\ra\la d-h|\p_n h \ra)=\la \p_m h|\p_n h \ra.
\end{equation}

    \chapter{电磁引力}

\cite{Maartens2008}. 

\section{时空张量转化为空间张量}

\begin{equation}
    h_{ab}:=g_{ab}+Z_a Z_b.
\end{equation}
\begin{equation}
    h_{a}^{\ph{a}b}=\delta_{a}^{\ph{a}b}+Z_aZ^b.
\end{equation}
\begin{equation}
    Z^ah_{ab}=0.
\end{equation}

\begin{equation}
    V_{\langle a\rangle}:=h_a^{\ph{a}b}V_{b}.
\end{equation}
\begin{equation}
    Z^aV_{\langle a\rangle}=0.
\end{equation}

\begin{equation}
    T_{\langle ab\rangle}:=h_{(a}^{\ph{(a}c}h_{b)}^{\ph{b)}d}T_{cd}-\frac{1}{3}h_{cd}T^{cd}h_{ab}.
\end{equation}
\begin{equation}
    Z^a(h_{a}^{\ph{a}c}h_{b}^{\ph{b}d}T_{cd})=0.
\end{equation}
\begin{equation}
    Z^a(h_{b}^{\ph{b}c}h_{a}^{\ph{a}d}T_{cd})=0.
\end{equation}
\begin{equation}
    Z^a(h_{(a}^{\ph{(a}c}h_{b)}^{\ph{b)}d}T_{cd})=0.
\end{equation}
\begin{equation}
    Z^a(h_{cd}T^{cd}h_{ab})=0.
\end{equation}
\begin{equation}
    Z^aT_{\langle ab\rangle}=0.
\end{equation}
\begin{equation}
    T_{\left(\langle ab\rangle\right)}=T_{\langle ab\rangle}.
\end{equation}
\begin{equation}
    h^{ab}T_{\langle ab\rangle}=0.
\end{equation}

\begin{equation}
    \varepsilon_{abc}:=\varepsilon_{abcd}Z^d.
\end{equation}
\begin{equation}
    \varepsilon_{0123}:=-\sqrt{\left\lvert g\right\rvert }.
\end{equation}

\begin{equation}
    T_a:=\frac{1}{2}\varepsilon_{abc}T^{[bc]}.
\end{equation}
\begin{equation}
    [U,V]_a:=\varepsilon_{abc}U^bV^c.
\end{equation}
\begin{equation}
    [S,T]_a:=\varepsilon_{abc}g_{de}S^{bd}T^{ce}.
\end{equation}

\begin{equation}
    \text{D}_tT^{a\dots}_{\ph{a\dots}b\dots}:=Z^c\nabla_c T^{a\dots}_{\ph{a\dots}b\dots}.
\end{equation}
\begin{equation}
    {}^3\nabla_a T^{b\dots}_{\ph{b\dots}c\dots}:=h_a^{\ph{a}p}h^b_{\ph{b}q}\dots h_c^{\ph{c}r}\dots\nabla_p T^{q\dots}_{\ph{q\dots}r\dots}.
\end{equation}

\begin{equation}
    (\text{div}\,V):={}^3\nabla^aV_a.
\end{equation}
\begin{equation}
    (\text{curl}\,V)_a:=\varepsilon_{bca}{}^3\nabla^bV^c.
\end{equation}
\begin{equation}
    (\text{div}\,T)_a:={}^3\nabla^bT_{ab}.
\end{equation}
\begin{equation}
    (\text{curl}\,T)_{ab}:=
    \varepsilon_{cd(a}{}^3\nabla^cg_{b)e}T^{ed}.
\end{equation}

\section{电磁空间矢量}

\begin{equation}
    {}^*\!F_{ab}:=\frac{1}{2}\varepsilon_{abcd}F^{cd}
\end{equation}
\begin{equation}
    E_a:=F_{ab}Z^b=E_{\langle a\rangle}.
\end{equation}
\begin{equation}
    B_a:={}^*\!F_{ab}Z^b=B_{\langle a\rangle}.
\end{equation}

\begin{equation}
    \rho=-Z^aJ_a.
\end{equation}
\begin{equation}
    j_a=h_a^{\ph{a}b}J_{b}.
\end{equation}

\begin{equation}
    \nabla_{[a}F_{bc]}=0.
\end{equation}
\begin{equation}
    \nabla^{a}F_{ab}=\mu J_{b}.
\end{equation}

\begin{equation}
    (\text{div}\,E)=\mu\rho-\dots.
\end{equation}
\begin{equation}
    (\text{div}\,B)=+\dots.
\end{equation}
\begin{equation}
    (\text{curl}\,E)_a+\dots=-\text{D}_t B_{\langle a\rangle}-\dots.
\end{equation}
\begin{equation}
    (\text{curl}\,B)_a+\dots=\mu j_a+\text{D}_t E_{\langle a\rangle}+\dots.
\end{equation}

\section{引力空间张量}

\begin{equation}
    {}^*\!C_{abcd}:=\frac{1}{2}\varepsilon_{abef}C^{ef}_{\ph{ef}cd}.
\end{equation}
\begin{equation}
    E_{ab}:=C_{acbd}Z^cZ^d=E_{\langle ab\rangle}.
\end{equation}
\begin{equation}
    B_{ab}:={}^*\!C_{acbd}Z^cZ^d=B_{\langle ab\rangle}.
\end{equation}

\begin{equation}
    (\text{div}\,E)_a=\kappa\frac{1}{3}{}^3\nabla_a\rho-\dots.
\end{equation}
\begin{equation}
    (\text{div}\,B)_a=\kappa(\rho+p)\omega_a+\dots.
\end{equation}
\begin{equation}
    (\text{curl}\,E)_{ab}+\dots=-\text{D}_t B_{\langle ab\rangle}-\dots.
\end{equation}
\begin{equation}
    (\text{curl}\,B)_{ab}+\dots=\kappa\frac{1}{2}(\rho+p)\sigma_{ab}+\text{D}_t E_{\langle ab\rangle}+\dots.
\end{equation}

%\section{近似}

%\begin{equation}
%    A^a=Z^b\nabla_bZ^a.
%\end{equation}
%\begin{equation}
%    \Theta=\nabla_aZ^a
%\end{equation}
%\begin{equation}
%    \sigma_{ab}=h_{(a}^ch_{b)}^d\nabla_dZ_c-\frac{1}{3}h_{ab}\nabla_eZ^e.
%\end{equation}
%\begin{equation}
%    \omega_{ab}=h_{[a}^ch_{b]}^d\nabla_dZ_c
%\end{equation}

%\begin{equation}
%    (\text{div}\,E)=\mu\rho.
%\end{equation}
%\begin{equation}
%    (\text{div}\,B)=0.
%\end{equation}
%\begin{equation}
%    (\text{curl}\,E)_a=-\text{D}_t B_{\langle a\rangle}.
%\end{equation}
%\begin{equation}
%    (\text{curl}\,B)_a=\mu j_a+\text{D}_t E_{\langle a\rangle}.
%\end{equation}

%\begin{equation}
%    (\text{div}\,E)_a=\kappa\frac{1}{3}{}^3\nabla_a\rho.
%\end{equation}
%\begin{equation}
%    (\text{div}\,B)_a=\kappa(\rho+p)\omega_a.
%\end{equation}
%\begin{equation}
%    (\text{curl}\,E)_{ab}=-\text{D}_t B_{\langle ab\rangle}.
%\end{equation}
%\begin{equation}
%    (\text{curl}\,B)_{ab}=\kappa\frac{1}{2}(\rho+p)\sigma_{ab}+\text{D}_t E_%{\langle ab\rangle}.
%\end{equation}

%\begin{equation}
%    \vec{\nabla}\cdot\vec{E}=\mu_0\rho.
%\end{equation}
%\begin{equation}
%    \vec{\nabla}\cdot\vec{B}=0.
%\end{equation}
%\begin{equation}
%    \vec{\nabla}\times\vec{E}=-\frac{\p}{\p t}\vec{B}.
%\end{equation}
%\begin{equation}
%    \vec{\nabla}\times\vec{B}=\mu_0 \vec{j}+\frac{\p}{\p t}\vec{E}.
%\end{equation}
%
%\begin{equation}
%    \rho_\text{p}=-\vec{\nabla}\cdot\vec{P}.
%\end{equation}
%\begin{equation}
%    \vec{D}=\frac{1}{\mu_0}\vec{E}+\vec{P}=\epsilon\vec{E}.
%\end{equation}
%
%\begin{equation}
%    \vec{j}_\text{m}=\vec{\nabla}\times\vec{M}.
%\end{equation}
%\begin{equation}
%    \vec{j}_\text{p}=\frac{\p}{\p t}\vec{P}
%\end{equation}
%\begin{equation}
%    \vec{H}=\frac{1}{\mu_0}\vec{B}-\vec{M}=\frac{1}{\mu}\vec{B}
%\end{equation}
%
%\begin{equation}
%    \vec{\nabla}\cdot\vec{D}=\rho_f.
%\end{equation}
%\begin{equation}
%    \vec{\nabla}\cdot\vec{B}=0.
%\end{equation}
%\begin{equation}
%    \vec{\nabla}\times\vec{E}=-\frac{\p}{\p t}\vec{B}.
%\end{equation}
%\begin{equation}
%    \vec{\nabla}\times\vec{H}=\vec{j}_\text{f}+\frac{\p}{\p t}\vec{D}.
%\end{equation}

    \chapter{Varying $G$}

\section{Modification of Amplitude}

\begin{equation}
    \p^c\p_c\bar{h}_{ab}=-16\pi\frac{G_0}{c_0^4}T_{ab},\quad\p^a\bar{h}_{ab}=0
\end{equation}
\begin{equation}
    \Gamma^c_{\ph{c}ab}=\frac{1}{2}\eta^{cd}(2\p_{(a}{h}_{b)d}-\p_d{h}_{ab})
\end{equation}
\begin{equation}
    U^a\p_aU^c+\Gamma^c_{\ph{c}ab}U^aU^b=0
\end{equation}
\begin{equation}
    U^a\p_aU^c=-\frac{1}{2}\eta^{cd}(2\p_{(a}{h}_{b)d}-\p_d{h}_{ab})U^aU^b
\end{equation}

\begin{equation}
    T_{ab}=c_0^2(2U_{(a}J_{b)}+U^cJ_cU_aU_b)
\end{equation}
\begin{equation}
    J_bc_0^2=-U^aT_{ab}
\end{equation}

\begin{equation}
    A_b
    =-\frac{1}{4}U^a\bar{h}_{ab}
\end{equation}
\begin{equation}
    A_0
    =-\frac{1}{4}c_0\bar{h}_{00}
    =-\frac{1}{2}c_0(\bar{h}_{00}-\frac{1}{2}\eta_{00}\eta^{00}\bar{h}_{00})=-\frac{1}{2}c_0h_{00}
\end{equation}
\begin{equation}
    A_i=-\frac{1}{4}c_0\bar{h}_{0i}=-\frac{1}{4}c_0{h}_{0i}
\end{equation}

\begin{equation}
    U^\mu\p_\mu U^i=-\frac{1}{2}\eta^{i\sigma}(\p_\mu{h}_{\nu\sigma}+\p_\nu{h}_{\mu\sigma}-\p_\sigma{h}_{\mu\nu})U^\mu U^\nu
\end{equation}
\begin{align}
    -\frac{1}{2}\eta^{i\sigma}(\p_0{h}_{0\sigma}+\p_0{h}_{0\sigma}-\p_\sigma{h}_{00})U^0 U^0
    &=\frac{1}{2}c_0^2\eta^{i\sigma}\p_\sigma{h}_{00}\\
    &=\frac{1}{2}c_0^2\p^i{h}_{00}\\
    &=-c_0\p^iA_0\\
    &=-E^i
\end{align}
\begin{align}
    -\frac{1}{2}\eta^{i\sigma}(\p_0{h}_{j\sigma}+\p_j{h}_{0\sigma}-\p_\sigma{h}_{0j})U^0 U^j
    &=-\frac{1}{2}c_0\eta^{i\sigma}(\p_j{h}_{0\sigma}-\p_\sigma{h}_{0j})v^j\\
    &=-\frac{1}{2}c_0\eta^{ik}(\p_j{h}_{0k}-\p_k{h}_{0j})v^j\\
    &=2\eta^{ik}(\p_jA_k-\p_kA_j)v^j\\
    &=-2\eta^{ik}(\p_kA_j-\p_jA_k)v^j\\
    &=-2(\p^iA_j-\p_jA^i)v^j\\
    &=-2\varepsilon^{i}_{\ph{i}jk}v^jB^k
\end{align}
\begin{align}
    -\frac{1}{2}\eta^{i\sigma}(\p_j{h}_{k\sigma}+\p_k{h}_{j\sigma}-\p_\sigma{h}_{jk})U^j U^k=0
\end{align}
\begin{equation}
    a^i=-E^i-4\varepsilon^{i}_{\ph{i}jk}v^jB^k
\end{equation}

\begin{equation}
    \p^i(\frac{1}{4\pi G_0}E_i)=\rho
\end{equation}
\begin{equation}
    \p^iB_i=0
\end{equation}
\begin{equation}
    \varepsilon^{i}_{\ph{i}jk}\p^jE^k=-\p_tB^i
\end{equation}
\begin{equation}
    \varepsilon^{i}_{\ph{i}jk}\p^j(\frac{c_0^2}{4\pi G_0}B^k)=j^i+\p_t(\frac{1}{4\pi G_0}E^i)
\end{equation}

\begin{equation}
    \varepsilon_{\text{G}0}:=\frac{1}{4\pi G_0},\quad
    \mu_{\text{G}0}:=\frac{4\pi G_0}{c_0^2}
\end{equation}
\begin{equation}
    \begin{cases}
        \vec{\nabla}\cdot(\varepsilon_{\text{G}0}\vec{E})=\rho\\
        \vec{\nabla}\cdot\vec{B}=0\\
        \vec{\nabla}\times\vec{E}=-\frac{\p}{\p t}\vec{B}\\
        \vec{\nabla}\times(\mu_{\text{G}0}^{-1}\vec{B})=\vec{j}+\frac{\p}{\p t}(\varepsilon_{\text{G}0}\vec{E})
    \end{cases}
\end{equation}
\begin{equation}
    \vec{a}=-\vec{E}-4\vec{v}\times\vec{B}
\end{equation}

\begin{equation}
    \varepsilon_{\text{G}}=\frac{1}{4\pi G},\quad
    \mu_{\text{G}}=\frac{4\pi G}{c^2}
\end{equation}
\begin{equation}
    x^\mu=(ct,x,y,z)
\end{equation}
\begin{equation}
    \begin{cases}
        \vec{\nabla}\cdot(\varepsilon_{\text{G}}\vec{E})=\rho\\
        \vec{\nabla}\cdot\vec{B}=0\\
        \vec{\nabla}\times\vec{E}=-\frac{\p}{\p t}\vec{B}\\
        \vec{\nabla}\times(\mu_{\text{G}}^{-1}\vec{B})=\vec{j}+\frac{\p}{\p t}(\varepsilon_{\text{G}}\vec{E})
    \end{cases}
\end{equation}
\begin{equation}
    \vec{a}=-\vec{E}-4\vec{v}\times\vec{B}
\end{equation}
\begin{equation}
    A_\mu=-\frac{1}{4}c_{}\bar{h}_{0\mu}
\end{equation}

\begin{equation}
    \begin{cases}
        \vec{\nabla}\cdot\vec{E}=\varepsilon_{\text{G}}^{-1}\rho\\
        \vec{\nabla}\times\vec{B}=\mu_{\text{G}}\vec{j}+\varepsilon_{\text{G}}\mu_{\text{G}}\frac{\p}{\p t}\vec{E}
    \end{cases}
\end{equation}
\begin{equation}
    \frac{1}{c^2}\frac{\p}{\p t}\varphi+\vec{\nabla}\cdot\vec{A}=0
\end{equation}
\begin{equation}
    \begin{cases}
        -\frac{1}{c^2}\frac{\p^2}{\p t^2}\varphi+\vec{\nabla}^2\varphi=\varepsilon_{\text{G}}^{-1}\rho\\
        -\frac{1}{c^2}\frac{\p^2}{\p t^2}\vec{A}+\vec{\nabla}^2\vec{A}=\mu_{\text{G}}\vec{j}
    \end{cases}
\end{equation}
\begin{equation}
    \begin{cases}
        -\frac{1}{c^2}\frac{\p^2}{\p t^2}c^{-1}\varphi+\vec{\nabla}^2c^{-1}\varphi=\mu_{\text{G}}c\rho\\
        -\frac{1}{c^2}\frac{\p^2}{\p t^2}\vec{A}+\vec{\nabla}^2\vec{A}=\mu_{\text{G}}\vec{j}
    \end{cases}
\end{equation}

\begin{equation}
    \begin{cases}
        \vec{\nabla}\cdot(\varepsilon_{\text{G}}\vec{E})=0\\
        \vec{\nabla}\cdot(\mu_{\text{G}}\vec{H})=0\\
        \vec{\nabla}\times\vec{E}=-\frac{\p}{\p t}(\mu_{\text{G}}\vec{H})\\
        \vec{\nabla}\times\vec{H}=+\frac{\p}{\p t}(\varepsilon_{\text{G}}\vec{E})
    \end{cases}
\end{equation}
\begin{equation}
    {E}_r=0,\quad{H}_r=0
\end{equation}
\begin{equation}
    \begin{cases}
        \frac{\varepsilon_{\text{G}}}{r\sin\theta}\frac{\p}{\p\theta}(\sin\theta{E}_\theta)+\frac{\varepsilon_{\text{G}}}{r\sin\theta}\frac{\p}{\p\phi}({E}_\phi)=0\\
        \frac{\mu_{\text{G}}}{r\sin\theta}\frac{\p}{\p\theta}(\sin\theta{H}_\theta)+\frac{\mu_{\text{G}}}{r\sin\theta}\frac{\p}{\p\phi}({H}_\phi)=0\\
        \frac{1}{r\sin\theta}[\frac{\p}{\p\theta}(\sin\theta{E}_\phi)-\frac{\p}{\p\phi}({E}_\theta)]\vec{e}_r-\frac{1}{r}\frac{\p}{\p r}(r{E}_\phi)\vec{e}_\theta+\frac{1}{r}\frac{\p}{\p r}(r{E}_\theta)\vec{e}_\phi=-\mu_{\text{G}}\frac{\p}{\p t}({H}_\theta\vec{e}_\theta+{H}_\phi\vec{e}_\phi)\\
        \frac{1}{r\sin\theta}[\frac{\p}{\p\theta}(\sin\theta{H}_\phi)-\frac{\p}{\p\phi}({H}_\theta)]\vec{e}_r-\frac{1}{r}\frac{\p}{\p r}(r{H}_\phi)\vec{e}_\theta+\frac{1}{r}\frac{\p}{\p r}(r{H}_\theta)\vec{e}_\phi=+\varepsilon_{\text{G}}\frac{\p}{\p t}({E}_\theta\vec{e}_\theta+{E}_\phi\vec{e}_\phi)
    \end{cases}
\end{equation}
\begin{equation}
    \vec{E}={E}_\theta\vec{e}_\theta,\quad\vec{H}={H}_\phi\vec{e}_\phi
\end{equation}
\begin{equation}
    \begin{cases}
        \frac{\varepsilon_{\text{G}}}{r\sin\theta}\frac{\p}{\p\theta}(\sin\theta E_\theta)=0\\
        \frac{\mu_{\text{G}}}{r\sin\theta}\frac{\p}{\p\phi}({H}_\phi)=0\\
        -\frac{1}{r\sin\theta}\frac{\p}{\p\phi}({E}_\theta)\vec{e}_r+\frac{1}{r}\frac{\p}{\p r}(r{E}_\theta)\vec{e}_\phi=-\mu_{\text{G}}\frac{\p}{\p t}({H}_\phi)\vec{e}_\phi\\
        +\frac{1}{r\sin\theta}\frac{\p}{\p\theta}(\sin\theta{H}_\phi)\vec{e}_r-\frac{1}{r}\frac{\p}{\p r}(r{H}_\phi)\vec{e}_\theta=+\varepsilon_{\text{G}}\frac{\p}{\p t}({E}_\theta)\vec{e}_\theta\\
    \end{cases}
\end{equation}
\begin{equation}
    \begin{cases}
        \frac{\p}{\p r}(r{E}_\theta)+\mu_{\text{G}}\frac{\p}{\p t}(r{H}_\phi)=0\\
        \frac{\p}{\p r}(r{H}_\phi)+\varepsilon_{\text{G}}\frac{\p}{\p t}(r{E}_\theta)=0\\
    \end{cases}
\end{equation}
\begin{equation}
    \begin{cases}
        \mu_{\text{G}}\frac{\p}{\p r}\mu_{\text{G}}^{-1}\frac{\p}{\p r}(r{E}_\theta)-\varepsilon_{\text{G}}\mu_{\text{G}}\frac{\p}{\p t}\frac{\p}{\p t}(r{E}_\theta)=0\\
        \varepsilon_{\text{G}}\frac{\p}{\p r}\varepsilon_{\text{G}}^{-1}\frac{\p}{\p r}(r{H}_\phi)-\varepsilon_{\text{G}}\mu_{\text{G}}\frac{\p}{\p t}\frac{\p}{\p t}(r{H}_\phi)=0\\
    \end{cases}
\end{equation}
\begin{equation}
    \begin{cases}
        \mu_{\text{G}}\frac{\p}{\p r}\mu_{\text{G}}^{-1}\frac{\p}{\p r}(r{E}_\theta)-\frac{\p}{\p(ct)}\frac{\p}{\p(ct)}(r{E}_\theta)=0\\
        \varepsilon_{\text{G}}\frac{\p}{\p r}\varepsilon_{\text{G}}^{-1}\frac{\p}{\p r}(r{H}_\phi)-\frac{\p}{\p(ct)}\frac{\p}{\p(ct)}(r{H}_\phi)=0\\
    \end{cases}
\end{equation}
\begin{equation}
    \begin{cases}
        \frac{\p}{\p r}\frac{\p}{\p r}(r{E}_\theta)-\frac{\p}{\p r}(\ln\mu_{\text{G}})\frac{\p}{\p r}(r{E}_\theta)-\frac{\p}{\p(ct)}\frac{\p}{\p(ct)}(r{E}_\theta)=0\\
        \frac{\p}{\p r}\frac{\p}{\p r}(r{H}_\phi)-\frac{\p}{\p r}(\ln\varepsilon_{\text{G}})\frac{\p}{\p r}(r{H}_\phi)-\frac{\p}{\p(ct)}\frac{\p}{\p(ct)}(r{H}_\phi)=0\\
    \end{cases}
\end{equation}
\begin{equation}
    \frac{\p^2}{\p r^2}f(r,t)-p(r)\frac{\p}{\p r}f(r,t)-\frac{\p^2}{\p (ct)^2}f(r,t)=0
\end{equation}
\begin{equation}
    f(r,t)=f(r)e^{-ikct}
\end{equation}
\begin{equation}
    \frac{\d^2}{\d r^2}f(r)-p(r)\frac{\d}{\d r}f(r)+k^2f(r)=0
\end{equation}
\begin{equation}
    \frac{\d^2}{\d r^2}f(r)-p\frac{\d}{\d r}f(r)+k^2f(r)=0
\end{equation}
\begin{equation}
    f(r)=e^{(p/2)r}[C_+e^{i\sqrt{k^2-(p/2)^2}r}+C_-e^{-i\sqrt{k^2-(p/2)^2}r}]
\end{equation}
\begin{equation}
    f(r,t)=e^{(p/2)r}[C_+e^{i(+\sqrt{k^2-(p/2)^2}r-kct)}+C_-e^{i(-\sqrt{k^2-(p/2)^2}r-kct)}]
\end{equation}
\begin{equation}
    f(r,t)=e^{(p/2)r}[C_+e^{i(+\sqrt{(\omega/c)^2-(p/2)^2}r-\omega t)}+C_-e^{i(-\sqrt{(\omega/c)^2-(p/2)^2}r-\omega t)}]
\end{equation}
\begin{equation}
    f(r,t)=e^{\int(p/2)\d r}[C_+e^{i(+\int\sqrt{(\omega/c)^2-(p/2)^2}\d r-\omega t)}+C_-e^{i(-\int\sqrt{(\omega/c)^2-(p/2)^2}\d r-\omega t)}]
\end{equation}
\begin{equation}
    \begin{cases}
        r_2\left\lvert {E}_\theta\right\rvert_{r=r_2}=r_1\left\lvert {E}_\theta\right\rvert_{r=r_1}e^{\int_{r_1}^{r_2}\frac{1}{2}\frac{\p}{\p r}(\ln\mu_{\text{G}})\,\d r}\\
        r_2\left\lvert {H}_\phi\right\rvert_{r=r_2}=r_1\left\lvert {H}_\phi\right\rvert_{r=r_1}e^{\int_{r_1}^{r_2}\frac{1}{2}\frac{\p}{\p r}(\ln\varepsilon_{\text{G}})\,\d r}
    \end{cases}
\end{equation}
\begin{equation}
    \begin{cases}
        {E}_2=\sqrt{\frac{{\mu_{\text{G}}}_2}{{\mu_{\text{G}}}_1}}\frac{r_1}{r_2}{E}_1\\
        {H}_2=\sqrt{\frac{{\varepsilon_{\text{G}}}_2}{{\varepsilon_{\text{G}}}_1}}\frac{r_1}{r_2}{H}_1
    \end{cases}
\end{equation}
\begin{equation}
    \begin{cases}
        {E}_2/c_2=\sqrt{\frac{{\mu_{\text{G}}}_2}{{\mu_{\text{G}}}_1}}\frac{c_1}{c_2}\frac{r_1}{r_2}{E}_1/c_1\\
        {B}_2=\sqrt{\frac{{\mu_{\text{G}}}_2}{{\mu_{\text{G}}}_1}}\frac{c_1}{c_2}\frac{r_1}{r_2}{B}_1
    \end{cases}
\end{equation}
\begin{equation}
    \begin{cases}
        (\omega/c_2)c_2(\bar{h}_{00})_2=\sqrt{\frac{{\mu_{\text{G}}}_2}{{\mu_{\text{G}}}_1}}\frac{c_1}{c_2}\frac{r_1}{r_2}(\omega/c_1)c_1(\bar{h}_{00})_1\\
        (\omega/c_2)c_2(\bar{h}_{0i})_2=\sqrt{\frac{{\mu_{\text{G}}}_2}{{\mu_{\text{G}}}_1}}\frac{c_1}{c_2}\frac{r_1}{r_2}(\omega/c_1)c_1(\bar{h}_{0i})_1
    \end{cases}
\end{equation}
\begin{equation}
    h_2=\sqrt{\frac{c_1^4/G_1}{c_2^4/G_2}}\frac{r_1}{r_2}h_1
\end{equation}

双星系统引力辐射本为
\begin{equation}
    h=\frac{\mathcal{M}[\pi \mathcal{M}F(t)]^{2/3}}{r}Q(\theta,\phi,\psi,\iota)\cos[\int 2\pi F(t)\,\d t]
\end{equation}
设双星系统常量$c^*$, $G^*$, 一观者临近双星系统且与双星系统相对静止, 其与双星系统距离为$r$, 测得强度$h_r$, 频率$F_r$, 则\footnote{$\mathcal{M}$和$c^*$, $G^*$简并, 所以可以笼统地仍记作$\mathcal{M}$.}
\begin{equation}
    h_r=\frac{\mathcal{M}[\pi \mathcal{M}F_r(t)]^{2/3}}{r/c^*}Q(\theta,\phi,\psi,\iota)
\end{equation}

设地球观者与双星系统距离为$d$, 双星系统红移为$z$, 测得强度$h_d$, 频率$F_d=F_r/(1+z)$, 则
\begin{align}
    h_d&=\sqrt{\frac{{c^*}^4/G^*}{c^4/G}}\frac{r}{d}h_r\\
    &=\sqrt{\frac{{c^*}^4/G^*}{c^4/G}}\frac{\mathcal{M}[\pi \mathcal{M}F_r(t)]^{2/3}}{d/c^*}Q(\theta,\phi,\psi,\iota)
\end{align}
所以地球观者测得
\begin{equation}
    h=\sqrt{\frac{{c^*}^4/G^*}{c^4/G}}\frac{\mathcal{M}[\pi \mathcal{M}F_r(t)]^{2/3}}{d/c^*}Q(\theta,\phi,\psi,\iota)\cos[\int 2\pi \frac{F_r(t)}{1+z}\,\d t]
\end{equation}
记$F_\text{obs}(t)=F_r(t)/(1+z)$, $\mathcal{M}_\text{obs}=\mathcal{M}(1+z)$, 光度距离$d_\text{L}=d(1+z)$, 则
\begin{align}
    h&=\sqrt{\frac{{c^*}^4/G^*}{c^4/G}}\frac{\mathcal{M}[\pi \mathcal{M}F_r(t)]^{2/3}}{d(1+z)/c^*}Q(\theta,\phi,\psi,\iota)\cos[\int 2\pi F_\text{obs}(t)\,\d t]\\
    &=\sqrt{\frac{{c^*}^4/G^*}{c^4/G}}\frac{\mathcal{M}_\text{obs}[\pi \mathcal{M}_\text{obs}F_\text{obs}(t)]^{2/3}}{d_\text{L}/c^*}Q(\theta,\phi,\psi,\iota)\cos[\int 2\pi F_\text{obs}(t)\,\d t]\\
    &=\sqrt{\frac{{c^*}^6/G^*}{c^6/G}}\frac{\mathcal{M}_\text{obs}[\pi \mathcal{M}_\text{obs}F_\text{obs}(t)]^{2/3}}{d_\text{L}/c}Q(\theta,\phi,\psi,\iota)\cos[\int 2\pi F_\text{obs}(t)\,\d t]
\end{align}
用引力波测距测得$d_\text{L,G}$, 则
\begin{equation}
    d_\text{L,G}=d_\text{L}\sqrt{\frac{c^6/G}{{c^*}^6/G^*}}
\end{equation}

\cite{Poisson1995}
\begin{equation}
    h(t)=\frac{\mathcal{M}[\pi \mathcal{M}F(t)]^{2/3}}{\xi\,d_\text{L}}Q(\text{angles})\cos\Phi(t)
\end{equation}
\begin{equation}
    \tilde{h}(f)=\frac{\sqrt{30}}{48\pi^{2/3}}\frac{\mathcal{M}^{5/6}Q}{\xi\,d_\text{L}}f^{-7/6}e^{i[2\pi ft(f)-\Phi(f)-\frac{\pi}{4}]}
\end{equation}
问题转化为估计$\xi$
\begin{equation}
    p(\mu)\propto p^{(0)}(\mu)\exp[
        -\frac{1}{2}\Gamma_{ab}(\mu^a-\hat{\mu}^a)(\mu^b-\hat{\mu}^b)
    ]
\end{equation}
\begin{equation}
    p^{(0)}(\mu)\propto\exp[
        -\frac{1}{2}\Gamma^{(0)}_{ab}(\mu^a-\bar{\mu}^a)(\mu^b-\bar{\mu}^b)
    ]
\end{equation}
设待估参数为$\mu=(\ln\xi,\ln (d_\text{L}/{d_\text{L}}_0),\ln Q,\dots)$, $\dots$为其他参数(如$\mathcal{M}$), 则$\tilde{h}_{,\ln\xi}=\tilde{h}_{,\ln (d_\text{L}/{d_\text{L}}_0)}=-\tilde{h}_{,\ln Q}=-\tilde{h}$, $\tilde{h}$对其他参数求偏导皆为纯虚数, 则由$\Gamma_{ab}=\left\langle h_{,a}|h_{,b}\right\rangle $和$\text{SNR}:=\rho=\sqrt{\left\langle h|h\right\rangle}$得
\begin{equation}
    \Gamma_{ab}=\begin{bmatrix}
        \rho^2&\rho^2&-\rho^2&0&\dots\\
        \rho^2&\rho^2&-\rho^2&0&\dots\\
        -\rho^2&-\rho^2&\rho^2&0&\dots\\
        0&0&0&?&\ldots\\
        \vdots&\vdots&\vdots&\vdots&\ddots 
    \end{bmatrix}
\end{equation}
又设
\begin{equation}
    \Gamma^{(0)}_{ab}=\begin{bmatrix}
        0&0&0&0&\dots\\
        0&1/\sigma_{\ln d_\text{L}}^2&0&0&\dots\\
        0&0&1/\sigma_{\ln Q}^2&0&\dots\\
        0 &0&0&0&\ldots\\
        \vdots&\vdots&\vdots&\vdots&\ddots 
    \end{bmatrix}
\end{equation}
则由$\Sigma_{ab}=(\Gamma^{(0)}_{ab}+\Gamma_{ab})^{-1}$得
\begin{equation}
    \Sigma_{ab}=\begin{bmatrix}
        \begin{bmatrix}
            \rho^2&\rho^2&-\rho^2\\
            \rho^2&\rho^2+1/\sigma_{\ln (d_\text{L}/{d_\text{L}}_0)}^2&-\rho^2\\
            -\rho^2&-\rho^2&\rho^2+1/\sigma_{\ln Q}^2
        \end{bmatrix}^{-1}&0\\
        0&[?]^{-1}
    \end{bmatrix}
\end{equation}
而
\begin{align}
    &\begin{bmatrix}
        \rho^2&\rho^2&-\rho^2\\
        \rho^2&\rho^2+1/\sigma_{\ln (d_\text{L}/{d_\text{L}}_0)}^2&-\rho^2\\
        -\rho^2&-\rho^2&\rho^2+1/\sigma_{\ln Q}^2
    \end{bmatrix}^{-1}\\
    &=\begin{bmatrix}
        1/\rho^2+\sigma_{\ln (d_\text{L}/{d_\text{L}}_0)}^2+\sigma_{\ln Q}^2&-\sigma_{\ln (d_\text{L}/{d_\text{L}}_0)}^2&\sigma_{\ln Q}^2\\
        -\sigma_{\ln (d_\text{L}/{d_\text{L}}_0)}^2&\sigma_{\ln (d_\text{L}/{d_\text{L}}_0)}^2&0\\
        \sigma_{\ln Q}^2&0&\sigma_{\ln Q}^2
    \end{bmatrix}
\end{align}

\section{Modification of Phase}

\begin{equation}
    \frac{d^2}{d z^2}H(z)+2p(z)\frac{d}{d z}H(z)+\left[\omega^2+q(z)\right]H(z)=0.\label{dessoleq}
\end{equation}
\begin{equation}\label{HAphi}
    H=Ae^{i\Phi}.
\end{equation}
$k=\frac{d \Phi}{d z}$,
\begin{equation}\label{rpart}
    \frac{d^2 A}{d z^2}+2p\frac{d A}{d z}+\left[\omega^2\left(1-\frac{k^2}{\omega^2}\right)+q\right]A=0,
\end{equation}
\begin{equation}\label{ipart}
    2\frac{d A}{d z}k+A\frac{d k}{d z}+2pAk=0,
\end{equation}
\begin{equation}
    2\frac{1}{A}\frac{d A}{d z}+\frac{1}{k}\frac{d k}{d z}+2p=0,
\end{equation}
\begin{equation}\label{Apk}
    A\propto e^{-\int p\,dz}k^{-1/2}.
\end{equation}
$\Gamma=e^{\int p \,dz}$ and $K=(k/\omega)^{-1/2}$,
\begin{equation}\label{equK0}
    \frac{d^2 K}{d z^2}-\left(\frac{1}{\Gamma}\frac{d^2\Gamma}{d z^2}-q\right)K+\omega^2K(1-K^{-4})=0,
\end{equation}
$\Xi=\frac{1}{\Gamma}\frac{d^2\Gamma}{d z^2}-q$ and make $\omega=1$,
\begin{equation}\label{equK}
    \frac{d^2 K}{d z^2}+K[(1-\Xi)-K^{-4}]=0.
\end{equation}
$\Xi=\text{const}$,
\begin{equation}\label{K0}
    K=(1-\Xi)^{-1/4}=1+\frac{1}{4}\Xi+\frac{5}{32}\Xi^2+O(\Xi^3),
\end{equation}
\begin{equation}
    k=(1-\Xi)^{1/2}=1-\frac{1}{2}\Xi-\frac{1}{8}\Xi^2+O(\Xi^3),
\end{equation}
$\Xi\neq\text{const}$, $\Xi(z)=\kappa^2\tilde{\Xi}(\tilde{z})$, where $\tilde{z}=\kappa z$.
\begin{equation}\label{equKs}
    K^3\frac{d^2 K}{d \tilde{z}^2}\kappa^2-K^4\tilde{\Xi}(\tilde{z})\kappa^2+K^4-1=0.
\end{equation}
\begin{equation}\label{K}
    K=\sum_{n=0}^\infty K_n(\tilde{z})\kappa^{2n},
\end{equation}
\begin{gather}
    K_0^4-1=0,\\
    K_0^3K_0''-K_0^4\tilde{\Xi}+4K_0^3K_1=0,\\
    (K_0^3K_1''+3K_0^2K_1K_0'')-4K_0^3K_1\tilde{\Xi}+(4K_0^3K_2+6K_0^2K_1^2)=0.
\end{gather}
\begin{gather}
    K_0=1,\\
    K_1=\frac{1}{4}\tilde{\Xi},\\
    K_2=\frac{5}{32}\tilde{\Xi}^2-\frac{1}{16}\frac{d^2\tilde{\Xi}}{d\tilde{z}^2},
\end{gather}

\cite{Poisson1995}
\begin{equation}
    h(t)=\frac{\mathcal{M}[\pi \mathcal{M}F(t)]^{2/3}}{d_\text{L}}Q(\text{angles})\cos\Phi(t)
\end{equation}
\begin{equation}
    \tilde{h}(f)=\frac{\sqrt{30}}{48\pi^{2/3}}\frac{\mathcal{M}^{5/6}Q}{d_\text{L}}f^{-7/6}e^{i[2\pi ft(f)-\Phi(f)-\frac{\pi}{4}]}
\end{equation}
\begin{equation}
    \tilde{h}(f)=\int h(t)e^{2\pi i f t}\d t
\end{equation}
\begin{equation}
    h(t)=\int \tilde{h}(f)e^{-2\pi i f t}\d f
\end{equation}
\begin{equation}
    A=\frac{\sqrt{30}}{48\pi^{2/3}}\frac{\mathcal{M}^{5/6}Q}{d_\text{L}}f^{-7/6}\,\d f
\end{equation}
\begin{equation}
    A\propto\Gamma^{-1}K
\end{equation}
\begin{equation}
    K|_{z=d_\text{L}}=K|_{z=0}=1
\end{equation}
\begin{equation}
    \frac{A|_{z=d_\text{L}}}{A|_{z=d_0}}=e^{-\int_0^{d_\text{L}}p\,d z}
\end{equation}
\begin{equation}
    A=\frac{\sqrt{30}}{48\pi^{2/3}}\frac{\mathcal{M}^{5/6}Q}{d_\text{L}}f^{-7/6}e^{-\int_0^{d_\text{L}}p\,d z}\,\d f
\end{equation}
\begin{equation}
    A=\frac{\sqrt{30}}{48\pi^{2/3}}\frac{\mathcal{M}^{5/6}Q}{e^{\int_0^{d_\text{L}}p\,d z}d_\text{L}}f^{-7/6}\,\d f
\end{equation}
\begin{equation}
    A=\frac{\sqrt{30}}{48\pi^{2/3}}\frac{\mathcal{M}^{5/6}Q}{\xi d_\text{L}}f^{-7/6}\,\d f
\end{equation}
\begin{equation}
    A=\mathcal{A}f^{-7/6}\,\d f
\end{equation}
\begin{equation}
    k=\omega[1-\frac{1}{2}\frac{\Xi}{\omega^2}]
\end{equation}
\begin{equation}
    \psi=\int k\,d z=2\pi ft(f)-\Phi(f)-\frac{\pi}{4}
\end{equation}
\begin{equation}
    \psi=\int k\,d z=2\pi ft(f)-\Phi(f)-\frac{\pi}{4}-\int_0^{d_\text{L}} \frac{1}{2}\frac{\Xi}{\omega^2}\omega\,d z
\end{equation}
\begin{equation}
    \psi=\int k\,d z=2\pi ft(f)-\Phi(f)-\frac{\pi}{4}-\frac{1}{2}\frac{\int_0^{d_\text{L}}\Xi\,d z}{(2\pi f)^2}(2\pi f)
\end{equation}
\begin{equation}
    \psi=\int k\,d z=2\pi ft(f)-\Phi(f)-\frac{\pi}{4}-\Omega(2\pi f)^{-1}
\end{equation}
\begin{equation}
    \tilde{h}(f)=\mathcal{A} f^{-7/6}e^{i[2\pi ft(f)-\Phi(f)-\frac{\pi}{4}-\Omega(2\pi f)^{-1}]}
\end{equation}
\begin{equation}
    2\pi f\Delta t(f)-\Delta \Phi(f)=(2\pi f)^{-1}
\end{equation}
\begin{equation}
    \frac{\d \Delta \Phi/\d f}{\d \Delta t/\d f}=2\pi f
\end{equation}
\begin{equation}
    \Delta \Phi(f)=-2(2\pi f)^{-1}
\end{equation}
\begin{equation}
    \quad\Delta t(f)=-(2\pi f)^{-2}
\end{equation}
\begin{equation}
    \Phi_{\text{1PN}}(f)=-\frac{1}{16}\frac{5}{3}(\frac{743}{336}+\frac{11}{4}\eta)(\pi\mathcal{M}f)^{-5/3}(\pi Mf)^{2/3}
\end{equation}
\begin{equation}
    t_{\text{1PN}}(f)=-\frac{5}{256}\frac{4}{3}(\frac{743}{336}+\frac{11}{4}\eta)\mathcal{M}(\pi\mathcal{M}f)^{-8/3}(\pi Mf)^{2/3}
\end{equation}
\begin{equation}
    \Delta \Phi(f)=2\Omega(2\pi f)^{-1}
\end{equation}
\begin{equation}
    \quad\Delta t(f)=\Omega(2\pi f)^{-2}
\end{equation}
\begin{equation}
    \frac{1}{16}\frac{5}{3}(\frac{743}{336}+\frac{11}{4}\eta)\mathcal{M}^{-5/3}M^{2/3}-\Omega
\end{equation}
\begin{equation}
    h(z,t)=h(z)e^{-i\omega t}
\end{equation}
\begin{equation}
    h(z)=\Gamma^{-1}(z)K(z)(C_+e^{+i\omega\int K^{-2}(z)\,\d z}+C_-e^{-i\omega\int K^{-2}(z)\,\d z})
\end{equation}
\begin{equation}
    h(z,t)=\int_{-\infty}^{+\infty}\tilde{h}(z;f)e^{-i2\pi f t}\,\d f
\end{equation}
\begin{equation}
    \tilde{h}(z;f)=\Gamma^{-1}(z)K(z;f)[C_+(f)e^{+i2\pi f\int K^{-2}(z;f)\,\d z}+C_-(f)e^{-i2\pi f\int K^{-2}(z;f)\,\d z}]
\end{equation}
\begin{equation}
    \tilde{h}_0(z;f)=\Gamma^{-1}(0)K(0;f)[C_+(f)e^{+i2\pi f\int\,\d z}+C_-(f)e^{-i2\pi f\int\,\d z}]
\end{equation}
\begin{equation}
    \tilde{h}(z;f)=\frac{\Gamma^{-1}(z)K(z;f)}{\Gamma^{-1}(0)K(0;f)}[C_+(f)e^{+i2\pi f\int K^{-2}(z)\,\d z}+C_-(f)e^{-i2\pi f\int K^{-2}(z)\,\d z}]
\end{equation}
\begin{equation}
    \tilde{h}_0(z;f)=[C_+(f)e^{+i2\pi f\int\,\d z}+C_-(f)e^{-i2\pi f\int\,\d z}]
\end{equation}
\begin{equation}
    K(z)=1+\frac{1}{4\omega^2}\Xi(z)
\end{equation}
\begin{equation}
    \tilde{h}(z;f)=\frac{\Gamma^{-1}(z)}{\Gamma^{-1}(0)}[1+\frac{\Xi(z)-\Xi(0)}{4}(2\pi f)^{-2}][C_+(f)e^{+i2\pi fz}e^{-i2\pi f\int\frac{\Xi(z)}{2}(2\pi f)^{-2}\,\d z}+C_-(f)e^{-i2\pi fz}e^{+i2\pi f\int\frac{\Xi(z)}{2}(2\pi f)^{-2}\,\d z}]
\end{equation}
\begin{equation}
    \tilde{h}(z;f)=\gamma(z)[1+\xi(z)(2\pi f)^{-2}][C_+(f)e^{+i2\pi fz}e^{+i\Omega(z)(2\pi f)^{-1}}+C_-(f)e^{-i2\pi fz}e^{-i\Omega(z)(2\pi f)^{-1}}]
\end{equation}
\begin{equation}
    \tilde{h}_0(z;f)=[C_+(f)e^{+i2\pi fz}+C_-(f)e^{-i2\pi fz}]
\end{equation}
\begin{equation}
    \tilde{h}_0(z;f)=C_+(f)e^{+i2\pi fz}+C_-(f)e^{-i2\pi fz}
\end{equation}
\begin{equation}
    \p_z\tilde{h}_0(z;f)=C_+(f)(i2\pi f)e^{+i2\pi fz}-C_-(f)(i2\pi f)e^{-i2\pi fz}
\end{equation}
\begin{equation}
    C_+(f)e^{+i2\pi fz}=\frac{1}{2}[\tilde{h}_0(z;f)+\p_z\tilde{h}_0(z;f)(i2\pi f)^{-1}]
\end{equation}
\begin{equation}
    C_-(f)e^{-i2\pi fz}=\frac{1}{2}[\tilde{h}_0(z;f)-\p_z\tilde{h}_0(z;f)(i2\pi f)^{-1}]
\end{equation}
\begin{equation}
    \tilde{h}=\mathcal{A}f^{-7/6}[1+\xi(2\pi f)^{-2}]e^{i\varphi}
\end{equation}
\begin{equation}
    \tilde{h}_{,\xi f_0^{-2}}=(2\pi)^{-2}f_0^{2}\mathcal{A}f^{-19/6}e^{i\varphi}
\end{equation}
\begin{equation}
    \tilde{h}_{,\xi f_0^{-2}}=\frac{(2\pi f/f_0)^{-2}}{1+\xi(2\pi f)^{-2}}\tilde{h}
\end{equation}
\begin{equation}
    \sim\left[\frac{(2\pi)^{-2}}{1+\xi(2\pi f_0)^{-2}}\right]^2\text{SNR}^2
\end{equation}
\begin{equation}
    \sim\text{SNR}^2
\end{equation}

    \bibliographystyle{abbrv}
    \bibliography{GWAstro}
\end{document}
