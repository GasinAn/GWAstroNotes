% 编译方式: xelatex -> bibtex -> xelatex*2
\documentclass{ctexbook}
\usepackage{amsfonts}
\usepackage{amsmath}
\usepackage{amssymb}
\usepackage{hyperref}
\usepackage{syntonly}
%\syntaxonly
\pagestyle{plain}
\makeatletter
\newcommand{\starttoc}{
    \chapter*{\contentsname}
    \@starttoc{toc}
}
\makeatother
\renewcommand{\tableofcontents}{\twocolumn\starttoc\onecolumn}
\hypersetup{
	colorlinks,
	linkcolor=blue,
	filecolor=pink,
	urlcolor=cyan,
	citecolor=red,
}
\def\p{\partial}
\title{引力波天文学笔记}
\author{GasinAn}
\begin{document}
    \maketitle
    \thispagestyle{empty}

\begin{flushleft}
    Copyright \textcopyright{} 2024 by GasinAn

    \ 

    All rights reserved. No part of this book may be reproduced, in any form or by any means, without permission in writing from the publisher, except by a \LaTeX er.

    \ 

    The author and publisher of this book have used their best efforts in preparing this book. These efforts include the development, research, and testing of the theories, technologies and programs to determine their effectiveness. The author and publisher make no warranty of any kind, express or implied, with regard to these techniques or programs contained in this book. The author and publisher shall not be liable in any event of incidental or consequential damages in connection with, or arising out of, the furnishing, performance, or use of these techniques or programs.

    \ 

    Printed in China
\end{flushleft}

    \tableofcontents
    %---------------------------------------------------------------------------
    \chapter{引力波}
    \section{Linear Gravity}
    \cite{Wald1984}. 流形$\mathbb{R}^{4}$. 任意坐标系$\{x^{\mu}\}$, $g_{\mu\nu}=\eta_{\mu\nu}+h_{\mu\nu}=\eta_{\mu\nu}+\gamma_{\mu\nu}s+\text{O}(s^2)$, 得
    \begin{equation}
        R_{\mu\nu\lambda\sigma}=\p_\sigma\p_{[\mu}h_{\lambda]\nu}-\p_\nu\p_{[\mu}h_{\lambda]\sigma}+\text{O}(s^2).
    \end{equation}
    $\bar{h}_{\mu\nu}:=h_{\mu\nu}-\frac{1}{2}\eta_{\mu\nu}\eta^{\lambda\sigma}h_{\lambda\sigma}$.
    \begin{equation}
        -\frac{1}{2} \partial^{\lambda} \partial_{\lambda} \bar{h}_{\mu \nu}+\partial^{\lambda} \partial_{(\mu} \bar{h}_{\nu) \lambda}-\frac{1}{2} \eta_{\mu \nu} \partial^{\lambda} \partial^{\sigma} \bar{h}_{\lambda \sigma}+\text{O}(s^2)=8 \pi T_{\mu \nu}.
    \end{equation}
    存在$\{x^{\mu}\}$, 使得$\p^{\nu}\bar{h}_{\mu\nu}+\text{O}(s^2)=0$ (Lorentz gauge). 
    令$\{x^{\mu}\}$满足$\p^{\nu}\bar{h}_{\mu\nu}+\text{O}(s^2)=0$, 则
    \begin{equation}
        \p^{\lambda}\p_{\lambda}\bar{h}_{\mu\nu}+\text{O}(s^2)=-16\pi T_{\mu \nu}.
    \end{equation}
    略去$\text{O}(s^2)$条件: $h_{\mu\nu}$, $\p_\lambda h_{\mu\nu}$\dots{}小.
    %---------------------------------------------------------------------------
    \bibliographystyle{plain}
    \bibliography{GWAstroNotes}
\end{document}
