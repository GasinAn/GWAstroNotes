\chapter{Fisher矩阵法}

\def\P{\mathbf{P} }
\cite{Finn1992}, 论证见\href{https://github.com/GasinAn/GWAstroNotes/blob/main/Finn/FinnNotes.pdf}{FinnNotes}.

\section{判断观测数据中有无信号}

$\Omega=A_0\cup A_m$, $A_0$为事件``无信号'', $A_m$为事件``有信号'', 测量结果为$G_t(\omega)$, 噪声$N_t(\omega)$, 信号$M_t(\omega)$,
\begin{equation}
    G_t(\omega)=\begin{cases}
        N_t(\omega)&\omega\in A_0,\\
        N_t(\omega)+M_t(\omega)&\omega\in A_m,
    \end{cases}
\end{equation}
实测得$g_t$, $A_g:=\{\omega:G_t(\omega)=g_t\}$, $A_g$为事件为``测得$g_t$'', 求$\P(A_m|A_g)$. 另认为信号依赖于参数$\vec{\mu}$, $A_m=\cup A_{\vec{\mu}}$, $A_{\vec{\mu}}$为事件``有信号且参数为$\mu$'', $p(\vec{\mu}):=p(A_{\vec{\mu}}|A_m)$
\begin{equation}
    \P(A_m|A_g)=\frac{\Lambda}{\Lambda+\P(A_0)/\P(A_m)},
\end{equation}
\begin{equation}
    \Lambda:=\int\d\vec{\mu}\,\lambda(\vec{\mu}),
\end{equation}
\begin{equation}
    \lambda(\vec{\mu}):=p(\vec{\mu})\exp[2\left\langle g(t)|m_{\vec{\mu}}(t)\right\rangle -\left\langle m_{\vec{\mu}}(t),m_{\vec{\mu}}(t)\right\rangle]
\end{equation}
\begin{equation}
    \left\langle \xi(t),\zeta(t)\right\rangle:=\int\d f\,\frac{\tilde{\xi}(f)\tilde{\zeta}(f)^*}{S_n(\left\lvert f\right\rvert )},
\end{equation}
\begin{equation}
    \tilde{q}(f):=\int\d t\,q(t)\exp[2\pi ift].
\end{equation}

\section{认定有信号后参数估计(MLE)}

实测得$g_t$且认定有信号, 事件$A_g\cap A_m$, 求使$p({A_{\vec{\mu}}}|A_g\cap A_m)$最大的$\vec{\mu}$, 记作$\hat{\vec{\mu}}$.
\begin{equation}
    p({A_{\vec{\mu}}}|A_g)=\frac{\lambda(\vec{\mu})}{\Lambda+\P(A_0)/\P(A_m)},
\end{equation}
\begin{equation}
    p({A_{\vec{\mu}}}|A_g\cap A_m)=\frac{\lambda(\vec{\mu})}{\Lambda},
\end{equation}
\begin{equation}
    \frac{\p\ln p(\vec{\mu})}{\p \vec{\mu}}|_{\vec{\mu}=\hat{\vec{\mu}}}+2\left\langle \frac{\p m_{\vec{\mu}}}{\p \vec{\mu}}|_{\vec{\mu}=\hat{\vec{\mu}}}(t),g(t)-m_{\vec{\mu}}|_{\vec{\mu}=\hat{\vec{\mu}}}(t)\right\rangle=0.
\end{equation}

\section{灵敏度}

若由$g_t$求得MLE为$\hat{\vec{\mu}}$, 则记$g\Rightarrow \hat{\vec{\mu}}$, $A_{\hat{\vec{\mu}}}:=\cup_{g\Rightarrow \hat{\vec{\mu}}}A_g$, $A_{\hat{\vec{\mu}}}$为事件``测得MLE为$\hat{\vec{\mu}}$'', $A_{\tilde{\vec{\mu}}}$为事件``有信号且参数为$\tilde{\vec{\mu}}$'', 求$p(A_{\tilde{\vec{\mu}}}|A_{\hat{\vec{\mu}}})$. 高SNR, $\tilde{\vec{\mu}}:=\hat{\vec{\mu}}+\delta\vec{\mu}$,
\begin{equation}
    p(A_{\hat{\vec{\mu}}+\delta\vec{\mu}}|A_{\hat{\vec{\mu}}})=\frac{\exp[-\frac{1}{2}\sum\mathcal{C}_{ij}^{-1}(\delta\mu_i-\overline{\delta\mu_i})(\delta\mu_j-\overline{\delta\mu_j})]}{[(2\pi)^N\text{det}(\mathcal{C}_{ij})^{1/2}]},
\end{equation}
\begin{equation}
    \mathcal{C}_{ij}^{-1}=2\left\langle \frac{\p m_{\vec{\mu}}}{\p \mu_i}|_{\vec{\mu}=\hat{\vec{\mu}}}(t),\frac{\p m_{\vec{\mu}}}{\p \mu_j}|_{\vec{\mu}=\hat{\vec{\mu}}}(t)\right\rangle,
\end{equation}
\begin{equation}
    \overline{\delta\mu_i}=-\sum\mathcal{C}_{ij}\frac{\p\ln p(\vec{\mu})}{\p \mu_j}|_{\vec{\mu}=\hat{\vec{\mu}}}.
\end{equation}

\section{认定有信号后参数估计(分布)}

\cite{Poisson1995}, 
\begin{equation}
    p({A_{\vec{\mu}}}|A_g\cap A_m)\propto p^{(0)}(\vec{\mu})\exp[
        -\frac{1}{2}\left\langle 
            m_{\vec{\mu}}(t)-g(t)|m_{\vec{\mu}}(t)-g(t)
        \right\rangle 
    ],
\end{equation}
\begin{equation}
    \left\langle \xi(t)|\zeta(t)\right\rangle:=2\int_0^\infty\frac{\tilde{\xi}(f)^*\tilde{\zeta}(f)+\tilde{\xi}(f)\tilde{\zeta}(f)^*}{S_n(f)}\,\d f,
\end{equation}
\begin{equation}
    \tilde{q}(f):=\int_{-\infty}^{\infty}q(t)e^{2\pi ift}\,\d t,
\end{equation}
\begin{equation}
    \frac{\p}{\p\vec{\mu}}\left\langle 
        m_{\vec{\mu}}(t)-g(t)|m_{\vec{\mu}}(t)-g(t)
    \right\rangle =\left\langle \frac{\p }{\p \vec{\mu}}m_{\vec{\mu}}(t)|m_{\vec{\mu}}(t)-g(t)\right\rangle,
\end{equation}
$\mu^a$估计为$\hat{\mu}^a$, 高SNR,
\begin{equation}
    \left\langle m_{,a}(t;\mu^b)|g(t)-m(t;\mu^b)\right\rangle|_{\mu^b=\hat{\mu}^b}=0,
\end{equation}
\begin{equation}
    \Gamma_{ab}:=\left\langle m_{,a}(t)|m_{,b}(t)\right\rangle,
\end{equation}
\begin{equation}
    p({A_{\mu^a}}|A_g\cap A_m)\propto p^{(0)}({\mu^a})\exp[
        -\frac{1}{2}\Gamma_{ab}(\mu^a-\hat{\mu}^a)(\mu^b-\hat{\mu}^b)
    ],
\end{equation}
\begin{equation}
    p^{(0)}({\mu^a}):\propto\exp[
        -\frac{1}{2}\Gamma^{(0)}_{ab}(\mu^a-\bar{\mu}^a)(\mu^b-\bar{\mu}^b)
    ].
\end{equation}
