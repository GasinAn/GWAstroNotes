\chapter{能量}

\section{BMS代数}

\def\Ip{\mathcal{I}^+}
\def\L{\mathcal{L}}
$\Gamma^{ab}_{\ph{ab}cd}:=n^an^b\tilde{h}_{cd}$, 若$\Ip$上光滑切矢场$\hat{\xi}^a$满足$\L_{\hat{\xi}}\Gamma^{ab}_{\ph{ab}cd}=0$, 则称$\hat{\xi}^a$为$\Ip$上的无限小对称性, 若$M$上光滑矢量场${\xi}^a$可光滑延拓至$\Ip$, $\Omega^2\L_{{\xi}}g_{ab}$也可光滑延拓至$\Ip$且在$\Ip$上为$0$, 则称${\xi}^a$为$M$上的无限小渐进对称性, 由$\L_{[u,v]}=[\L_u,\L_v]$可知$\Ip$上全体无限小对称性的集合上可定义Lie括号, 称得到的Lie代数为BMS代数.

\section{SPI代数}
若$\tilde{M}$上切矢$\eta^a$为单位切矢的类空曲线满足$\lim_{\to i_0}\tilde{h}_{ab}(\tilde{A}^b+\Omega^{-1}\tilde{\nabla}^b\Omega)=0$和其他必要条件, 则称曲线为正规曲线. 若两正规曲线在$i_0$处$\eta^a$和$\tilde{A}^a$相等, 则称两正规曲线等价, 4维流形$S:=\{\text{正规曲线等价类}\}$称为$i_0$的吹胀, S上有两个特殊场$v^a$和$h_{ab}$, 若$S$上矢量场${\xi}^a$满足$\L_{{\xi}}v^a=0$且$\L_{{\xi}}h_{ab}=0$, 则称${\xi}^a$为$i_0$上的无限小对称性, 由$\L_{[u,v]}=[\L_u,\L_v]$可知$i_0$上全体无限小对称性的集合上可定义Lie括号, 称得到的Lie代数为SPI代数.
