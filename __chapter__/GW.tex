\chapter{引力波}

\section{Linearized Gravity}

\cite{Wald1984}. 流形$\mathbb{R}^{4}$. 任意坐标系$\{x^{\mu}\}$, $g_{\mu\nu}=\eta_{\mu\nu}+h_{\mu\nu}=\eta_{\mu\nu}+\gamma_{\mu\nu}s+\text{O}(s^2)$, 得
\begin{equation}
    R_{\mu\nu\lambda\sigma}=\p_\sigma\p_{[\mu}h_{\lambda]\nu}-\p_\nu\p_{[\mu}h_{\lambda]\sigma}+\text{O}(s^2).
\end{equation}
$\bar{h}_{\mu\nu}:=h_{\mu\nu}-\frac{1}{2}\eta_{\mu\nu}\eta^{\lambda\sigma}h_{\lambda\sigma}=h_{\mu\nu}-\frac{1}{2}\eta_{\mu\nu}h$.
\begin{equation}
    -\frac{1}{2} \partial^{\lambda} \partial_{\lambda} \bar{h}_{\mu \nu}+\partial^{\lambda} \partial_{(\mu} \bar{h}_{\nu) \lambda}-\frac{1}{2} \eta_{\mu \nu} \partial^{\lambda} \partial^{\sigma} \bar{h}_{\lambda \sigma}+\text{O}(s^2)=8 \pi T_{\mu \nu}.
\end{equation}
存在$\{x^{\mu}\}$, 使得$\p^{\nu}\bar{h}_{\mu\nu}+\text{O}(s^2)=0$ (Lorentz gauge). 
令$\{x^{\mu}\}$满足$\p^{\nu}\bar{h}_{\mu\nu}+\text{O}(s^2)=0$, 则
\begin{equation}
    \p^{\lambda}\p_{\lambda}\bar{h}_{\mu\nu}+\text{O}(s^2)=-16\pi T_{\mu \nu}.\label{lin_gravity}
\end{equation}
略去$\text{O}(s^2)$条件: $h_{\mu\nu}$, $\p_\lambda h_{\mu\nu}$\dots{}小.

\section{Radiation Gauge}

\cite{Wald1984}. 存在$\{x^{\mu}\}$, 使得``无源处'' $h+\text{O}(s^2)=0$ (TT gauge \cite{Wang2020})且$h_{0\mu}+\text{O}(s^2)=0$. 

\section{Quadrupole Approximation}

\cite{Wald1984}. 下略$\text{O}(s^2)$. 由\eqref{lin_gravity}得
\begin{equation}
    \bar{h}_{\mu\nu}(t,\vec{r}) = 4\int 
    \frac{T_{\mu\nu}(t-\left\lvert\vec{r}-\vec{r}'\right\rvert,\vec{r}')}{\left\lvert\vec{r}-\vec{r}'\right\rvert}\,\d V'.
\end{equation}
\begin{align}
    \hat{\bar{h}}_{\mu\nu}(\omega,\vec{r})&:=\frac{1}{\sqrt{2\pi}}
    \int\bar{h}_{\mu\nu}(t,\vec{r})e^{i\omega t}\d t\\
    &=4\int 
    \frac{\hat{T}_{\mu\nu}(\omega,\vec{r}')}{\left\lvert\vec{r}-\vec{r}'\right\rvert}e^{i\omega\left\lvert\vec{r}-\vec{r}'\right\rvert}\,\d V'.
\end{align}
由$\p^{\nu}\bar{h}_{\mu\nu}=0$,
\begin{equation}
    -i\omega\hat{\bar{h}}_{0\mu}=\sum_{i}\frac{\p\hat{\bar{h}}_{i\mu}}{\p x^{i}}.
\end{equation}
$\left\lvert \vec{r}\right\rvert \gg \left\lvert \vec{r}'\right\rvert$且$\omega\ll1/\left\lvert \vec{r}'\right\rvert$,
\begin{equation}
    \hat{\bar{h}}_{ij}(\omega,\vec{r}) = 4
    \frac{e^{i\omega\left\lvert\vec{r}\right\rvert}}{\left\lvert\vec{r}\right\rvert}
    \int \hat{T}_{ij}(\omega,\vec{r}')\,\d V'.
\end{equation}
\begin{align}
    \int \hat{T}_{ij}\,\d V'
    &=\int \sum_{k}(\hat{T}_{kj}\frac{\p x'^i}{\p x'^k})\,\d V'\\
    &= \sum_{k}\left[\int \frac{\p }{\p x'^k}(\hat{T}_{kj}x'^i)\,\d V'-\int \frac{\p \hat{T}_{kj}}{\p x'^k}x'^i\,\d V'\right]\\
    &= \sum_{k}\int \p'_k\,(\hat{T}_{kj}x'^i)\,\d V' - \sum_{k}\int \frac{\p \hat{T}_{kj}}{\p x'^k}x'^i\,\d V'\\
    &= \int \hat{T}_{kj}x'^i\,\d S' - \sum_{k}\int \frac{\p \hat{T}_{kj}}{\p x'^k}x'^i\,\d V'\\
    &= -\sum_{k}\int \frac{\p \hat{T}_{kj}}{\p x'^k}x'^i\,\d V'\\
    &= -\int (\sum_{k}\p'_k\hat{T}_{kj})x'^i\,\d V'\\
    &= -\int (\p_0\hat{T}_{0j})x'^i\,\d V'\\
    &= -i\omega\int \hat{T}_{0j}x'^i\,\d V'\\
    &= \int \hat{T}_{(ij)}\,\d V'\\
    &= -i\omega\int \hat{T}_{0(j}x'^{i)}\,\d V'\\
    &= -\frac{i\omega}{2}\int (\hat{T}_{0j}x'^{i}+\hat{T}_{0i}x'^{j})\,\d V',\\
\end{align}
\begin{align}
    -\frac{i\omega}{2}\int (\hat{T}_{0j}x'^{i}+\hat{T}_{0i}x'^{j})\,\d V'
    &= -\frac{i\omega}{2}\int \sum_{k}(\hat{T}_{0k}x'^{i}\frac{\p x'^{j}}{\p x'^{k}}+\hat{T}_{0k}\frac{\p x'^{i}}{\p x'^{k}}x'^{j})\,\d V'\\
    &= -\frac{i\omega}{2}\sum_{k}\left[\int \frac{\p }{\p x'^k}(\hat{T}_{0k}x'^ix'^j)\,\d V'-\int \frac{\p \hat{T}_{0k}}{\p x'^k}x'^ix'^j\,\d V'\right]\\
    &= -\frac{i\omega}{2}\sum_{k}\int \p'_k\,(\hat{T}_{0k}x'^ix'^j)\,\d V' +\frac{i\omega}{2}\sum_{k}\int \frac{\p \hat{T}_{0k}}{\p x'^k}x'^ix'^j\,\d V'\\
    &= -\frac{i\omega}{2}\sum_{k}\int \hat{T}_{0k}x'^ix'^j\,\d S' +\frac{i\omega}{2}\sum_{k}\int \frac{\p \hat{T}_{0k}}{\p x'^k}x'^ix'^j\,\d V'\\
    &= \frac{i\omega}{2}\sum_{k}\int \frac{\p \hat{T}_{0k}}{\p x'^k}x'^ix'^j\,\d V'\\
    &= \frac{i\omega}{2}\int (\sum_{k}\p'_k\hat{T}_{0k})x'^ix'^j\,\d V'\\
    &= \frac{i\omega}{2}\int (\p_0\hat{T}_{00})x'^ix'^j\,\d V'\\
    &= -\frac{\omega^2}{2}\int \hat{T}_{00}\,x'^ix'^j\,\d V'.
\end{align}
\begin{equation}
    q_{ij}(t):=\int{T}_{00}\,x'^ix'^j\,\d V',
\end{equation}
\begin{equation}
    \hat{\bar{h}}_{ij}(\omega,\vec{r}) = -2\omega^2\frac{e^{i\omega\left\lvert\vec{r}\right\rvert}}{\left\lvert\vec{r}\right\rvert}\hat{q}_{ij}(\omega),
\end{equation}
\begin{equation}
    {\bar{h}}_{ij}(t,\vec{r}) = \frac{2}{\left\lvert\vec{r}\right\rvert}
    \frac{\d^2}{\d t^2}{q}_{ij}(t-\left\lvert\vec{r}\right\rvert).\label{qf}
\end{equation}

\section{$+$ Mode and $\times$ Mode}

寻新标架$(e'^1)_a=(e^+)_a$, $(e'^2)_a=(e^\times)_a$, $(e'^3)_a=(e^r)_a$, ${\bar{h}}_{ij}(e^i)_a(e^j)_b={\bar{h}}'_{ij}(e'^i)_a(e'^j)_b$, 取$x$, $y$分量后去迹, $h_+=\frac{1}{2}({\bar{h}}'_{11}-{\bar{h}}'_{22})$, $h_\times={\bar{h}}'_{12}={\bar{h}}'_{21}$? \cite{Sathyaprakash2009}

\cite{Blanchet1997}, $\vec{n}:=\frac{\vec{r}}{\left\lvert\vec{r}\right\rvert}$,
\begin{equation}
    h_{ij}^{\text{TT}}=\frac{2}{\left\lvert\vec{r}\right\rvert}\mathcal{P}_{ijkm}\frac{\d^2}{\d t^2}{Q}^{km}(t-\left\lvert\vec{r}\right\rvert), \label{TT}
\end{equation}
\begin{equation}
    \mathcal{P}_{ijkm}:=
    \left(\delta_{ik} -\vec{n}_i\vec{n}_k\right)
    \left(\delta_{jm} -\vec{n}_j\vec{n}_m\right)
    -\frac{1}{2}
    \left(\delta_{ij} -\vec{n}_i\vec{n}_j\right)
    \left(\delta_{km} -\vec{n}_k\vec{n}_m\right),
\end{equation}
\begin{equation}
    Q^{km}(t):=\int{T}_{00}\,\left(x'^kx'^m-\frac{1}{3}\delta^{km}\sum_nx'^nx'^n\right)\,\d V'
\end{equation}

\section{电磁---引力对比}

\begin{equation}
    A_\mu(t,\vec{r})=\frac{\mu_0}{4\pi}\int\frac{J_\mu(t-\left\lvert \vec{r}-\vec{r}'\right\rvert ,\vec{r}')}{\left\lvert \vec{r}-\vec{r}'\right\rvert}\d V'
\end{equation}
\begin{equation}
    \bar{h}_{\mu\nu}(t,\vec{r})=4G\int\frac{T_{\mu\nu}(t-\left\lvert\vec{r}-\vec{r}'\right\rvert,\vec{r}')}{\left\lvert\vec{r}-\vec{r}'\right\rvert}\,\d V'
\end{equation}

\begin{equation}
    A_\mu(t,\vec{r})=\frac{1}{\sqrt{2\pi}}
    \int\hat{A}_{\mu}(\omega,\vec{r})e^{-i\omega t}\d t
\end{equation}
\begin{equation}
    \bar{h}_{\mu\nu}(t,\vec{r})=\frac{1}{\sqrt{2\pi}}
    \int\hat{\bar{h}}_{\mu\nu}(\omega,\vec{r})e^{-i\omega t}\d t
\end{equation}

\begin{equation}
    \hat{A}_{\mu}(\omega,\vec{r})=\frac{\mu_0}{4\pi}\int\frac{\hat{J}_{\mu}(\omega,\vec{r}')}{\left\lvert\vec{r}-\vec{r}'\right\rvert}e^{i\omega\left\lvert\vec{r}-\vec{r}'\right\rvert}\,\d V'
\end{equation}
\begin{equation}
    \hat{\bar{h}}_{\mu\nu}(\omega,\vec{r})=4G\int\frac{\hat{T}_{\mu\nu}(\omega,\vec{r}')}{\left\lvert\vec{r}-\vec{r}'\right\rvert}e^{i\omega\left\lvert\vec{r}-\vec{r}'\right\rvert}\,\d V'
\end{equation}

\begin{equation}
    \hat{A}_{\mu}(\omega,\vec{r})=\frac{\mu_0}{4\pi}\frac{e^{i\omega\left\lvert\vec{r}\right\rvert}}{\left\lvert\vec{r}\right\rvert}\int\hat{J}_{\mu}(\omega,\vec{r}')e^{-i\omega(\frac{\vec{r}}{\left\lvert\vec{r}\right\rvert}\cdot\vec{r}')}\,\d V'
\end{equation}
\begin{equation}
    \hat{\bar{h}}_{\mu\nu}(\omega,\vec{r})=4G\frac{e^{i\omega\left\lvert\vec{r}\right\rvert}}{\left\lvert\vec{r}\right\rvert}\int\hat{T}_{\mu\nu}(\omega,\vec{r}')e^{-i\omega(\frac{\vec{r}}{\left\lvert\vec{r}\right\rvert}\cdot\vec{r}')}\,\d V'
\end{equation}

\begin{equation}
    \hat{A}_{\mu}(\omega,\vec{r})=\frac{\mu_0}{4\pi}\frac{e^{i\omega\left\lvert\vec{r}\right\rvert}}{\left\lvert\vec{r}\right\rvert}\int\hat{J}_{\mu}(\omega,\vec{r}')\left[1-i\omega(\frac{\vec{r}}{\left\lvert\vec{r}\right\rvert}\cdot\vec{r}')-\dots\right]\,\d V'
\end{equation}
\begin{equation}
    \hat{\bar{h}}_{\mu\nu}(\omega,\vec{r})=4G\frac{e^{i\omega\left\lvert\vec{r}\right\rvert}}{\left\lvert\vec{r}\right\rvert}\int\hat{T}_{\mu\nu}(\omega,\vec{r}')\left[1-i\omega(\frac{\vec{r}}{\left\lvert\vec{r}\right\rvert}\cdot\vec{r}')-\dots\right]\,\d V'
\end{equation}

\subsection{电偶极---引力对比}

\begin{equation}
    \hat{A}_i=\frac{\mu_0}{4\pi}\frac{e^{i\omega\left\lvert\vec{r}\right\rvert}}{\left\lvert\vec{r}\right\rvert}\int\hat{J}_i\,\d V'
\end{equation}
\begin{equation}
    \hat{\bar{h}}_{ij}=4G\frac{e^{i\omega\left\lvert\vec{r}\right\rvert}}{\left\lvert\vec{r}\right\rvert}\int\hat{T}_{ij}\,\d V'
\end{equation}

\begin{equation}
    \int \hat{J}_{i}\,\d V'=-i\omega\int\hat{J}_{0}x'^i\,\d V'
\end{equation}
\begin{equation}
    \int\hat{T}_{ij}\,\d V'=-\frac{\omega^2}{2}\int \hat{T}_{00}\,x'^ix'^j\,\d V'
\end{equation}

\begin{equation}
    \hat{p}_i=\int\hat{J}_{0}x'^i\,\d V'
\end{equation}
\begin{equation}
    \hat{q}_{ij}=\int \hat{T}_{00}\,x'^ix'^j\,\d V'
\end{equation}

\begin{equation}
    \hat{A}_i=\frac{\mu_0}{4\pi}\frac{e^{i\omega\left\lvert\vec{r}\right\rvert}}{\left\lvert\vec{r}\right\rvert}(-i\omega \hat{p}_i)
\end{equation}
\begin{equation}
    \hat{\bar{h}}_{ij}=4G\frac{e^{i\omega\left\lvert\vec{r}\right\rvert}}{\left\lvert\vec{r}\right\rvert}(-\frac{\omega^2}{2}\hat{q}_{ij})
\end{equation}

\begin{equation}
    {A}_i=\frac{\mu_0}{4\pi}\frac{1}{\left\lvert\vec{r}\right\rvert}\frac{\d}{\d t}p_i(t-\left\lvert\vec{r}\right\rvert)
\end{equation}
\begin{equation}
    {\bar{h}}_{ij}=4G\frac{1}{\left\lvert\vec{r}\right\rvert}\frac{1}{2}\frac{\d^2}{\d t^2}q_{ij}(t-\left\lvert\vec{r}\right\rvert)
\end{equation}

\subsection{电四极---引力对比}

\begin{equation}
    \hat{A}_{i}(\omega,\vec{r})=\frac{\mu_0}{4\pi}\frac{e^{i\omega\left\lvert\vec{r}\right\rvert}}{\left\lvert\vec{r}\right\rvert}()(-i\omega)\int\hat{J}_{i}(\omega,\vec{r}')(\frac{\vec{r}}{\left\lvert\vec{r}\right\rvert}\cdot\vec{r}')\,\d V'
\end{equation}

\begin{equation}
    \hat{A}_i=\frac{\mu_0}{4\pi}\frac{e^{i\omega\left\lvert\vec{r}\right\rvert}}{\left\lvert\vec{r}\right\rvert}(-i\omega)\int\hat{J}_i'n^j{x}_j'\,\d V'
\end{equation}

\begin{equation}
    \hat{A}_i=\frac{\mu_0}{4\pi}\frac{e^{i\omega\left\lvert\vec{r}\right\rvert}}{\left\lvert\vec{r}\right\rvert}(-i\omega)\int n^j{x}_j'\hat{J}_i'\,\d V'
\end{equation}

\begin{equation}
    \hat{A}_i=\frac{\mu_0}{4\pi}\frac{e^{i\omega\left\lvert\vec{r}\right\rvert}}{\left\lvert\vec{r}\right\rvert}(-i\omega)n^j\!\left[\int {x}_{(j}'\hat{J}_{i)}'\,\d V'\right]
\end{equation}

\begin{align}
    \int {x}_{(j}'\hat{J}_{i)}'\,\d V'
    &= \frac{1}{2}\int (\hat{J}_{j}'{x}_{i}'+\hat{J}_{i}'{x}_{j}')\,\d V'\\
    &= \frac{1}{2}\int \sum_{k}(\hat{J}_{k}'x'^{i}\frac{\p x'^{j}}{\p x'^{k}}+\hat{J}_{k}'\frac{\p x'^{i}}{\p x'^{k}}x'^{j})\,\d V'\\
    &= \frac{1}{2}\sum_{k}\left[\int \frac{\p }{\p x'^k}(\hat{J}_{k}'x'^ix'^j)\,\d V'-\int \frac{\p \hat{J}_{k}'}{\p x'^k}x'^ix'^j\,\d V'\right]\\
    &= \frac{1}{2}\sum_{k}\int \p'_k\,(\hat{J}_{k}'x'^ix'^j)\,\d V' -\frac{1}{2}\sum_{k}\int \frac{\p \hat{J}_{k}'}{\p x'^k}x'^ix'^j\,\d V'\\
    &= \frac{1}{2}\sum_{k}\int \hat{J}_{k}'x'^ix'^j\,\d S' -\frac{1}{2}\sum_{k}\int \frac{\p \hat{J}_{k}'}{\p x'^k}x'^ix'^j\,\d V'\\
    &= -\frac{1}{2}\sum_{k}\int \frac{\p \hat{J}_{k}'}{\p x'^k}x'^ix'^j\,\d V'\\
    &= -\frac{1}{2}\int (\sum_{k}\p'_k\hat{J}_{k}')x'^ix'^j\,\d V'\\
    &= -\frac{1}{2}\int (\p_0\hat{J}_{0}')x'^ix'^j\,\d V'\\
    &= -\frac{i\omega}{2}\int \hat{J}_{0}'x'^ix'^j\,\d V'
\end{align}

\begin{equation}
    \hat{D}_{ij}=\int \hat{J}_{0}'\,x'^ix'^j\,\d V'
\end{equation}

\begin{equation}
    \hat{A}_i=\frac{\mu_0}{4\pi}\frac{e^{i\omega\left\lvert\vec{r}\right\rvert}}{\left\lvert\vec{r}\right\rvert}(-\frac{\omega^2}{2}n^j\hat{D}_{ij}
    )
\end{equation}

\begin{equation}
    {A}_i=\frac{\mu_0}{4\pi}\frac{1}{\left\lvert\vec{r}\right\rvert}n^j\frac{1}{2}\frac{\d^2}{\d t^2}D_{ij}(t-\left\lvert\vec{r}\right\rvert)
\end{equation}

\section{Varying $G$}

\begin{equation}
    T_{ab}=2U_{(a}J_{b)}+U^cJ_cU_aU_b
\end{equation}
\begin{equation}
    J_b=-U^aT_{ab}
\end{equation}
\begin{equation}
    \p^a\bar{h}_{ab}=0
\end{equation}
\begin{equation}
    \p^c\p_c\bar{h}_{ab}=-16\pi\frac{G_0}{c_0^3}T_{ab}
\end{equation}
\begin{equation}
    \Gamma^c_{\ph{c}ab}=\frac{1}{2}\eta^{cd}(2\p_{(a}{h}_{b)d}-\p_d{h}_{ab})
\end{equation}
\begin{equation}
    U^a\p_aU^c+\Gamma^c_{\ph{c}ab}U^aU^b=0
\end{equation}
\begin{equation}
    U^a\p_aU^c=-\frac{1}{2}\eta^{cd}(2\p_{(a}{h}_{b)d}-\p_d{h}_{ab})U^aU^b
\end{equation}

\begin{equation}
    A_b
    =-\frac{1}{4}U^a\bar{h}_{ab}
\end{equation}
\begin{equation}
    A_0
    =-\frac{1}{4}c_0\bar{h}_{00}
    =-\frac{1}{2}c_0(\bar{h}_{00}-\frac{1}{2}\eta_{00}\eta^{00}\bar{h}_{00})=-\frac{1}{2}c_0h_{00}
\end{equation}
\begin{equation}
    A_i=-\frac{1}{4}c_0\bar{h}_{0i}=-\frac{1}{4}c_0{h}_{0i}
\end{equation}

\begin{equation}
    U^\mu\p_\mu U^i=-\frac{1}{2}\eta^{i\sigma}(\p_\mu{h}_{\nu\sigma}+\p_\nu{h}_{\mu\sigma}-\p_\sigma{h}_{\mu\nu})U^\mu U^\nu
\end{equation}
\begin{align}
    -\frac{1}{2}\eta^{i\sigma}(\p_0{h}_{0\sigma}+\p_0{h}_{0\sigma}-\p_\sigma{h}_{00})U^0 U^0
    &=\frac{1}{2}c_0^2\eta^{i\sigma}\p_\sigma{h}_{00}\\
    &=\frac{1}{2}c_0^2\p^i{h}_{00}\\
    &=c_0\p^iA_0\\
    &=-E^i
\end{align}
\begin{align}
    -\frac{1}{2}\eta^{i\sigma}(\p_0{h}_{j\sigma}+\p_j{h}_{0\sigma}-\p_\sigma{h}_{0j})U^0 U^j
    &=-\frac{1}{2}c_0\eta^{i\sigma}(\p_j{h}_{0\sigma}-\p_\sigma{h}_{0j})v^j\\
    &=-\frac{1}{2}c_0\eta^{ik}(\p_j{h}_{0k}-\p_k{h}_{0j})v^j\\
    &=2\eta^{ik}(\p_jA_k-\p_kA_j)v^j\\
    &=2(\p_jA^i-\p^iA_j)v^j\\
    &=-2\varepsilon^{i}_{\ph{i}jk}v^jB^k
\end{align}
\begin{align}
    -\frac{1}{2}\eta^{i\sigma}(\p_j{h}_{k\sigma}+\p_k{h}_{j\sigma}-\p_\sigma{h}_{jk})U^j U^k=0
\end{align}
\begin{equation}
    a^i=-E^i-4\varepsilon^{i}_{\ph{i}jk}v^jB^k
\end{equation}

\begin{equation}
    \p^i(\frac{c_0}{16\pi G_0}E_i)=\rho
\end{equation}
\begin{equation}
    \p^iB_i=0
\end{equation}
\begin{equation}
    \varepsilon^{i}_{\ph{i}jk}\p^jE^k=-\p_tB^i
\end{equation}
\begin{equation}
    \varepsilon^{i}_{\ph{i}jk}\p^j(\frac{c_0^3}{16\pi G_0}B^k)=j^i+\p_t(\frac{c_0}{16\pi G_0}E^i)
\end{equation}

\begin{equation}
    \varepsilon_{\text{G}0}:=\frac{c_0}{16\pi G_0},\quad
    \mu_{\text{G}0}:=\frac{16\pi G_0}{c_0^3}
\end{equation}
\begin{equation}
    \begin{cases}
        \vec{\nabla}\cdot(\varepsilon_{\text{G}0}\vec{E})=\rho\\
        \vec{\nabla}\cdot\vec{B}=0\\
        \vec{\nabla}\times\vec{E}=-\frac{\p}{\p t}\vec{B}\\
        \vec{\nabla}\times(\mu_{\text{G}0}^{-1}\vec{B})=\vec{j}+\frac{\p}{\p t}(\varepsilon_{\text{G}0}\vec{E})
    \end{cases}
\end{equation}
\begin{equation}
    \vec{a}=-\vec{E}-4\vec{v}\times\vec{B}
\end{equation}

\begin{equation}
    \vec{a}=-\vec{E}-4\vec{v}\times\vec{B}
\end{equation}
\begin{equation}
    \begin{cases}
        \vec{\nabla}\cdot(\varepsilon_{\text{G}}\vec{E})=\rho\\
        \vec{\nabla}\cdot\vec{B}=0\\
        \vec{\nabla}\times\vec{E}=-\frac{\p}{\p t}\vec{B}\\
        \vec{\nabla}\times(\mu_{\text{G}}^{-1}\vec{B})=\vec{j}+\frac{\p}{\p t}(\varepsilon_{\text{G}}\vec{E})
    \end{cases}
\end{equation}

\begin{equation}
    \begin{cases}
        \vec{\nabla}\cdot(\varepsilon_{\text{G}}\vec{E})=0\\
        \vec{\nabla}\cdot(\mu_{\text{G}}\vec{H})=0\\
        \vec{\nabla}\times\vec{E}=-\frac{\p}{\p t}(\mu_{\text{G}}\vec{H})\\
        \vec{\nabla}\times\vec{H}=+\frac{\p}{\p t}(\varepsilon_{\text{G}}\vec{E})
    \end{cases}
\end{equation}
\begin{equation}
    {E}_r=0,\quad{H}_r=0
\end{equation}
\begin{equation}
    \begin{cases}
        \frac{\varepsilon_{\text{G}}}{r\sin\theta}\frac{\p}{\p\theta}(\sin\theta{E}_\theta)+\frac{\varepsilon_{\text{G}}}{r\sin\theta}\frac{\p}{\p\phi}({E}_\phi)=0\\
        \frac{\mu_{\text{G}}}{r\sin\theta}\frac{\p}{\p\theta}(\sin\theta{H}_\theta)+\frac{\mu_{\text{G}}}{r\sin\theta}\frac{\p}{\p\phi}({H}_\phi)=0\\
        \frac{1}{r\sin\theta}[\frac{\p}{\p\theta}(\sin\theta{E}_\phi)-\frac{\p}{\p\phi}({E}_\theta)]\vec{e}_r-\frac{1}{r}\frac{\p}{\p r}(r{E}_\phi)\vec{e}_\theta+\frac{1}{r}\frac{\p}{\p r}(r{E}_\theta)\vec{e}_\phi=-\mu_{\text{G}}\frac{\p}{\p t}({H}_\theta\vec{e}_\theta+{H}_\phi\vec{e}_\phi)\\
        \frac{1}{r\sin\theta}[\frac{\p}{\p\theta}(\sin\theta{H}_\phi)-\frac{\p}{\p\phi}({H}_\theta)]\vec{e}_r-\frac{1}{r}\frac{\p}{\p r}(r{H}_\phi)\vec{e}_\theta+\frac{1}{r}\frac{\p}{\p r}(r{H}_\theta)\vec{e}_\phi=+\varepsilon_{\text{G}}\frac{\p}{\p t}({E}_\theta\vec{e}_\theta+{E}_\phi\vec{e}_\phi)
    \end{cases}
\end{equation}
\begin{equation}
    \vec{E}={E}_\theta\vec{e}_\theta,\quad\vec{H}={H}_\phi\vec{e}_\phi
\end{equation}
\begin{equation}
    \begin{cases}
        \frac{\varepsilon_{\text{G}}}{r\sin\theta}\frac{\p}{\p\theta}(\sin\theta E_\theta)=0\\
        \frac{\mu_{\text{G}}}{r\sin\theta}\frac{\p}{\p\phi}({H}_\phi)=0\\
        -\frac{1}{r\sin\theta}\frac{\p}{\p\phi}({E}_\theta)\vec{e}_r+\frac{1}{r}\frac{\p}{\p r}(r{E}_\theta)\vec{e}_\phi=-\mu_{\text{G}}\frac{\p}{\p t}({H}_\phi)\vec{e}_\phi\\
        +\frac{1}{r\sin\theta}\frac{\p}{\p\theta}(\sin\theta{H}_\phi)\vec{e}_r-\frac{1}{r}\frac{\p}{\p r}(r{H}_\phi)\vec{e}_\theta=+\varepsilon_{\text{G}}\frac{\p}{\p t}({E}_\theta)\vec{e}_\theta\\
    \end{cases}
\end{equation}
\begin{equation}
    \begin{cases}
        \frac{\p}{\p r}(r{E}_\theta)+\mu_{\text{G}}\frac{\p}{\p t}(r{H}_\phi)=0\\
        \frac{\p}{\p r}(r{H}_\phi)+\varepsilon_{\text{G}}\frac{\p}{\p t}(r{E}_\theta)=0\\
    \end{cases}
\end{equation}
\begin{equation}
    \begin{cases}
        \mu_{\text{G}}\frac{\p}{\p r}\mu_{\text{G}}^{-1}\frac{\p}{\p r}(r{E}_\theta)-\varepsilon_{\text{G}}\mu_{\text{G}}\frac{\p}{\p t}\frac{\p}{\p t}(r{E}_\theta)=0\\
        \varepsilon_{\text{G}}\frac{\p}{\p r}\varepsilon_{\text{G}}^{-1}\frac{\p}{\p r}(r{H}_\phi)-\varepsilon_{\text{G}}\mu_{\text{G}}\frac{\p}{\p t}\frac{\p}{\p t}(r{H}_\phi)=0\\
    \end{cases}
\end{equation}
\begin{equation}
    \begin{cases}
        \mu_{\text{G}}\frac{\p}{\p r}\mu_{\text{G}}^{-1}\frac{\p}{\p r}(r{E}_\theta)-\frac{\p}{\p(v_\text{p}t)}\frac{\p}{\p(v_\text{p}t)}(r{E}_\theta)=0\\
        \varepsilon_{\text{G}}\frac{\p}{\p r}\varepsilon_{\text{G}}^{-1}\frac{\p}{\p r}(r{H}_\phi)-\frac{\p}{\p(v_\text{p}t)}\frac{\p}{\p(v_\text{p}t)}(r{H}_\phi)=0\\
    \end{cases}
\end{equation}
\begin{equation}
    \begin{cases}
        \frac{\p}{\p r}\frac{\p}{\p r}(r{E}_\theta)-\frac{\p}{\p r}(\ln\mu_{\text{G}})\frac{\p}{\p r}(r{E}_\theta)-\frac{\p}{\p(v_\text{p}t)}\frac{\p}{\p(v_\text{p}t)}(r{E}_\theta)=0\\
        \frac{\p}{\p r}\frac{\p}{\p r}(r{H}_\phi)-\frac{\p}{\p r}(\ln\varepsilon_{\text{G}})\frac{\p}{\p r}(r{H}_\phi)-\frac{\p}{\p(v_\text{p}t)}\frac{\p}{\p(v_\text{p}t)}(r{H}_\phi)=0\\
    \end{cases}
\end{equation}
\begin{equation}
    \frac{\p^2}{\p r^2}f(r,t)-p(r)\frac{\p}{\p r}f(r,t)-\frac{\p^2}{\p t^2}f(r,t)=0
\end{equation}
\begin{equation}
    \frac{\d^2}{\d r^2}f(r)-p(r)\frac{\d}{\d r}f(r)+f(r)=0
\end{equation}
\begin{equation}
    \frac{\d^2}{\d r^2}f(r)-p\frac{\d}{\d r}f(r)+f(r)=0
\end{equation}
\begin{equation}
    f(r)=e^{(p/2)r}[C_+e^{i\sqrt{1-(p/2)^2}r}+C_-e^{-i\sqrt{1-(p/2)^2}r}]
\end{equation}
\begin{equation}
    f(r)=e^{\int(p/2)\,\d r}[C_+e^{i\int\sqrt{1-(p/2)^2}\,\d r}+C_-e^{-i\int\sqrt{1-(p/2)^2}\,\d r}]
\end{equation}
\begin{equation}
    \begin{cases}
        r_2\left\lvert {E}_\theta\right\rvert_{r=r_2}=r_1\left\lvert {E}_\theta\right\rvert_{r=r_1}e^{\int_{r_1}^{r_2}\frac{1}{2}\frac{\p}{\p r}(\ln\mu_{\text{G}})\,\d r}\\
        r_2\left\lvert {H}_\phi\right\rvert_{r=r_2}=r_1\left\lvert {H}_\phi\right\rvert_{r=r_1}e^{\int_{r_1}^{r_2}\frac{1}{2}\frac{\p}{\p r}(\ln\varepsilon_{\text{G}})\,\d r}
    \end{cases}
\end{equation}
\begin{equation}
    \begin{cases}
        r_2\left\lvert {E}_\theta\right\rvert_{r=r_2}=r_1\left\lvert {E}_\theta\right\rvert_{r=r_1}\sqrt{\frac{\left\lvert \mu_{\text{G}}\right\rvert_{r=r_2}}{\left\lvert \mu_{\text{G}}\right\rvert_{r=r_1}}}\\
        r_2\left\lvert {H}_\phi\right\rvert_{r=r_2}=r_1\left\lvert {H}_\phi\right\rvert_{r=r_1}\sqrt{\frac{\left\lvert \varepsilon_{\text{G}}\right\rvert_{r=r_2}}{\left\lvert \varepsilon_{\text{G}}\right\rvert_{r=r_1}}}
    \end{cases}
\end{equation}
\begin{equation}
    L|_{r=r_2}=L|_{r=r_1}\frac{v_\text{p}|_{r=r_1}}{v_\text{p}|_{r=r_2}}
\end{equation}
\begin{equation}
    \begin{cases}
        \Delta t_{{E}_\theta}|_{r=r_1}^{r=r_2}=\int_{r_1}^{r_2}\frac{\sqrt{1-[\frac{1}{2}\frac{\p}{\p r}(\ln\mu_{\text{G}})]^2}}{v_\text{p}}\d r\\
        \Delta t_{{H}_\phi}|_{r=r_1}^{r=r_2}=\int_{r_1}^{r_2}\frac{\sqrt{1-[\frac{1}{2}\frac{\p}{\p r}(\ln\varepsilon_{\text{G}})]^2}}{v_\text{p}}\d r\\
    \end{cases}
\end{equation}

\begin{equation}
    S_\text{G}\propto\dot{h}^2\propto\omega^2 h^2
\end{equation}
\begin{equation}
    S_\text{G}\propto\frac{c^3}{G}\omega^2h^2
\end{equation}

双星系统引力辐射本为
\begin{equation}
    h=\frac{\mathcal{M}[\pi \mathcal{M}F(t)]^{2/3}}{r}Q(\theta,\phi,\psi,\iota)\cos[\int 2\pi F(t)\,\d t]
\end{equation}
设双星系统常量$c^*$, $G^*$, 一观者临近双星系统且与双星系统相对静止, 其与双星系统距离为$r$, 测得强度$h_r$, 频率$F_r$, 则\footnote{$\mathcal{M}$和$c^*$, $G^*$简并, 所以可以笼统地仍记作$\mathcal{M}$.}
\begin{equation}
    h_r=\frac{\mathcal{M}[\pi \mathcal{M}F_r(t)]^{2/3}}{r/c^*}Q(\theta,\phi,\psi,\iota)
\end{equation}
与双星系统距离为$r$的观者测得的引力辐射光度$L_r\propto4\pi r^2({c^*}^3/G^*)F_r^2 h_r^2$, 设地球观者与双星系统距离为$d$, 双星系统红移为$z$, 测得强度$h_d$, 频率$F_d$, 则地球观者测得的引力辐射光度正比于$L_d\propto4\pi d^2(c^3/G)F_d^2 h_d^2$, 且有$L_d=(c^*/c)L_r/(1+z)^2$, 所以$r^2({c^*}^3/G^*)F_r^2 h_r^2/(1+z)^2=(c^*/c)d^2(c^3/G)F_d^2 h_d^2$,又有$F_d=F_r/(1+z)$, 所以$r^2({c^*}^3/G^*)h_r^2=(c^*/c)d^2(c^3/G)h_d^2$, 则
\begin{align}
    h_d&=\sqrt{\frac{{c^*}^2/G^*}{c^2/G}}\frac{r^2}{d^2}h_r\\
    &=\sqrt{\frac{{c^*}^2/G^*}{c^2/G}}\frac{\mathcal{M}[\pi \mathcal{M}F_r(t)]^{2/3}}{d/c^*}Q(\theta,\phi,\psi,\iota)
\end{align}
所以地球观者测得
\begin{equation}
    h=\sqrt{\frac{{c^*}^2/G^*}{c^2/G}}\frac{\mathcal{M}[\pi \mathcal{M}F_r(t)]^{2/3}}{d/c^*}Q(\theta,\phi,\psi,\iota)\cos[\int 2\pi \frac{F_r(t)}{1+z}\,\d t]
\end{equation}
记$F_\text{obs}(t)=F_r(t)/(1+z)$, $\mathcal{M}_\text{obs}=\mathcal{M}(1+z)$, 光度距离$d_\text{L}=d(1+z)$, 则
\begin{align}
    h&=\sqrt{\frac{{c^*}^2/G^*}{c^2/G}}\frac{\mathcal{M}[\pi \mathcal{M}F_r(t)]^{2/3}}{d(1+z)/c^*}Q(\theta,\phi,\psi,\iota)\cos[\int 2\pi F_\text{obs}(t)\,\d t]\\
    &=\sqrt{\frac{{c^*}^2/G^*}{c^2/G}}\frac{\mathcal{M}_\text{obs}[\pi \mathcal{M}_\text{obs}F_\text{obs}(t)]^{2/3}}{d_\text{L}/c^*}Q(\theta,\phi,\psi,\iota)\cos[\int 2\pi F_\text{obs}(t)\,\d t]\\
    &=\sqrt{\frac{{c^*}^4/G^*}{c^4/G}}\frac{\mathcal{M}_\text{obs}[\pi \mathcal{M}_\text{obs}F_\text{obs}(t)]^{2/3}}{d_\text{L}/c}Q(\theta,\phi,\psi,\iota)\cos[\int 2\pi F_\text{obs}(t)\,\d t]
\end{align}
用引力波测距测得$d_\text{L,G}$, 则
\begin{equation}
    d_\text{L,G}=d_\text{L}\sqrt{\frac{c^4/G}{{c^*}^4/G^*}}
\end{equation}

\cite{Poisson1995}
\begin{equation}
    h(t)=\frac{\mathcal{M}[\pi \mathcal{M}F(t)]^{2/3}}{\xi\,d_\text{L}}Q(\text{angles})\cos\Phi(t)
\end{equation}
\begin{equation}
    \tilde{h}(f)=\frac{\sqrt{30}}{48\pi^{2/3}}\frac{\mathcal{M}^{5/6}Q}{\xi\,d_\text{L}}f^{-7/6}e^{i[2\pi ft(f)-\Phi(f)-\frac{\pi}{4}]}
\end{equation}
问题转化为估计$\xi$
\begin{equation}
    p(\mu)\propto p^{(0)}(\mu)\exp[
        -\frac{1}{2}\Gamma_{ab}(\mu^a-\hat{\mu}^a)(\mu^b-\hat{\mu}^b)
    ]
\end{equation}
\begin{equation}
    p^{(0)}(\mu)\propto\exp[
        -\frac{1}{2}\Gamma^{(0)}_{ab}(\mu^a-\bar{\mu}^a)(\mu^b-\bar{\mu}^b)
    ]
\end{equation}
设待估参数为$\mu=(\ln\xi,\ln (d_\text{L}/{d_\text{L}}_0),\ln Q,\dots)$, $\dots$为其他参数(如$\mathcal{M}$), 则$\tilde{h}_{,\ln\xi}=\tilde{h}_{,\ln (d_\text{L}/{d_\text{L}}_0)}=-\tilde{h}_{,\ln Q}=-\tilde{h}$, $\tilde{h}$对其他参数求偏导皆为纯虚数, 则由$\Gamma_{ab}=\left\langle h_{,a}|h_{,b}\right\rangle $和$\text{SNR}:=\rho=\sqrt{\left\langle h|h\right\rangle}$得
\begin{equation}
    \Gamma_{ab}=\begin{bmatrix}
        \rho^2&\rho^2&-\rho^2&0&\dots\\
        \rho^2&\rho^2&-\rho^2&0&\dots\\
        -\rho^2&-\rho^2&\rho^2&0&\dots\\
        0&0&0&?&\ldots\\
        \vdots&\vdots&\vdots&\vdots&\ddots 
    \end{bmatrix}
\end{equation}
又设
\begin{equation}
    \Gamma^{(0)}_{ab}=\begin{bmatrix}
        0&0&0&0&\dots\\
        0&1/\sigma_{\ln d_\text{L}}^2&0&0&\dots\\
        0&0&1/\sigma_{\ln Q}^2&0&\dots\\
        0 &0&0&0&\ldots\\
        \vdots&\vdots&\vdots&\vdots&\ddots 
    \end{bmatrix}
\end{equation}
则由$\Sigma_{ab}=(\Gamma^{(0)}_{ab}+\Gamma_{ab})^{-1}$得
\begin{equation}
    \Sigma_{ab}=\begin{bmatrix}
        \begin{bmatrix}
            \rho^2&\rho^2&-\rho^2\\
            \rho^2&\rho^2+1/\sigma_{\ln (d_\text{L}/{d_\text{L}}_0)}^2&-\rho^2\\
            -\rho^2&-\rho^2&\rho^2+1/\sigma_{\ln Q}^2
        \end{bmatrix}^{-1}&0\\
        0&[?]^{-1}
    \end{bmatrix}
\end{equation}
而
\begin{align}
    &\begin{bmatrix}
        \rho^2&\rho^2&-\rho^2\\
        \rho^2&\rho^2+1/\sigma_{\ln (d_\text{L}/{d_\text{L}}_0)}^2&-\rho^2\\
        -\rho^2&-\rho^2&\rho^2+1/\sigma_{\ln Q}^2
    \end{bmatrix}^{-1}\\
    &=\begin{bmatrix}
        1/\rho^2+\sigma_{\ln (d_\text{L}/{d_\text{L}}_0)}^2+\sigma_{\ln Q}^2&-\sigma_{\ln (d_\text{L}/{d_\text{L}}_0)}^2&\sigma_{\ln Q}^2\\
        -\sigma_{\ln (d_\text{L}/{d_\text{L}}_0)}^2&\sigma_{\ln (d_\text{L}/{d_\text{L}}_0)}^2&0\\
        \sigma_{\ln Q}^2&0&\sigma_{\ln Q}^2
    \end{bmatrix}
\end{align}
