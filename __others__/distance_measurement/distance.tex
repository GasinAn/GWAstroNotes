\def\documentclassname{ctexart}
\documentclass[a4paper,12pt]{\documentclassname}
\usepackage{amsmath}
\usepackage{amssymb}
\usepackage{amsthm}
\usepackage{color}
\usepackage{graphicx}
\usepackage{geometry}
\usepackage{hyperref}
\usepackage{ifthen}
\usepackage{listings}
\usepackage{marginnote}
\usepackage{mathrsfs}
\usepackage{syntonly}
\usepackage{textcomp}
\usepackage{ulem}
\usepackage{verbatim}
\usepackage{xcolor}
%\syntaxonly
%\geometry{a5paper}
\pagestyle{empty}
\hyphenation{}
\normalem
\hypersetup{
    colorlinks,
    linkcolor=blue,
    filecolor=pink,
    urlcolor=cyan,
    citecolor=red,
}
\lstset{
    language=[18]Fortran,
    basicstyle=\ttfamily,
    keywordstyle=\bfseries\color{green},
    identifierstyle=\color{blue},
    stringstyle=\color{black},
    commentstyle=\itshape\color{cyan},
    showspaces=false,
    showstringspaces=false,
}
\def\b{\boldsymbol}
\def\d{\mathrm{d}}
\def\p{\partial}
\def\ph{\phantom}
\def\t{\text}
\def\ti{\tilde}
\def\v{\vec}
\def\La{\Leftarrow}
\def\Ra{\Rightarrow}
\newcommand{\tabincell}[2]{\begin{tabular}{@{}#1@{}}#2\end{tabular}}
\DeclareMathOperator{\sgn}{sgn}
\DeclareMathOperator{\atanxy}{atan2}
\DeclareMathOperator{\Arg}{Arg}
\theoremstyle{definition}
\ifthenelse{\equal{\documentclassname}{article}}{
    \newtheorem{definition}{Definition}
    \newtheorem{theorem}{Theorem}
}{}
\ifthenelse{\equal{\documentclassname}{ctexart}}{
    \newtheorem{definition}{定义}
    \newtheorem{theorem}{定理}
}{}
\title{}
\author{}
\date{}
\begin{document}
    \section{光度距离}
    设发射时刻为$t_1$, 接收时刻为$t_2$, 则光度距离$L_\text{l}=(L/4\pi l)^{1/2}=(\frac{\d E_1}{\d t_1}/4\pi (\frac{\d E_2}{\d t_2}/4\pi D^2))^{1/2}=D((\d E_1/\d E_2)(\d t_2/\d t_1))^{1/2}=D((\hbar\omega_1/\hbar\omega_2)(\omega_1/\omega_2))^{1/2}=D(\omega_1/\omega_2)=D(\lambda_2/\lambda_1)=D(1+z)$.
    \section{角直径距离}
    设发射时刻为$t_1$, 接收时刻为$t_2$, source坐标为$(-x,0,0)$, source边缘坐标为$(-x,\pm\Delta y,0)$, detector坐标为$(0,0,0)$, 则source边缘辐射的世界线空间投影满足$\d x: \d y=x: \pm\Delta y$, 测得角直径$\theta=2\Delta y/x$, 固有线直径为$\Delta=2\Delta y$, 角直径距离$D_\text{a}=\Delta/\theta=x$, 物理距离$D=(a(t_2)/a(t_1))x=(1+z)x$, $D_\text{a}=D/(1+z)$.
    \section{自行距离}
    设发射时刻为$t_1$, 接收时刻为$t_2$, 则自行距离$D_\text{p}=v/\mu=((v\d t_1)/(\mu\d t_2))(\d t_2/\d t_1)=(\Delta/\theta)(\d t_2/\d t_1)=D_\text{a}(\d t_2/\d t_1)=D_\text{a}(\omega_1/\omega_2)=D_\text{a}(\lambda_2/\lambda_1)=D_\text{a}(1+z)=D$.
\end{document}
