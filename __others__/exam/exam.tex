\documentclass[12pt]{ctexart}
\usepackage{amsmath}
\usepackage{amssymb}
\usepackage{amsthm}
\usepackage{color}
\usepackage{graphicx}
\usepackage{geometry}
\usepackage{hyperref}
\usepackage{marginnote}
\usepackage{mathrsfs}
\usepackage{syntonly}
\usepackage{textcomp}
\usepackage{ulem}
\usepackage{verbatim}
%\syntaxonly
%\geometry{a5paper}
\hyphenation{}
\normalem
\hypersetup{
    colorlinks,
    linkcolor=blue,
    filecolor=pink,
    urlcolor=cyan,
    citecolor=red,
}
\title{}
\author{}
\begin{document}
甚高频1081012hz

暴涨精质化quintessential暴胀1091010hz10-3010-31随机背景\\doi:10.1103/PhysRevD.60.123511

https:||www.tgalimberti.com|BSc\_Research.pdf,ddh 2Hdh k2h=0,

10.1103/PhysRevD.68.044017互相交换能量.1010-1012hz10-2710-29连续谱



李芳昱109-1014hz效应doi: 10.7498/aps.41.1919电磁同步谐振系统方案10-30\\仪器doi.org/10.4236/jmp.2011.26060

意大利双球形腔耦合方案,仪器108hz10-19,10.1088/0264-9381/20/15/316\\效应 
10.1088/0305-4470/11/10/013 电磁减震模

\end{document}
