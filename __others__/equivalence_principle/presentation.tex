%\documentclass[aspectratio=169]{ctexbeamer}
\documentclass{ctexbeamer}
\usepackage{amsmath}
\usepackage{amssymb}
\usepackage{amsthm}
\usepackage{color}
\usepackage{graphicx}
\usepackage{geometry}
\usepackage{hyperref}
\usepackage{marginnote}
\usepackage{mathrsfs}
\usepackage{syntonly}
\usepackage{textcomp}
\usepackage{ulem}
\usepackage{verbatim}
%\syntaxonly
%\geometry{a5paper}
\hyphenation{}
\normalem
\hypersetup{
    colorlinks,
    linkcolor=blue,
    filecolor=pink,
    urlcolor=cyan,
    citecolor=red,
}
\def\b{\boldsymbol}
\def\d{\mathrm{d}}
\def\p{\partial}
\newcommand{\tabincell}[2]{\begin{tabular}{@{}#1@{}}#2\end{tabular}}
\DeclareMathOperator{\sgn}{sgn}
\DeclareMathOperator{\atanxy}{atan2}
\title{}
\author{}
\date{}
\begin{document}
\begin{frame}
    常量变化与EP违反的联系

    EP\footnote{lrr-2014-4}

    WEP\\
    1. The trajectory of a freely
    falling ``test'' body\footnote{one not acted upon by such forces as electromagnetism and too small to be
    affected by tidal gravitational forces} is independent of its internal structure and composition.
\end{frame}
\begin{frame}
    常量变化与EP违反的联系

    EP

    EEP\\
    1. The trajectories of freely
    falling test bodies are independent of its internal structure and composition (UFF).\\
    2. The outcome of any local non-gravitational experiment is independent of  where and when in
    the universe it is performed (LPI) and the velocity of the
    freely-falling reference frame in which it is performed (LLI).
\end{frame}
\begin{frame}
    常量变化与EP违反的联系

    EP

    SEP\\
    1. The trajectories of freely
    falling self-gravitating bodies as well as test bodies are independent of its internal structure and composition.\\
    2. The outcome of any local test experiment is independent of  where and when in
    the universe it is performed and the velocity of the
    freely-falling reference frame in which it is performed.
\end{frame}
\begin{frame}
    常量变化与EP违反的联系

    $G$变化导致UFF违反\\
    质点作用量为
    \begin{equation*}
        S=-\int mc\sqrt{-g_{\mu\nu}u^{\mu}u^{\nu}}\d t
    \end{equation*}
    $m$和常量有关, 由$S$得
    \begin{equation*}
        u^\nu\nabla_\nu u^\mu=\dots
    \end{equation*}
    如果常量不是常量, $m$就不是常量, 就有$\dots\ne 0$.\\
    如果$m$是个宏观物体, 有内势能要计入$m$, 内势能和$G$有关.
\end{frame}
\end{document}
