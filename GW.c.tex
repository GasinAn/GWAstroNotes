\chapter{引力波}

\cite{Wald1984}.

\section{Linearized Gravity}

流形$\mathbb{R}^{4}$. 任意坐标系$\{x^{\mu}\}$, $g_{\mu\nu}=\eta_{\mu\nu}+h_{\mu\nu}=\eta_{\mu\nu}+\gamma_{\mu\nu}s+\text{O}(s^2)$, 得
\begin{equation}
    R_{\mu\nu\lambda\sigma}=\p_\sigma\p_{[\mu}h_{\lambda]\nu}-\p_\nu\p_{[\mu}h_{\lambda]\sigma}+\text{O}(s^2).
\end{equation}
$\bar{h}_{\mu\nu}:=h_{\mu\nu}-\frac{1}{2}\eta_{\mu\nu}\eta^{\lambda\sigma}h_{\lambda\sigma}=h_{\mu\nu}-\frac{1}{2}\eta_{\mu\nu}h$.
\begin{equation}
    -\frac{1}{2} \partial^{\lambda} \partial_{\lambda} \bar{h}_{\mu \nu}+\partial^{\lambda} \partial_{(\mu} \bar{h}_{\nu) \lambda}-\frac{1}{2} \eta_{\mu \nu} \partial^{\lambda} \partial^{\sigma} \bar{h}_{\lambda \sigma}+\text{O}(s^2)=8 \pi T_{\mu \nu}.
\end{equation}
存在$\{x^{\mu}\}$, 使得$\p^{\nu}\bar{h}_{\mu\nu}+\text{O}(s^2)=0$ (Lorentz gauge). 
令$\{x^{\mu}\}$满足$\p^{\nu}\bar{h}_{\mu\nu}+\text{O}(s^2)=0$, 则
\begin{equation}
    \p^{\lambda}\p_{\lambda}\bar{h}_{\mu\nu}+\text{O}(s^2)=-16\pi T_{\mu \nu}.\label{lin_gravity}
\end{equation}
略去$\text{O}(s^2)$条件: $h_{\mu\nu}$, $\p_\lambda h_{\mu\nu}$\dots{}小.

\section{Radiation Gauge}

存在$\{x^{\mu}\}$, 使得$h+\text{O}(s^2)=0$ (TT gauge \cite{Wang2020})且$h_{0\mu}+\text{O}(s^2)=0$. 

\section{Quadrupole Approximation}

下略$\text{O}(s^2)$. 由\eqref{lin_gravity}得
\begin{equation}
    \bar{h}_{\mu\nu}(t,\vec{r}) = 4\int 
    \frac{T_{\mu\nu}(t-\left\lvert\vec{r}-\vec{r}'\right\rvert,\vec{r}')}{\left\lvert\vec{r}-\vec{r}'\right\rvert}\,\d V'.
\end{equation}
\begin{align}
    \hat{\bar{h}}_{\mu\nu}(\omega,\vec{r})&:=\frac{1}{\sqrt{2\pi}}
    \int\bar{h}_{\mu\nu}(t,\vec{r})e^{i\omega t}\d t\\
    &=4\int 
    \frac{\hat{T}_{\mu\nu}(\omega,\vec{r}')}{\left\lvert\vec{r}-\vec{r}'\right\rvert}e^{i\omega\left\lvert\vec{r}-\vec{r}'\right\rvert}\,\d V'.
\end{align}
由$\p^{\nu}\bar{h}_{\mu\nu}=0$,
\begin{equation}
    -i\omega\hat{\bar{h}}_{0\mu}=\sum_{i}\frac{\p\hat{\bar{h}}_{i\mu}}{\p x^{i}}.
\end{equation}
$\left\lvert \vec{r}\right\rvert \gg \left\lvert \vec{r}'\right\rvert$且$\omega\ll1/\left\lvert \vec{r}'\right\rvert$,
\begin{equation}
    \hat{\bar{h}}_{ij}(\omega,\vec{r}) = 4
    \frac{e^{i\omega\left\lvert\vec{r}\right\rvert}}{\left\lvert\vec{r}\right\rvert}
    \int \hat{T}_{ij}(\omega,\vec{r}')\,\d V'.
\end{equation}
\begin{align}
    \int \hat{T}_{ij}\,\d V'
    &= -\frac{\omega^2}{2}\int \hat{T}_{00}\,x'^ix'^j\,\d V'?
\end{align}
\begin{equation}
    q_{ij}(t):=\int{T}_{00}\,x'^ix'^j\,\d V'
\end{equation}
\begin{equation}
    \hat{\bar{h}}_{ij}(\omega,\vec{r}) = -2\omega^2\frac{e^{i\omega\left\lvert\vec{r}\right\rvert}}{\left\lvert\vec{r}\right\rvert}\hat{q}_{ij}(\omega),
\end{equation}
\begin{equation}
    {\bar{h}}_{ij}(t,\vec{r}) = \frac{2}{\left\lvert\vec{r}\right\rvert}
    \frac{\d^2}{\d t^2}{q}_{ij}(t-\left\lvert\vec{r}\right\rvert).\label{qf}
\end{equation}

\section{$+$ Mode and $\times$ Mode}

寻新标架$(e'^1)_a=(e^+)_a$, $(e'^2)_a=(e^\times)_a$, $(e'^3)_a=(e^r)_a$, ${\bar{h}}_{ij}(e^i)_a(e^j)_b={\bar{h}}'_{ij}(e'^i)_a(e'^j)_b$, 取$x$, $y$分量后去迹, $h_+=\frac{1}{2}({\bar{h}}'_{11}-{\bar{h}}'_{22})$, $h_\times={\bar{h}}'_{12}={\bar{h}}'_{21}$? \cite{Sathyaprakash2009}

\cite{Blanchet1997}, $\vec{n}:=\frac{\vec{r}}{\left\lvert\vec{r}\right\rvert}$,
\begin{equation}
    h_{ij}^{\text{TT}}=\frac{2}{\left\lvert\vec{r}\right\rvert}\mathcal{P}_{ijkm}\frac{\d^2}{\d t^2}{Q}^{km}(t-\left\lvert\vec{r}\right\rvert), \label{TT}
\end{equation}
\begin{equation}
    \mathcal{P}_{ijkm}:=
    \left(\delta_{ik} -\vec{n}_i\vec{n}_k\right)
    \left(\delta_{jm} -\vec{n}_j\vec{n}_m\right)
    -\frac{1}{2}
    \left(\delta_{ij} -\vec{n}_i\vec{n}_j\right)
    \left(\delta_{km} -\vec{n}_k\vec{n}_m\right),
\end{equation}
\begin{equation}
    Q^{km}(t):=\int{T}_{00}\,\left(x'^kx'^m-\frac{1}{3}\delta^{km}\sum_nx'^nx'^n\right)\,\d V'
\end{equation}
